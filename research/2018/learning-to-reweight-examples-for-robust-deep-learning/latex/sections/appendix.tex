% !TEX root = ../main.tex
\appendix

\section{Reweighting in an MLP}
\label{sec:mlp_derive}
We show the complete derivation below on calculating the example weights in an MLP network.
\begin{align}
&\frac{\partial}{\partial \epsilon_{i,t}} \EE{f^v(\theta_{t+1}(\epsilon))}
\Bigr|_{\epsilon_{i,t}=0}\\
=&\frac{1}{m}\sum_{j=1}^m \frac{\partial}{\partial \epsilon_{i,t}} f_j^v(\theta_{t+1}(\epsilon))
\Bigr|_{\epsilon_{i,t}=0}\\
=&\frac{1}{m}\sum_{j=1}^m \frac{\partial f_j^v(\theta)}{\partial \theta}\Bigr|_{\theta=\theta_t}^\top
\frac{\partial \theta_{t+1}(\epsilon_{i,t})} {\partial \epsilon_{i,t}}\Bigr|_{\epsilon_{i,t}=0}\\
\propto&-\frac{1}{m}\sum_{j=1}^m \frac{\partial f_j^v(\theta)}{\partial \theta}\Bigr|_{\theta=\theta_t}^\top
\frac{\partial f_i(\theta)}{\partial \theta}\Bigr|_{\theta=\theta_t}\\
=&-\frac{1}{m}\sum_{j=1}^m \sum_{l=1}^L
\frac{\partial f_j^v}{\partial \theta_l}\Bigr|_{\theta_l=\theta_{l,t}}^\top
\frac{\partial f_i}{\partial \theta_{l}}\Bigr|_{\theta_l=\theta_{l,t}}\\
=&-\frac{1}{m}\sum_{j=1}^m \sum_{l=1}^L
\vecc \left( \tilde{z}_{j,l-1}^v {g_{j,l}^v}^\top \right)^\top
\vecc \left( \tilde{z}_{i,l-1} g_{i,l}^\top \right)\\
=&-\frac{1}{m}\sum_{j=1}^m \sum_{l=1}^L \sum_{p=1}^{D_1} \sum_{q=1}^{D_2}
\tilde{z}_{j,l-1,p}^v g_{j,l,q}^v \tilde{z}_{i,l-1,p} g_{i,l,q}\\
=&-\frac{1}{m}\sum_{j=1}^m \sum_{l=1}^L
\sum_{p=1}^{D_1} \tilde{z}_{j,l-1,p}^v \tilde{z}_{i,l-1,p}
\sum_{q=1}^{D_2} g_{j,l,q}^v g_{i,l,q}\\
=&-\frac{1}{m}\sum_{j=1}^m \sum_{l=1}^L
(\tilde{z}^v_{j,l-1}{}^\top
\tilde{z}_{i,l-1})
(g^v_{j,l}{}^\top g_{i,l}).
\end{align}

\section{Convergence of our method}
\label{sec:lemproof}
This section provides the proof for Lemma~\ref{lem:convergence}.

\begin{lem*}
\input{sections/lemma1_appendix}
\end{lem*}

\begin{proof}
Suppose we have a small validation set with $M$ clean data $\{x_1, x_2, \cdots, x_M\}$, each
associating with a validation loss function $f_i(\theta)$, where $\theta$ is the parameter of the
model. The overall validation loss would be,
\begin{align}
  G(\theta) = \frac{1}{M} \sum_{i=1}^M f_i(\theta).
\end{align}
Now, suppose we have another $N-M$ training data, $\{x_{M+1}, x_{M+2}, \cdots, x_N\}$, and we add
those validation data into this set to form our large training dataset $T$, which has $N$ data in
total. The overall training loss would be,
\begin{align}
  F(\theta) = \frac{1}{M} \sum_{i=1}^N f_i(\theta).
\end{align}

For simplicity, since $M \ll N$, we assume that the validation data is a subset of the training
data. During training, we take a mini-batch $B$ of training data at each step, and $|B| = n$.
Following some similar derivation as Appendix \ref{sec:mlp_derive}, we have the following update
rules:
\begin{align}
\theta_{t+1} = \theta_t - \frac{\alpha_t}{n} \sum_{i \in B} \max \left\{\nabla G^\top \nabla f_i, 0
\right\} \nabla f_i,
\label{eq:updaterule}
\end{align}
where $\alpha_t$ is the learning rate at time-step $t$. Since all gradients are taken at $\theta_t$,
we omit $\theta_t$ in our notations.

Since the validation loss $G(\theta)$ is Lipschitz-smooth, we have
\begin{align}
G(\theta_{t+1}) \leq G(\theta_t) + \nabla G^\top\Delta \theta + \frac{L}{2} \lVert \Delta \theta \rVert^2.
\end{align}

Plugging our updating rule (Eq. \ref{eq:updaterule}),
\begin{align}
G(\theta_{t+1}) \leq G(\theta_t) - I_1 + I_2,
\end{align}
where,
\begin{align}
%\begin{split}
I_1 &=  \frac{\alpha_t}{n} \sum_{i \in B} \max \{\nabla G^\top \nabla f_i, 0 \}\nabla G^\top \nabla f_i\\
&=  \frac{\alpha_t}{n} \sum_{i \in B} \max \{\nabla G^\top \nabla f_i, 0 \}^2,\\
%\end{split}
\end{align}
and,
\begin{align}
I_2 &= \frac{L}{2} \lVert \frac{\alpha_t}{n} \sum_{i \in B} \max \{\nabla
G^\top \nabla f_i, 0 \}\nabla f_i \rVert^2\\
    &\le  \frac{L}{2} \frac{\alpha^2_t}{n^2} \sum_{i \in B} \lVert
          \max \left \{\nabla G^\top \nabla f_i, 0 \right \}\nabla f_i \right \rVert^2\label{eq:firstineq}\\
    &=    \frac{L}{2} \frac{\alpha^2_t}{n^2} \sum_{i \in B}
          \max \left \{\nabla G^\top \nabla f_i, 0 \right \}^2
          \left \lVert \nabla f_i \rVert^2\\
    &\le  \frac{L}{2} \frac{\alpha^2_t}{n^2} \sum_{i \in B}
          \max \left \{\nabla G^\top \nabla f_i, 0 \right \}^2 \sigma^2\label{eq:secondieq}.
\end{align}

The first inequality (Eq.~\ref{eq:firstineq}) comes from the triangle inequality. The second
inequality (Eq.~\ref{eq:secondieq}) holds since $f_i$ has $\sigma$-bounded gradients. If we denote
$\mathcal{T}_t = \sum_{i \in B} \max
\{\nabla G^\top \nabla f_i, 0 \}^2$, where $t$ stands for the time-step $t$, then
\begin{align}
\label{eq:decreasingvalid}
G(\theta_{t+1}) \leq G(\theta_t) - \frac{\alpha_t}{n} \mathcal{T}_t
                     \left(1 - \frac{L \alpha_t \sigma^2}{2 n } \right).
\end{align}

Note that by definition, $\mathcal{T}_t$ is non-negative, and since $\alpha_t \le
\frac{2n}{L\sigma^2}$, if follows that that $G(\theta_{t+1}) \le G(\theta_t)$ for any $t$.

Next, we prove $\EEsub{\mathcal{T}_t}{t}= 0$ if and only if $\nabla G = 0$, and
$\EEsub{\mathcal{T}_t}{t} > 0$ if and only if $\nabla G \neq 0$, where the expectation is taken over
all possible training batches at time step $t$. %For simplicity, we assume the training batch size $n$
%is equal to the validation size $M$. For other cases, the proof will be similiar.
%Given this assumption, there is a non-zero probability
%$p$ that the sampled training batch is happened to be the validation set $\{x_1, x_2, \cdots,
%x_M\}$.
It is obvious that when $\nabla G = 0$, $\EEsub{\mathcal{T}_t}{t} = 0$. If $\nabla G \neq
0$, from the inequality below, we firstly know that there must exist a validation example $x_{j, 0 \leq j \leq M}$ such
that $\nabla G^\top \nabla f_j > 0$,
\begin{align}
\label{eq:positivedotprodexist}
0 < \lVert \nabla G \rVert^2 = \nabla G^\top \nabla G = \frac{1}{M} \sum_{i=1}^M \nabla G^\top \nabla f_i.
\end{align}
Secondly, there is a non-zero possibility $p$ to sample a training batch $B$ such that it contains this data $x_j$.
Also noticing that $\mathcal{T}_t$ is a non-negative random variable, we have,
\begin{align}
\label{eq:EETlargerzero}
%\begin{split}
\EEsub{\mathcal{T}_t}{t} &\geq p \sum_{i \in B} \max\{\nabla G^\top \nabla f_i, 0\}^2\\
&\geq p \max\{\nabla G^\top \nabla f_j, 0\}^2 \\
&= p \left( \nabla G^\top \nabla f_j \right)^2 > 0.
%\end{split}
\end{align}

Therefore, if we take expectation over the training batch on both sides of Eq. \ref{eq:decreasingvalid},
we can conclude that,
\begin{align}
\EEsub{G(\theta_{t+1})}{t} \leq G(\theta_{t}),
\end{align}
where the equality holds if and only if $\nabla G = 0$. This finishes our proof for Lemma 1.
\end{proof}

\section{Convergence rate of our method}
\label{sec:thmproof}
This section provides proof for Theorem 2.

\begin{thm*}
% !TEX root = ../main.tex
Suppose $G$, $f_i$ and $\alpha_t$ satisfy the aforementioned conditions, then
Algorithm~\ref{alg:ad} achieves $\EE{\lVert \nabla G(\theta_t) \rVert^2} \leq \epsilon$ in
$O(1/\epsilon^2)$ steps. More specifically,
\begin{align}
\min\limits_{0 < t < T} \EE{\lVert \nabla G(\theta_t) \rVert^2} \leq \frac{C}{\sqrt{T}},
\end{align}
where $C$ is some constant independent of the convergence process.

\end{thm*}

\begin{proof}
From the proof of Lemma~\ref{lem:convergence}, we have
\begin{align}
%\begin{split}
&\frac{\alpha_t}{n} \left(1 - \frac{L \alpha_t \sigma^2}{2n} \right) \EEsub{\mathcal{T}_t}{0 \sim t} \\
\leq& \EEsub{G(\theta_t)}{0 \sim t-1} - \EEsub{G(\theta_{t+1})}{0 \sim t}.
%\end{split}
\end{align}

If we let $\alpha_t$ to be a constant $\alpha < \frac{2n}{L\sigma^2}$ (or a decay positive sequence
upper bounded by $\alpha$), and let $\kappa = \left(1 - \frac{L \alpha \sigma^2}{2n} \right)\alpha/n
> 0$, then we have,
\begin{align}
%\begin{split}
 \kappa \sum_{t=0}^T \EEsub{\mathcal{T}_t}{0 \sim t} &\leq \EEsub{G(\theta_0)}{0} - \EEsub{G(\theta_{T+1})}{0 \sim T}\\
  &\leq G(\theta_0) - G(\theta^*),
%\end{split}
\end{align}

where $G(\theta^*)$ is the global minimum of function $G$. Therefore, it is obvious to see that
there exist a time-step $ 0 \leq \tau \leq T$ such that,
\begin{align}
\EEsub{\mathcal{T}_{\tau}}{0 \sim \tau} \leq \frac{G(\theta_0) - G(\theta^*)}{\kappa T}.
\end{align}

We next prove that for this time-step $\tau$, the gradient square $\EEsub{\lVert \nabla G(\theta_{\tau})
\rVert^2}{0 \sim \tau-1}$ is smaller than $O(1/\sqrt{T})$. %We know that there is a positive
%probability $p$ that the sampled training batch is happened to be the validation set $\{x_1, x_2,
%\cdots, x_M\}$, and this probability $p$ is only determined by constants $M$, $N$, and $n$.
Considering such $M$ training batches $B_1, B_2, \cdots, B_M$ such that $B_i$ is guaranteed to contain $x_i$. We know that those batches have non-zero sampling probability, denoted as $p_1, p_2, \cdots, p_M$. We also denote $p = \min\{p_1, p_2, \cdots, p_M\}$.
Now, we have,
\begin{align}
M\EEsub{\mathcal{T}_{\tau}}{0 \sim \tau}
&= \EEsub{M\EEsub{\mathcal{T}_{\tau}}{\tau}}{0 \sim \tau-1}\\
&\geq \EEsub{\sum_{k = 1}^M p_k \sum_{i \in B_k} \max\{\nabla G^\top \nabla f_i, 0\}^2}{0 \sim \tau-1} \label{ineq:first} \\
&\geq p\EEsub{\sum_{k = 1}^M \sum_{i \in B_k} \max\{\nabla G^\top \nabla f_i, 0\}^2}{0 \sim \tau-1} \\
&\geq p\EEsub{\sum_{i=1}^M \max\{\nabla G^\top \nabla f_i, 0\}^2}{0 \sim \tau-1} \\
&= p \sum_{i=1}^M \EEsub{\max\{\nabla G^\top \nabla f_i, 0\}^2}{0 \sim \tau-1}       \\
&\geq p \sum_{i=1}^M \left (\EEsub{\max\{\nabla G^\top \nabla f_i, 0\}}{0 \sim \tau-1} \right)^2 \label{ineq:second}     \\
&\geq \frac{p}{M} \left ( \sum_{i=1}^M \EEsub{\max\{\nabla G^\top \nabla f_i, 0\}}{0 \sim \tau-1} \right)^2\label{ineq:third}.
\end{align}
The inequality in Eq.~\ref{ineq:first} comes from the non-negativeness of $\mathcal{T}_t$, the
inequality in Eq.~\ref{ineq:second} comes from the property of expectation, and the final inequality
in Eq.~\ref{ineq:third} comes from the Cauchy-Schwartz inequality. Therefore,
\begin{align}
%\begin{split}
&\sum_{i=1}^M \EEsub{\max\{\nabla G^\top \nabla f_i, 0\}}{0 \sim \tau-1} \\
\leq& M\sqrt{\frac{(G(\theta_0) - G(\theta^*))}{p \kappa}} \sqrt{\frac{1}{T}},
%\end{split}
\end{align}
and so,
\begin{align}
\label{eq:finalequa}
%\begin{split}
&\EEsub{\lVert \nabla G(\theta_{\tau}) \rVert^2}{0 \sim \tau-1}\\
=& \EEsub{\nabla G^\top \nabla G}{0 \sim \tau-1}\\
=& \EEsub{\nabla G^\top \left( \frac{\sum_{i=0}^M \nabla f_i}{M} \right)}{0 \sim \tau-1}\\
\leq&  \frac{1}{M} \sum_{i=1}^M \EEsub{\max\{\nabla G^\top \nabla f_i, 0\}}{0 \sim \tau-1}\\
\leq&  \sqrt{\frac{G(\theta_0) - G(\theta^*)}{p \kappa}} \sqrt{\frac{1}{T}}.
%\end{split}
\end{align}
Therefore, we can conclude that conclude that our algorithm can always achieve $\min \limits_{0 < t < T} \EE{\lVert
\nabla G(\theta_t)\rVert^2} \leq O(\sqrt{1/T})$ in $T$ steps, and this finishes our proof of
Theorem~\ref{thm:convergencerate}.
\end{proof}
