% !TEX root = ../main.tex
\begin{abstract}
In few-shot classification, we are interested in learning algorithms that train a classifier from
only a handful of labeled examples. Recent progress in few-shot classification has featured 
meta-learning, in which a parameterized model for a learning algorithm is defined and trained on episodes
representing different classification problems, each with a small labeled training set and its
corresponding test set. In this work, we advance this few-shot classification paradigm towards a
scenario where unlabeled examples are also available within each episode. We consider two
situations: one where all unlabeled examples are assumed to belong to the same set of classes as the
labeled examples of the episode, as well as the more challenging situation where examples from other
{\it distractor} classes are also provided. To address this paradigm, we propose novel extensions of
Prototypical Networks~\citep{snell2017protonet} that are augmented with the ability to use unlabeled
examples when producing prototypes. These models are trained in an end-to-end way on episodes, to
learn to leverage the unlabeled examples successfully. We evaluate these methods on versions of the
Omniglot and {\it mini}ImageNet benchmarks, adapted to this new framework augmented with unlabeled
examples. We also propose a new split of ImageNet, consisting of a large set of classes, with a
hierarchical structure. Our experiments confirm that our Prototypical Networks can learn to improve
their predictions due to unlabeled examples, much like a semi-supervised algorithm would.
% \blfootnote{{Code available at \url{https://github.com/renmengye/few-shot-ssl-public}}} 
\end{abstract}
