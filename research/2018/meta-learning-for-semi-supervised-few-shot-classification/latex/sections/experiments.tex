% !TEX root = ../main.tex
\section{Experiments}

\subsection{Datasets}

We evaluate the performance of our model on three datasets: two benchmark few-shot classification
datasets and a novel large-scale dataset that we hope will be useful for future few-shot learning
work.

\textbf{Omniglot} \citep{lake2011oneshot} is a dataset of 1,623 handwritten characters from 50
alphabets. Each character was drawn by 20 human subjects. We follow the few-shot setting proposed by
\citet{vinyals2016matchingnet}, in which the images are resized to $28 \times 28$ pixels and
rotations in multiples of 90$^\circ$ are applied, yielding 6,492 classes in total. These are split
into 4,112 training classes, 688 validation classes, and 1,692 testing classes.

\textbf{\textit{mini}ImageNet} \citep{vinyals2016matchingnet} is a modified version of the ILSVRC-12
dataset \citep{russakovsky2015imagenet}, in which 600 images for each of 100 classes were randomly
chosen to be part of the dataset. We rely on the class split used by \citet{ravi2017oneshot}. These
splits use 64 classes for training, 16 for validation, and 20 for test. All images are of size 84
$\times$ 84 pixels.

\textbf{\textit{tiered}ImageNet} is our proposed dataset for few-shot classification. Like
\textit{mini}Imagenet, it is a subset of ILSVRC-12. However, \textit{tiered}ImageNet represents a
larger subset of ILSVRC-12 (608 classes rather than 100 for \textit{mini}ImageNet). Analogous to
Omniglot, in which characters are grouped into alphabets, \textit{tiered}ImageNet groups classes
into broader categories corresponding to higher-level nodes in the ImageNet \citep{deng2009imagenet}
hierarchy. There are 34 categories in total, with each category containing between 10 and 30
classes. These are split into 20 training, 6 validation and 8 testing categories (details of the
dataset can be found in the supplementary material). This ensures that all of the training classes
are sufficiently distinct from the testing classes, unlike \textit{mini}ImageNet and other
alternatives such as \textit{rand}ImageNet proposed by  \citet{vinyals2016matchingnet}. For example,
``pipe organ'' is a training class and ``electric guitar'' is a test class in the
\citet{ravi2017oneshot} split of  \textit{mini}Imagenet, even though they are both musical
instruments. This scenario would not occur in \textit{tiered}ImageNet since ``musical instrument''
is a high-level category and as such is not split between training and test classes. This represents
a more realistic few-shot learning scenario since in general we cannot assume that test classes will
be similar to those seen in training. Additionally, the tiered structure of \textit{tiered}ImageNet
may be useful for few-shot learning approaches that can take advantage of hierarchical relationships
between classes. We leave such interesting extensions for future work.

\subsection{Adapting the Datasets for Semi-Supervised Learning}
For each dataset, we first create an additional split to separate the images of each class into
disjoint labeled and unlabeled sets. For Omniglot and {\it tiered}ImageNet we sampled 10\% of the
images of each class to form the labeled split. The remaining 90\% can only be used in the unlabeled
portion of episodes. For {\it mini}ImageNet we instead used 40\% of the data for the labeled split
and the remaining 60\% for the unlabeled, since we noticed that 10\% was too small to achieve
reasonable performance and avoid overfitting. We report the average classification scores over 10
random splits of labeled and unlabeled portions of the training set, with uncertainty computed in
standard error (standard deviation divided by the square root of the total number of splits).

We would like to emphasize that due to this labeled/unlabeled split, we are using strictly less
label information than in the previously-published work on these datasets. Because of this, we do
not expect our results to match the published numbers, which should instead be interpreted as an
upper-bound for the performance of the semi-supervised models defined in this work.

Episode construction then is performed as follows. For a given dataset, we create a training episode
by first sampling $N$ classes uniformly at random from the set of training classes ${\cal C}_{\rm
train}$. We then sample $K$ images from the labeled split of each of these classes to form the
support set, and $M$ images from the unlabeled split of each of these classes to form the unlabeled
set. Optionally, when including distractors, we additionally sample $H$ other classes from the set
of training classes and $M$ images from the unlabeled split of each to act as the distractors. These
distractor images are added to the unlabeled set along with the unlabeled images of the $N$ classes
of interest (for a total of $MN + MH$ images). The query portion of the episode is comprised of a
fixed number of images from the labeled split of each of the $N$ chosen classes. Test episodes are
created analogously, but with the $N$ classes (and optionally the $H$ distractor classes) sampled
from ${\cal C}_{\rm test}$. Note that we used $M=5$ for training and $M=20$ for testing, thus also
measuring the ability of the models to generalize to a larger unlabeled set size. We also used
$H=N=5$, i.e.\ used 5 classes for both the labeled classes and the disctractor classes.

In each dataset we compare our three semi-supervised models with two baselines. The first baseline,
referred to as ``Supervised'' in our tables, is an ordinary Prototypical Network that is trained in
a purely supervised way on the labeled split of each dataset. The second baseline, referred to as
``Semi-Supervised Inference'', uses the embedding function learned by this supervised Prototypical
Network, but performs semi-supervised refinement of the prototypes at inference time using a step of
Soft $k$-Means refinement. This is to be contrasted with our semi-supervised models that perform
this refinement both at training time and at test time, therefore learning a different embedding
function. We evaluate each model in two settings: one where all unlabeled examples belong to the
classes of interest, and a more challenging one that includes distractors. Details of the model
hyperparameters can be found in Appendix~\ref{sec:hyperparam} and our online repository\footnote{
Code available at
\url{https://github.com/renmengye/few-shot-ssl-public}}.

% !TEX root = marvin.tex
  \begin{table*}[t]
  \begin{center}
    \centering
      \resizebox{0.9\textwidth}{!}{%
      \begin{tabular}{c|ccc|ccc|ccc|ccc}
      \toprule
                  & \multicolumn{3}{c}{n = 25, 1 Agent}                & \multicolumn{3}{c}{n = 25, 2 Agents}                   & \multicolumn{3}{c}{n = 50, 2 Agents}            & \multicolumn{3}{c}{n = 100, 5 Agents}              \\
      Method      & Cost          & Gap             & Runtime          & Cost           & Gap             & Runtime             & Cost          & Gap             & Runtime       & Cost          & Gap             & Runtime          \\
      \midrule
      \midrule
      Oracle      & 1.16          & 0.00\%          & 71.3             & 1.28           & 0.00\%          & 438                 & 1.85          & 0.00\%          & 902           & 3.19          & 0.00\%          & 2430             \\
      \midrule
      Random      & 4.45          & 284.9\%         & \textbf{3.15}    & 4.47           & 249.4\%         & \textbf{1.50}       & 8.25          & 345.2\%         & \textbf{1.82} & 18.9          & 492.2\%         & \textbf{2.83}    \\
      Greedy      & 2.12          & 73.7\%          & 3.37             & 2.33           & 81.8\%          & 2.11                & 3.55          & 91.5\%          & 3.57          & 10.4          & 227.0\%         & 22.3             \\
      LKH3        & 1.26          & 8.84\%          & 71.2             & 1.80           & 40.5\%          & 438                 & 2.54          & 37.3\%          & 902           & 6.14          & 92.5\%          & 2430             \\
      \midrule
      GVIN        & 1.37          & 18.8\%          & 52.5             & 1.48           & 15.9\%          & 44.2                & 2.45          & 32.1\%          & 63.4          & 5.41          & 69.6\%          & 48.6             \\
      GAT         & 1.53          & 32.5\%          & 43.0             & 1.56           & 21.6\%          & 29.1                & 2.58          & 39.7\%          & 38.0          & 5.43          & 70.2\%          & 38.2             \\
      AM          & 4.90          & 322.4\%         & 161              & -              & -               & -                   & -             & -               & -             & -             & -               & -                \\
      EAN         & 2.89          & 145.8\%         & 212              & -              & -               & -                   & -             & -               & -             & -             & -               & -                \\
      MARVIN (IL) & 1.37          & 18.0\%          & 62.8             & 1.42           & 11.3\%          & 66.6                & 2.21          & 19.0\%          & 71.5          & \textbf{4.36} & \textbf{36.7\%} & 72.8 \\
      MARVIN (RL) & \textbf{1.25} & \textbf{8.17\%} & 62.8             & \textbf{1.32}  & \textbf{2.87\%} & 56.6                & \textbf{2.12} & \textbf{14.5\%} & 71.4          & 4.62          & 44.9\%          & 72.8 \\
      \bottomrule
      \end{tabular}%
      }
      \caption{Average graph traversal cost on realistic graphs; Time cost in hours; Runtime in milliseconds.}
        \label{tab:1}
        \vspace{-0.2in}
        \end{center}
    \end{table*}


