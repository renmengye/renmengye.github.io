% !TEX root = ../main.tex
\section{Experiments}

\subsection{Datasets}

We evaluate the performance of our model on three datasets: two benchmark few-shot classification
datasets and a novel large-scale dataset that we hope will be useful for future few-shot learning
work.

\textbf{Omniglot} \citep{lake2011oneshot} is a dataset of 1,623 handwritten characters from 50
alphabets. Each character was drawn by 20 human subjects. We follow the few-shot setting proposed by
\citet{vinyals2016matchingnet}, in which the images are resized to $28 \times 28$ pixels and
rotations in multiples of 90$^\circ$ are applied, yielding 6,492 classes in total. These are split
into 4,112 training classes, 688 validation classes, and 1,692 testing classes.

\textbf{\textit{mini}ImageNet} \citep{vinyals2016matchingnet} is a modified version of the ILSVRC-12
dataset \citep{russakovsky2015imagenet}, in which 600 images for each of 100 classes were randomly
chosen to be part of the dataset. We rely on the class split used by \citet{ravi2017oneshot}. These
splits use 64 classes for training, 16 for validation, and 20 for test. All images are of size 84
$\times$ 84 pixels.

\textbf{\textit{tiered}ImageNet} is our proposed dataset for few-shot classification. Like
\textit{mini}Imagenet, it is a subset of ILSVRC-12. However, \textit{tiered}ImageNet represents a
larger subset of ILSVRC-12 (608 classes rather than 100 for \textit{mini}ImageNet). Analogous to
Omniglot, in which characters are grouped into alphabets, \textit{tiered}ImageNet groups classes
into broader categories corresponding to higher-level nodes in the ImageNet \citep{deng2009imagenet}
hierarchy. There are 34 categories in total, with each category containing between 10 and 30
classes. These are split into 20 training, 6 validation and 8 testing categories (details of the
dataset can be found in the supplementary material). This ensures that all of the training classes
are sufficiently distinct from the testing classes, unlike \textit{mini}ImageNet and other
alternatives such as \textit{rand}ImageNet proposed by  \citet{vinyals2016matchingnet}. For example,
``pipe organ'' is a training class and ``electric guitar'' is a test class in the
\citet{ravi2017oneshot} split of  \textit{mini}Imagenet, even though they are both musical
instruments. This scenario would not occur in \textit{tiered}ImageNet since ``musical instrument''
is a high-level category and as such is not split between training and test classes. This represents
a more realistic few-shot learning scenario since in general we cannot assume that test classes will
be similar to those seen in training. Additionally, the tiered structure of \textit{tiered}ImageNet
may be useful for few-shot learning approaches that can take advantage of hierarchical relationships
between classes. We leave such interesting extensions for future work.

\subsection{Adapting the Datasets for Semi-Supervised Learning}
For each dataset, we first create an additional split to separate the images of each class into
disjoint labeled and unlabeled sets. For Omniglot and {\it tiered}ImageNet we sampled 10\% of the
images of each class to form the labeled split. The remaining 90\% can only be used in the unlabeled
portion of episodes. For {\it mini}ImageNet we instead used 40\% of the data for the labeled split
and the remaining 60\% for the unlabeled, since we noticed that 10\% was too small to achieve
reasonable performance and avoid overfitting. We report the average classification scores over 10
random splits of labeled and unlabeled portions of the training set, with uncertainty computed in
standard error (standard deviation divided by the square root of the total number of splits).

We would like to emphasize that due to this labeled/unlabeled split, we are using strictly less
label information than in the previously-published work on these datasets. Because of this, we do
not expect our results to match the published numbers, which should instead be interpreted as an
upper-bound for the performance of the semi-supervised models defined in this work.

Episode construction then is performed as follows. For a given dataset, we create a training episode
by first sampling $N$ classes uniformly at random from the set of training classes ${\cal C}_{\rm
train}$. We then sample $K$ images from the labeled split of each of these classes to form the
support set, and $M$ images from the unlabeled split of each of these classes to form the unlabeled
set. Optionally, when including distractors, we additionally sample $H$ other classes from the set
of training classes and $M$ images from the unlabeled split of each to act as the distractors. These
distractor images are added to the unlabeled set along with the unlabeled images of the $N$ classes
of interest (for a total of $MN + MH$ images). The query portion of the episode is comprised of a
fixed number of images from the labeled split of each of the $N$ chosen classes. Test episodes are
created analogously, but with the $N$ classes (and optionally the $H$ distractor classes) sampled
from ${\cal C}_{\rm test}$. Note that we used $M=5$ for training and $M=20$ for testing, thus also
measuring the ability of the models to generalize to a larger unlabeled set size. We also used
$H=N=5$, i.e.\ used 5 classes for both the labeled classes and the disctractor classes.

In each dataset we compare our three semi-supervised models with two baselines. The first baseline,
referred to as ``Supervised'' in our tables, is an ordinary Prototypical Network that is trained in
a purely supervised way on the labeled split of each dataset. The second baseline, referred to as
``Semi-Supervised Inference'', uses the embedding function learned by this supervised Prototypical
Network, but performs semi-supervised refinement of the prototypes at inference time using a step of
Soft $k$-Means refinement. This is to be contrasted with our semi-supervised models that perform
this refinement both at training time and at test time, therefore learning a different embedding
function. We evaluate each model in two settings: one where all unlabeled examples belong to the
classes of interest, and a more challenging one that includes distractors. Details of the model
hyperparameters can be found in Appendix~\ref{sec:hyperparam} and our online repository\footnote{
Code available at
\url{https://github.com/renmengye/few-shot-ssl-public}}.

% !TEX root = top.tex
\section{Experiments}
In this section, we use our proposed GHN to search for the best CNN architecture for image
classification. First, we evaluate the GHN on the standard CIFAR \citep{krizhevsky2009cifar} and
ImageNet \citep{russakovsky2015imagenet} architecture search benchmarks. Next, we apply GHN on an
``anytime prediction'' task where we optimize the speed-accuracy tradeoff that is key for many
real-time applications. Finally, we benchmark the GHN's  predicted-performance correlation and
explore various factors in an ablation study.
% !TEX root = top.tex
\subsection{NAS benchmarks}
\begin{table}[t]
\vspace{-0.7cm}
\caption{Comparison with image classifiers found by state-of-the-art NAS methods which employ a random search on CIFAR-10. Results shown are mean $\pm$ standard deviation.}
\label{table:Results1}
\footnotesize
\begin{center}
\begin{tabular}{ c c c c } 
Method & Search Cost (GPU days) & Param $\times 10^6$ & Accuracy   \\ 
\hline
SMASHv1 \citep{brock2017smash} &? & 4.6 & 94.5 \\
SMASHv2 \citep{brock2017smash} & 3 & 16.0 & 96.0\\
One-Shot Top (F=32) \citep{bender2018understanding} & 4  & 2.7 $\pm$ 0.3 & 95.5 $\pm$ 0.1\\
One-Shot Top (F=64) \citep{bender2018understanding} & 4 & 10.4 $\pm$ 1.0 & 95.9 $\pm$ 0.2\\
\hline
\hline
Random (F=32) & - & 4.6 $\pm$ 0.6 & 94.6 $\pm$ 0.3\\ 
GHN Top (F=32) &  0.42  & 5.1 $\pm$ 0.6 & 95.7 $\pm$ 0.1\\ 
\end{tabular}
\end{center}
%\footnotesize{$^*$ \cite{bender2018understanding} 16 GPU hours training + 80 GPU hours evaluation. }
\end{table}
\begin{table}[t]
\vspace{-0.5cm}
\caption{Comparison with image classifiers found by state-of-the-art NAS methods which employ advanced search methods on CIFAR-10. Results shown are mean $\pm$ standard deviation.}
\label{table:Results2}
\vspace{-0.2cm}
\footnotesize
\begin{center}
\begin{tabular}{ c c c c } 
Method & Search Cost (GPU days) & Param $\times 10^6$ & Accuracy   \\ 
\hline
NASNet-A  \citep{zoph2017learning} & 1800 & 3.3 & 97.35 \\
%AmoebaNet-A  + CutOut \cite{real2018regularized} & 1 GPU, 0.45 days & 4.6 & 97.11 \\
%AmoebaNet-B + CutOut \cite{real2018regularized} & 1 GPU, 0.45 days & 4.6 & 97.11 \\
ENAS Cell search  \citep{pham2018efficient} & 0.45  & 4.6 & 97.11 \\
DARTS (first order)  \citep{liu2018darts} &  1.5  & 2.9  & 97.06  \\
DARTS (second order) \citep{liu2018darts} & 4  & 3.4 & 97.17 $\pm$ 0.06\\
\hline
\hline
GHN Top-Best, 1K (F=32) & 0.84  & 5.7 & 97.16 $\pm$ 0.07 \\
%GHN Top-Best, 20K (F=32) & 4.17 & 4.7 & 97.24 $\pm$ 0.05 \\
\end{tabular}
\end{center}
\end{table}

% !TEX root = top.tex
\begin{table}[t]
\vspace{-0.5cm}
\caption{Comparison with image classifiers found by state-of-the-art NAS methods which employ advanced search methods on ImageNet-Mobile.}
\vspace{-0.2cm}
\label{table:Results3}
\footnotesize
\begin{center}
\begin{tabular}{ c c c c } 
Method & Search Cost (GPU days) & Param $\times 10^6$ & Accuracy   \\ 
\hline
%One-Shot Top (F=24) \citep{bender2018understanding} & 3.3+? $^*$ & 6.8 $\pm$ 0.9 & 70.7 $\pm$ 0.6\\
%One-Shot Top (F=32) \citep{bender2018understanding} & 3.3+? $^*$ & 11.9 $\pm$ 1.5 & 72.6 $\pm$ 0.4\\
NASNet-A  \citep{zoph2017learning} & 1800 & 5.3 & 74.0 \\
%NASNet-B  \citep{zoph2017learning} & 1800 & 5.3 & 72.8 \\
NASNet-C  \citep{zoph2017learning} & 1800 & 4.9 & 72.5 \\
AmoebaNet-A \citep{real2018regularized} & 3150 & 5.1 & 74.5  \\
%AmoebaNet-B \citep{real2018regularized} & 3150 & 5.3 & 74.0  \\
AmoebaNet-C \citep{real2018regularized} & 3150 & 6.4 & 75.7  \\
PNAS \citep{liu2017progressive} & 225 & 	5.1 &  74.2\\
DARTS (second order) \citep{liu2018darts} & 4  & 4.9 & 73.1\\
\hline
\hline
GHN Top-Best, 1K & 0.84  & 6.1  & 73.0  \\
%GHN Top-Best, 20K & 4.17  & 5.0  & 73.2  \\
\end{tabular}
\end{center}
\vspace{-0.5cm}
\end{table}

\subsubsection{CIFAR-10}
\label{section:cifar10}
We conduct our initial set of experiments on CIFAR-10 \citep{krizhevsky2009cifar}, which contains 10
object classes and 50,000 training images and 10,000 test images of size 32$\times$32$\times$3. We
use 5,000 images split from the training set as our validation set.

\vspace{-0.25cm}
\paragraph{Search space:} 
Following existing NAS methods, we choose to search for optimal blocks rather than the entire
network. Each block contains 17 nodes, with 8 possible operations. The final architecture is formed
by stacking 18 blocks. The spatial size is halved and the number of channels is doubled after blocks
6 and 12. These settings are all chosen following recent NAS methods
\citep{zoph2016neural,pham2018efficient,liu2018darts}, with details in the Appendix.

\vspace{-0.25cm}
\paragraph{Training:}
For the GNN module, we use a standard GRU cell \citep{cho14gru} with hidden size 32 and 2
layer MLP with hidden size 32 as the recurrent cell function $U$ and message function $M$
respectively. The shared hypernetwork $H \left(\cdot; \vvphi\right)$ is a 2-layer MLP with hidden
size 64. From the results of ablations studies in Section~\ref{section:ablations}, the GHN is
trained with blocks with $N=7$ nodes and $T=5$ propagations under the forward-backward scheme, using
the ADAM optimizer \citep{kingma2015adam}. Training details of the final selected architectures are
chosen to follow existing works and can be found in the Appendix.
\vspace{-0.25cm}
\paragraph{Evaluation:}
First, we compare to similar methods that use random search with a  hypernetwork or a one-shot model
as a surrogate search signal. We randomly sample 10 architectures and train until convergence for
our random baseline. Next, we randomly sample 1000 architectures, and select the top 10 performing
architectures with GHN generated weights, which we refer to as GHN Top. Our reported search cost
includes both the GHN training and evaluation phase. Shown in Table~\ref{table:Results1}, the GHN
achieves competitive results with nearly an order of magnitude reduction in search cost.

In Table~\ref{table:Results2}, we compare with methods which use more advanced search methods, such
as reinforcement learning and evolution. Once again, we sample 1000 architectures and use the GHN to
select the top 10. To make a fair comparison for random search, we train the top 10 for a short
period before selecting the best to train until convergence. The accuracy reported for GHN Top-Best
is the average of 5 runs  of the same final architecture. Note that all methods in
Table~\ref{table:Results2} use CutOut~\citep{devriescutout17}. GHN achieves very competitive results
with a simple random search algorithm, while only using a fraction of the total search cost. Using
advanced search methods with GHNs may bring further gains.

\subsubsection{ImageNet-Mobile}
We also run our GHN algorithm on the ImageNet dataset \citep{russakovsky2015imagenet}, which
contains 1.28 million training images. We report the top-1  accuracy on the 50,000 validation
images. Following existing literature, we conduct the ImageNet experiments in the mobile setting,
where the model is constrained to be under 600M FLOPS. We directly transfer the best architecture
block found in the CIFAR-10 experiments, using an initial convolution layer of stride 2 before
stacking 14 blocks with scale reduction at blocks 1, 2, 6 and 10. The total number of flops is
constrained by choosing the initial number of channels. We follow existing NAS methods on the
training procedure of the final architecture; details can be found in the Appendix. As shown in
Table \ref{table:Results3} the transferred block is competitive with other NAS methods which require
a far greater search cost.
% !TEX root = top.tex
\subsection{Anytime Prediction}
In the real-time setting, the computational budget available can vary for each test case and cannot
be known ahead of time. This is formalized in anytime prediction, \citep{grubb2012speedboost}  the
setting in which for each test example $\rvx$, there is non-deterministic computational budget $B$
drawn from the joint distribution $P(\rvx, B)$. The goal is then to minimize the expected loss $L(f)
= \E\left[ L\left( f(\rvx), B \right)\right]_{P(\rvx, B)}$, where $f(\cdot)$ is the model and
$L(\cdot)$ is the loss for an $f(\cdot)$ that must produce a prediction within the budget $B$.

We conduct experiments on CIFAR-10. Our anytime search space consists of networks with 3 cells
containing 24, 16, and 8 nodes. Each node is given the additional properties: 1) the spatial size it
operates at and 2) if an early-exit classifier is attached to it. A node enforces its spatial size
by pooling or upsampling any input feature maps inputs that are of different scale. Note that while
a naive one-shot model would triple its size to include three different parameter sets at three
different scales, the GHN is negligibly affected by such a change. The GHN uses the area under the
predicted accuracy-FLOPS curve as its selection criteria. The search space, contains various
convolution and pooling operators. Training methodology of the final architectures are chosen to
match \cite{huang2017multi} and can be found in the Appendix.

Figure \ref{fig:test1} shows a comparison with the various methods presented by
\cite{huang2017multi}. Our experiments show that the best searched architectures can outperform the
current state-of-the-art human designed networks. We see the GHN is amenable to the changes proposed
above, and can find efficient architectures with a random search when used with a strong search
space.

\iflatexml
\begin{figure}
  \includegraphics[width=4\linewidth]{figures/anytime_compare.png}
  \captionof{figure}{Comparison with state-of-the-art\\ human-designed networks on CIFAR-10.}
  \label{fig:test1}
\end{figure}
\begin{figure}
  \includegraphics[width=4\linewidth]{figures/anytime_randoms.png}
  \captionof{figure}{Comparison between random 10 and\\ top 10 networks on CIFAR-10.}
  \label{fig:test2}
\end{figure}
\else
\begin{figure}[t]
\vspace{-0.5cm}
\centering
\begin{minipage}{.48\textwidth}
  \centering
  \includegraphics[width=0.8\linewidth]{figures/anytime_compare.pdf}
  \captionof{figure}{Comparison with state-of-the-art\\ human-designed networks on CIFAR-10.}
\label{table:Results4}
  \label{fig:test1}
\end{minipage}%
\begin{minipage}{.48\textwidth}
  \centering
  \includegraphics[width=0.8\linewidth]{figures/anytime_randoms.pdf}
  \captionof{figure}{Comparison between random 10 and\\ top 10 networks on CIFAR-10.}
  \label{fig:test2}
\end{minipage}
\end{figure}
\fi
% !TEX root = top.tex
\subsection{Predicted performance correlation (CIFAR-10)}
\begin{table}[t]
\caption{Benchmarking the correlation between the predicted and true performance of the GHN against SGD and a one-shot model baselines. Results are on CIFAR-10.}
\vspace{-0.2cm}
\label{table:correlation}
\small
\begin{center}
\begin{tabular}{ c c c c c} 
Method & \multicolumn{2}{c}{Computation cost}   & \multicolumn{2}{c}{Correlation}    \\ 
%\hline
 & Initial (GPU hours) & Per arch. (GPU seconds)  & Random-100 & Top-50   \\ 
\hline
SGD 10 Steps & - & 0.9 & 0.26 & -0.05\\
SGD 100 Steps & - & 9 & 0.59 & 0.06\\
SGD 200 Steps & - & 18 & 0.62 & 0.20 \\
SGD 1000 Steps & - & 90 & 0.77 & 0.26 \\
One-Shot & 9.8 & 0.06 & 0.58 & 0.31\\
\hline
\hline
GHN & 6.1 & 0.08 & 0.68 & 0.48
\end{tabular}
\end{center}
\end{table}

In this section, we evaluate  whether the parameters generated from GHN can be indicative of the
final performance. Our metric is the correlation between the accuracy of a model with trained
weights vs. GHN generated weights. We use a fixed set of 100 random architectures that have not been
seen by the GHN during training, and we train them for 50 epochs to obtain our ``ground-truth''
accuracy, and finally compare with the accuracy obtained from GHN generated weights. We report the
Pearson's R score on all 100 random architectures and the top 50 performing architectures (i.e.\
above average architectures). Since we are interested in searching for the best architecture,
obtaining a higher correlation on top performing architectures is more meaningful.

To evaluate the effectiveness of GHN, we further consider two baselines: 1) training a network with
SGD from scratch for a varying number of steps, and 2) our own implementation of the one-shot model
proposed by \citet{pham2018efficient}, where nodes store a set of shared parameters for each
possible operation. Unlike GHN, which is compatible with varying number of nodes, the one-shot model
must be trained with $N=17$ nodes to match the evaluation. The GHN is trained with $N=7$, $T=5$
using forward-backward propagation. These GHN parameters are selected based on the results found in
Section~\ref{section:ablations}.

Table \ref{table:correlation} shows performance correlation and search cost of SGD, the one-shot
model, and our GHN. Note that GHN clearly outperforms the one-shot model, showing the effectiveness
of dynamically predicting parameters based on graph topology. While it takes 1000 SGD steps to
surpasses GHN in the ``Random-100'' setting, GHN is still the strongest in the ``Top-50'' setting,
which is more important for architecture search. Moreover, compared to GHN, running 1000 SGD steps
for every random architecture is over 1000 times more computationally expensive. In contrast, GHN
only requires a pre-training stage of 6 hours, and afterwards, the trained GHN can be used to
efficiently evaluate a massive number of random architectures of different sizes.
% !TEX root = top.tex
\subsection{Ablation Studies (CIFAR-10)}
\label{section:ablations}

\iflatexml
\begin{figure}
  \includegraphics[width=4\linewidth]{figures/nodes.png}
  \caption{Vary number of nodes; $T=5$ , forward-backward}
  \label{fig:sfig2}
\end{figure}
\begin{figure}
  \includegraphics[width=4\linewidth]{figures/tsteps.png}
\caption{ GHN when varying the number of nodes and propagation scheme}
% \label{fig:ghn_hyps}
  \label{fig:sfig1}
\end{figure}
\else
\begin{figure}[t]
\vspace{-1.0cm}
 \begin{center}
\begin{subfigure}{.48\textwidth}
  \includegraphics[width=0.8\linewidth]{figures/nodes.pdf}
  \caption{Vary number of nodes; $T=5$ , forward-backward}
  \label{fig:sfig2}
\end{subfigure}
\begin{subfigure}{.48\textwidth}
  \centering
  \includegraphics[width=0.8\linewidth]{figures/tsteps.pdf}
  \caption{Vary propagation schemes, $N=7$ }
  \label{fig:sfig1}
\end{subfigure}%
\caption{ GHN when varying the number of nodes and propagation scheme}
\label{fig:ghn_hyps}
\end{center}
\vspace{-0.5cm}
\end{figure}
\fi

\vspace{-0.25cm}
\paragraph{Number of graph nodes:}
The GHN is compatible with varying number of nodes - graphs used in training need not be the same
size as the graphs used for evaluation. Figure~\ref{fig:sfig2} shows how GHN performance varies as a
function of the number of nodes employed during training - fewer nodes generally produces better
performance. While the GHN has difficulty learning on larger graphs, likely due to the vanishing
gradient problem, it can generalize well from just learning on smaller graphs. Note that all GHNs
are  tested with the full graph size ($N=17$ nodes).

\vspace{-0.25cm}
\paragraph{Number of propagation steps:}
We now compare  the forward-backward propagation scheme with the regular synchronous propagation
scheme. Note that $T=1$ synchronous step corresponds to one full forward-backward phase. As shown in
Figure~\ref{fig:sfig1},  the forward-backward scheme consistently outperforms the synchronous
scheme. More propagation steps also help improving the performance, with a diminishing return. While
the forward-backward scheme is less amenable to acceleration from parallelization due to its
sequential nature, it is possible to parallelize the evaluation phase across multiple GHNs when
testing the fitness of candidate architectures.

\begin{wraptable}[8]{r}{5.5cm}
\iflatexml
\else
\footnotesize
\fi
\vspace{-0.4cm}
\begin{center}
\begin{tabular}{ c c c c} 
SP & PE & \multicolumn{2}{c}{Correlation}    \\ 
 &&  Random-100 & Top-50   \\ 
\hline
\xmark & \xmark & 0.24 & 0.15\\
\xmark & \cmark  &  0.44 & 0.37\\
\cmark & \cmark  & 0.68 & 0.48 
\end{tabular}
\end{center}
\vspace{-0.1in}
\caption{Stacked GHN Correlation. SP denotes share parameters and PE denotes passing embeddings}
\label{table:stacked}
\end{wraptable}

\vspace{-0.25cm}
\paragraph{Stacked GHN for architectural motifs:}
We also evaluate different design choices of GHNs on representing architectural motifs. We compare
1) individual GHNs, each predicting one block independently, 
2) a stacked GHN where individual GHN's
   pass on their graph embedding without sharing parameters, 
3) a stacked GHN with shared parameters (our proposed approach). 
As shown in Table~\ref{table:stacked},  passing messages between GHN's is crucial, and sharing parameters produces better performance.

