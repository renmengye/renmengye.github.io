% !TEX root = marvin.tex
\section{Experiments}
\subsection{Implementation Details and Baselines}
Each graph node has 16 communication channels for each agent. Similarly, the dimension of the
encoding vectors is 16. For combining the dot-product attention with the distance matrix, a 3-layer
[16-16-16] MLP with ReLU activation is used. When training, we set the learning rate of our model to
be 1e-3 using the Adam optimizer, with a decay rate of 0.1 every 2000 epochs. We train our model for
5000 epochs. We use a batch size of 50 graphs, each of which has up to 25 nodes. We train our
network with two agents only and evaluate with settings varying from one to nine agents.

% !TEX root = marvin.tex
\begin{figure*}[t]
\begin{center}
\iflatexml\includegraphics[width=6\textwidth]{figs/scalability.png}
\else
\fi
\centerline{\includegraphics[width=0.95\textwidth,trim={1cm 1cm 1cm 0}]{figs/scalability.pdf}}
\caption{Number of agents performing traversal and corresponding cost of the traversal
(trained using RL)}
\label{fig:scalability}
\end{center}
\vspace{-0.25in}
\end{figure*}
% !TEX root = top.tex
\section{Experiments}
In this section, we use our proposed GHN to search for the best CNN architecture for image
classification. First, we evaluate the GHN on the standard CIFAR \citep{krizhevsky2009cifar} and
ImageNet \citep{russakovsky2015imagenet} architecture search benchmarks. Next, we apply GHN on an
``anytime prediction'' task where we optimize the speed-accuracy tradeoff that is key for many
real-time applications. Finally, we benchmark the GHN's  predicted-performance correlation and
explore various factors in an ablation study.
% !TEX root = top.tex
\subsection{NAS benchmarks}
\begin{table}[t]
\vspace{-0.7cm}
\caption{Comparison with image classifiers found by state-of-the-art NAS methods which employ a random search on CIFAR-10. Results shown are mean $\pm$ standard deviation.}
\label{table:Results1}
\footnotesize
\begin{center}
\begin{tabular}{ c c c c } 
Method & Search Cost (GPU days) & Param $\times 10^6$ & Accuracy   \\ 
\hline
SMASHv1 \citep{brock2017smash} &? & 4.6 & 94.5 \\
SMASHv2 \citep{brock2017smash} & 3 & 16.0 & 96.0\\
One-Shot Top (F=32) \citep{bender2018understanding} & 4  & 2.7 $\pm$ 0.3 & 95.5 $\pm$ 0.1\\
One-Shot Top (F=64) \citep{bender2018understanding} & 4 & 10.4 $\pm$ 1.0 & 95.9 $\pm$ 0.2\\
\hline
\hline
Random (F=32) & - & 4.6 $\pm$ 0.6 & 94.6 $\pm$ 0.3\\ 
GHN Top (F=32) &  0.42  & 5.1 $\pm$ 0.6 & 95.7 $\pm$ 0.1\\ 
\end{tabular}
\end{center}
%\footnotesize{$^*$ \cite{bender2018understanding} 16 GPU hours training + 80 GPU hours evaluation. }
\end{table}
\begin{table}[t]
\vspace{-0.5cm}
\caption{Comparison with image classifiers found by state-of-the-art NAS methods which employ advanced search methods on CIFAR-10. Results shown are mean $\pm$ standard deviation.}
\label{table:Results2}
\vspace{-0.2cm}
\footnotesize
\begin{center}
\begin{tabular}{ c c c c } 
Method & Search Cost (GPU days) & Param $\times 10^6$ & Accuracy   \\ 
\hline
NASNet-A  \citep{zoph2017learning} & 1800 & 3.3 & 97.35 \\
%AmoebaNet-A  + CutOut \cite{real2018regularized} & 1 GPU, 0.45 days & 4.6 & 97.11 \\
%AmoebaNet-B + CutOut \cite{real2018regularized} & 1 GPU, 0.45 days & 4.6 & 97.11 \\
ENAS Cell search  \citep{pham2018efficient} & 0.45  & 4.6 & 97.11 \\
DARTS (first order)  \citep{liu2018darts} &  1.5  & 2.9  & 97.06  \\
DARTS (second order) \citep{liu2018darts} & 4  & 3.4 & 97.17 $\pm$ 0.06\\
\hline
\hline
GHN Top-Best, 1K (F=32) & 0.84  & 5.7 & 97.16 $\pm$ 0.07 \\
%GHN Top-Best, 20K (F=32) & 4.17 & 4.7 & 97.24 $\pm$ 0.05 \\
\end{tabular}
\end{center}
\end{table}

% !TEX root = top.tex
\begin{table}[t]
\vspace{-0.5cm}
\caption{Comparison with image classifiers found by state-of-the-art NAS methods which employ advanced search methods on ImageNet-Mobile.}
\vspace{-0.2cm}
\label{table:Results3}
\footnotesize
\begin{center}
\begin{tabular}{ c c c c } 
Method & Search Cost (GPU days) & Param $\times 10^6$ & Accuracy   \\ 
\hline
%One-Shot Top (F=24) \citep{bender2018understanding} & 3.3+? $^*$ & 6.8 $\pm$ 0.9 & 70.7 $\pm$ 0.6\\
%One-Shot Top (F=32) \citep{bender2018understanding} & 3.3+? $^*$ & 11.9 $\pm$ 1.5 & 72.6 $\pm$ 0.4\\
NASNet-A  \citep{zoph2017learning} & 1800 & 5.3 & 74.0 \\
%NASNet-B  \citep{zoph2017learning} & 1800 & 5.3 & 72.8 \\
NASNet-C  \citep{zoph2017learning} & 1800 & 4.9 & 72.5 \\
AmoebaNet-A \citep{real2018regularized} & 3150 & 5.1 & 74.5  \\
%AmoebaNet-B \citep{real2018regularized} & 3150 & 5.3 & 74.0  \\
AmoebaNet-C \citep{real2018regularized} & 3150 & 6.4 & 75.7  \\
PNAS \citep{liu2017progressive} & 225 & 	5.1 &  74.2\\
DARTS (second order) \citep{liu2018darts} & 4  & 4.9 & 73.1\\
\hline
\hline
GHN Top-Best, 1K & 0.84  & 6.1  & 73.0  \\
%GHN Top-Best, 20K & 4.17  & 5.0  & 73.2  \\
\end{tabular}
\end{center}
\vspace{-0.5cm}
\end{table}

\subsubsection{CIFAR-10}
\label{section:cifar10}
We conduct our initial set of experiments on CIFAR-10 \citep{krizhevsky2009cifar}, which contains 10
object classes and 50,000 training images and 10,000 test images of size 32$\times$32$\times$3. We
use 5,000 images split from the training set as our validation set.

\vspace{-0.25cm}
\paragraph{Search space:} 
Following existing NAS methods, we choose to search for optimal blocks rather than the entire
network. Each block contains 17 nodes, with 8 possible operations. The final architecture is formed
by stacking 18 blocks. The spatial size is halved and the number of channels is doubled after blocks
6 and 12. These settings are all chosen following recent NAS methods
\citep{zoph2016neural,pham2018efficient,liu2018darts}, with details in the Appendix.

\vspace{-0.25cm}
\paragraph{Training:}
For the GNN module, we use a standard GRU cell \citep{cho14gru} with hidden size 32 and 2
layer MLP with hidden size 32 as the recurrent cell function $U$ and message function $M$
respectively. The shared hypernetwork $H \left(\cdot; \vvphi\right)$ is a 2-layer MLP with hidden
size 64. From the results of ablations studies in Section~\ref{section:ablations}, the GHN is
trained with blocks with $N=7$ nodes and $T=5$ propagations under the forward-backward scheme, using
the ADAM optimizer \citep{kingma2015adam}. Training details of the final selected architectures are
chosen to follow existing works and can be found in the Appendix.
\vspace{-0.25cm}
\paragraph{Evaluation:}
First, we compare to similar methods that use random search with a  hypernetwork or a one-shot model
as a surrogate search signal. We randomly sample 10 architectures and train until convergence for
our random baseline. Next, we randomly sample 1000 architectures, and select the top 10 performing
architectures with GHN generated weights, which we refer to as GHN Top. Our reported search cost
includes both the GHN training and evaluation phase. Shown in Table~\ref{table:Results1}, the GHN
achieves competitive results with nearly an order of magnitude reduction in search cost.

In Table~\ref{table:Results2}, we compare with methods which use more advanced search methods, such
as reinforcement learning and evolution. Once again, we sample 1000 architectures and use the GHN to
select the top 10. To make a fair comparison for random search, we train the top 10 for a short
period before selecting the best to train until convergence. The accuracy reported for GHN Top-Best
is the average of 5 runs  of the same final architecture. Note that all methods in
Table~\ref{table:Results2} use CutOut~\citep{devriescutout17}. GHN achieves very competitive results
with a simple random search algorithm, while only using a fraction of the total search cost. Using
advanced search methods with GHNs may bring further gains.

\subsubsection{ImageNet-Mobile}
We also run our GHN algorithm on the ImageNet dataset \citep{russakovsky2015imagenet}, which
contains 1.28 million training images. We report the top-1  accuracy on the 50,000 validation
images. Following existing literature, we conduct the ImageNet experiments in the mobile setting,
where the model is constrained to be under 600M FLOPS. We directly transfer the best architecture
block found in the CIFAR-10 experiments, using an initial convolution layer of stride 2 before
stacking 14 blocks with scale reduction at blocks 1, 2, 6 and 10. The total number of flops is
constrained by choosing the initial number of channels. We follow existing NAS methods on the
training procedure of the final architecture; details can be found in the Appendix. As shown in
Table \ref{table:Results3} the transferred block is competitive with other NAS methods which require
a far greater search cost.
% !TEX root = top.tex
\subsection{Anytime Prediction}
In the real-time setting, the computational budget available can vary for each test case and cannot
be known ahead of time. This is formalized in anytime prediction, \citep{grubb2012speedboost}  the
setting in which for each test example $\rvx$, there is non-deterministic computational budget $B$
drawn from the joint distribution $P(\rvx, B)$. The goal is then to minimize the expected loss $L(f)
= \E\left[ L\left( f(\rvx), B \right)\right]_{P(\rvx, B)}$, where $f(\cdot)$ is the model and
$L(\cdot)$ is the loss for an $f(\cdot)$ that must produce a prediction within the budget $B$.

We conduct experiments on CIFAR-10. Our anytime search space consists of networks with 3 cells
containing 24, 16, and 8 nodes. Each node is given the additional properties: 1) the spatial size it
operates at and 2) if an early-exit classifier is attached to it. A node enforces its spatial size
by pooling or upsampling any input feature maps inputs that are of different scale. Note that while
a naive one-shot model would triple its size to include three different parameter sets at three
different scales, the GHN is negligibly affected by such a change. The GHN uses the area under the
predicted accuracy-FLOPS curve as its selection criteria. The search space, contains various
convolution and pooling operators. Training methodology of the final architectures are chosen to
match \cite{huang2017multi} and can be found in the Appendix.

Figure \ref{fig:test1} shows a comparison with the various methods presented by
\cite{huang2017multi}. Our experiments show that the best searched architectures can outperform the
current state-of-the-art human designed networks. We see the GHN is amenable to the changes proposed
above, and can find efficient architectures with a random search when used with a strong search
space.

\iflatexml
\begin{figure}
  \includegraphics[width=4\linewidth]{figures/anytime_compare.png}
  \captionof{figure}{Comparison with state-of-the-art\\ human-designed networks on CIFAR-10.}
  \label{fig:test1}
\end{figure}
\begin{figure}
  \includegraphics[width=4\linewidth]{figures/anytime_randoms.png}
  \captionof{figure}{Comparison between random 10 and\\ top 10 networks on CIFAR-10.}
  \label{fig:test2}
\end{figure}
\else
\begin{figure}[t]
\vspace{-0.5cm}
\centering
\begin{minipage}{.48\textwidth}
  \centering
  \includegraphics[width=0.8\linewidth]{figures/anytime_compare.pdf}
  \captionof{figure}{Comparison with state-of-the-art\\ human-designed networks on CIFAR-10.}
\label{table:Results4}
  \label{fig:test1}
\end{minipage}%
\begin{minipage}{.48\textwidth}
  \centering
  \includegraphics[width=0.8\linewidth]{figures/anytime_randoms.pdf}
  \captionof{figure}{Comparison between random 10 and\\ top 10 networks on CIFAR-10.}
  \label{fig:test2}
\end{minipage}
\end{figure}
\fi
% !TEX root = top.tex
\subsection{Predicted performance correlation (CIFAR-10)}
\begin{table}[t]
\caption{Benchmarking the correlation between the predicted and true performance of the GHN against SGD and a one-shot model baselines. Results are on CIFAR-10.}
\vspace{-0.2cm}
\label{table:correlation}
\small
\begin{center}
\begin{tabular}{ c c c c c} 
Method & \multicolumn{2}{c}{Computation cost}   & \multicolumn{2}{c}{Correlation}    \\ 
%\hline
 & Initial (GPU hours) & Per arch. (GPU seconds)  & Random-100 & Top-50   \\ 
\hline
SGD 10 Steps & - & 0.9 & 0.26 & -0.05\\
SGD 100 Steps & - & 9 & 0.59 & 0.06\\
SGD 200 Steps & - & 18 & 0.62 & 0.20 \\
SGD 1000 Steps & - & 90 & 0.77 & 0.26 \\
One-Shot & 9.8 & 0.06 & 0.58 & 0.31\\
\hline
\hline
GHN & 6.1 & 0.08 & 0.68 & 0.48
\end{tabular}
\end{center}
\end{table}

In this section, we evaluate  whether the parameters generated from GHN can be indicative of the
final performance. Our metric is the correlation between the accuracy of a model with trained
weights vs. GHN generated weights. We use a fixed set of 100 random architectures that have not been
seen by the GHN during training, and we train them for 50 epochs to obtain our ``ground-truth''
accuracy, and finally compare with the accuracy obtained from GHN generated weights. We report the
Pearson's R score on all 100 random architectures and the top 50 performing architectures (i.e.\
above average architectures). Since we are interested in searching for the best architecture,
obtaining a higher correlation on top performing architectures is more meaningful.

To evaluate the effectiveness of GHN, we further consider two baselines: 1) training a network with
SGD from scratch for a varying number of steps, and 2) our own implementation of the one-shot model
proposed by \citet{pham2018efficient}, where nodes store a set of shared parameters for each
possible operation. Unlike GHN, which is compatible with varying number of nodes, the one-shot model
must be trained with $N=17$ nodes to match the evaluation. The GHN is trained with $N=7$, $T=5$
using forward-backward propagation. These GHN parameters are selected based on the results found in
Section~\ref{section:ablations}.

Table \ref{table:correlation} shows performance correlation and search cost of SGD, the one-shot
model, and our GHN. Note that GHN clearly outperforms the one-shot model, showing the effectiveness
of dynamically predicting parameters based on graph topology. While it takes 1000 SGD steps to
surpasses GHN in the ``Random-100'' setting, GHN is still the strongest in the ``Top-50'' setting,
which is more important for architecture search. Moreover, compared to GHN, running 1000 SGD steps
for every random architecture is over 1000 times more computationally expensive. In contrast, GHN
only requires a pre-training stage of 6 hours, and afterwards, the trained GHN can be used to
efficiently evaluate a massive number of random architectures of different sizes.
% !TEX root = top.tex
\subsection{Ablation Studies (CIFAR-10)}
\label{section:ablations}

\iflatexml
\begin{figure}
  \includegraphics[width=4\linewidth]{figures/nodes.png}
  \caption{Vary number of nodes; $T=5$ , forward-backward}
  \label{fig:sfig2}
\end{figure}
\begin{figure}
  \includegraphics[width=4\linewidth]{figures/tsteps.png}
\caption{ GHN when varying the number of nodes and propagation scheme}
% \label{fig:ghn_hyps}
  \label{fig:sfig1}
\end{figure}
\else
\begin{figure}[t]
\vspace{-1.0cm}
 \begin{center}
\begin{subfigure}{.48\textwidth}
  \includegraphics[width=0.8\linewidth]{figures/nodes.pdf}
  \caption{Vary number of nodes; $T=5$ , forward-backward}
  \label{fig:sfig2}
\end{subfigure}
\begin{subfigure}{.48\textwidth}
  \centering
  \includegraphics[width=0.8\linewidth]{figures/tsteps.pdf}
  \caption{Vary propagation schemes, $N=7$ }
  \label{fig:sfig1}
\end{subfigure}%
\caption{ GHN when varying the number of nodes and propagation scheme}
\label{fig:ghn_hyps}
\end{center}
\vspace{-0.5cm}
\end{figure}
\fi

\vspace{-0.25cm}
\paragraph{Number of graph nodes:}
The GHN is compatible with varying number of nodes - graphs used in training need not be the same
size as the graphs used for evaluation. Figure~\ref{fig:sfig2} shows how GHN performance varies as a
function of the number of nodes employed during training - fewer nodes generally produces better
performance. While the GHN has difficulty learning on larger graphs, likely due to the vanishing
gradient problem, it can generalize well from just learning on smaller graphs. Note that all GHNs
are  tested with the full graph size ($N=17$ nodes).

\vspace{-0.25cm}
\paragraph{Number of propagation steps:}
We now compare  the forward-backward propagation scheme with the regular synchronous propagation
scheme. Note that $T=1$ synchronous step corresponds to one full forward-backward phase. As shown in
Figure~\ref{fig:sfig1},  the forward-backward scheme consistently outperforms the synchronous
scheme. More propagation steps also help improving the performance, with a diminishing return. While
the forward-backward scheme is less amenable to acceleration from parallelization due to its
sequential nature, it is possible to parallelize the evaluation phase across multiple GHNs when
testing the fitness of candidate architectures.

\begin{wraptable}[8]{r}{5.5cm}
\iflatexml
\else
\footnotesize
\fi
\vspace{-0.4cm}
\begin{center}
\begin{tabular}{ c c c c} 
SP & PE & \multicolumn{2}{c}{Correlation}    \\ 
 &&  Random-100 & Top-50   \\ 
\hline
\xmark & \xmark & 0.24 & 0.15\\
\xmark & \cmark  &  0.44 & 0.37\\
\cmark & \cmark  & 0.68 & 0.48 
\end{tabular}
\end{center}
\vspace{-0.1in}
\caption{Stacked GHN Correlation. SP denotes share parameters and PE denotes passing embeddings}
\label{table:stacked}
\end{wraptable}

\vspace{-0.25cm}
\paragraph{Stacked GHN for architectural motifs:}
We also evaluate different design choices of GHNs on representing architectural motifs. We compare
1) individual GHNs, each predicting one block independently, 
2) a stacked GHN where individual GHN's
   pass on their graph embedding without sharing parameters, 
3) a stacked GHN with shared parameters (our proposed approach). 
As shown in Table~\ref{table:stacked},  passing messages between GHN's is crucial, and sharing parameters produces better performance.

%\subsection{Baselines}
We compare our approach with the following baselines.
\paragraph{Random:} This baseline consists of visiting each node that has not been completely mapped
yet in a random order.

\vspace{-0.1in}
\paragraph{Greedy:} Each agent visits the closest node that still needs to be mapped. It assumes
that the agents are able to communicate which nodes have been fully mapped.

\vspace{-0.1in}
\paragraph{LKH3:} represents the best  performance of the iterative solver given limited information.
We first allow the solver ~\citep{lkh3} to calculate the optimal path for covering each
 node exactly once. Then, the solver calculates a new optimal path over all the remaining
nodes that must be mapped. This is repeated until all nodes have been fully mapped. In essence, the
solver performs VRP traversals until all nodes have been visited the desired number of times.

\vspace{-0.1in}
\paragraph{GVIN:} The Generalized Value Iteration Network~\citep{gvin} uses a GNN to propagate
values on a graph. While the original implementation does not integrate communication between
multiple agents, we enhanced  GVIN with our attention communication module for fair comparison. This
model was trained with imitation learning to achieve  best performance.

\vspace{-0.1in}
\paragraph{GAT:} Graph Attention Networks~\citep{gat} are similar to our method, in that they
exchange information according to attention between two nodes in order to convey complex
information. However, standard GAT architectures do not encode the distance matrix information and
instead assume all edges have an equal weight, limiting their capabilities.
% in a weighted graph domain.
While GATs are not necessarily designed to solve the TSP or VRP problems,  they  remain
one of the state-of-the-art solutions for graph and network encoding.

\vspace{-0.1in}
\paragraph{AM:} The Attention Model~\citep{am}  has been used as a deep learning VRP solver. Since
it has no natural way to encode the information in the adjacency matrix, we add a Graph Convolution Network (GCN) % \raquel{you havent defined GCN}
encoding module at the beginning to perform this encoding and use imitation learning for training. The
modification that the authors suggest for the AM algorithm to allow it to perform a VRP traversal
does not allow for dynamic route adaptation, as is required in our environment. We therefore only
evaluate this method in the single agent scenario.
%\raquel{this is a very odd statement. Where is this modification suggested?}
%\quin{This modification for how to perform VRP is suggested in the original AM paper. They
%compute the VRP solution by repeatedly sending out a single agent and having it return back to
%the starting location instead of trying to coordinate the agents in tandem}

\vspace{-0.1in}
\paragraph{EAN:} Encode-Attend-Navigate~\citep{ean} is a deep learning TSP solver designed very
similarly to AM, but with a slightly different architecture. We enhance it with an additional GCN
module at its beginning to encode the adjacency matrix in the same fashion as with AM. This model
was trained with imitation learning as well.

\vspace{-0.1in}
\paragraph{Oracle:} This is the upper bound performance that an agent could have possibly achieved
if given global information about all the hidden states. This solution is found by providing the
LKH3 solver with details about all the hidden variables, and solving for the optimal plan. We transform
the adjacency matrix by increasing the edge weights of the nodes effected by congestion, and by
duplicating the nodes that will require multiple passes to be fully mapped.

% !TEX root = marvin.tex
\begin{figure*}[t]
\centering
\iflatexml
\includegraphics[width=6\textwidth]{figs/figall.png}
\else
\begin{small}
\vspace{-0.1in}
\begin{tabular}{llll}
(a) & (b) & (c) & (d)\\
\includegraphics[height=4.1cm,trim={0.2cm 0 0.4cm 0},clip]{figs/diff_cost.pdf} &
\includegraphics[height=4.1cm,trim={0.2cm 0 0.8cm 0},clip]{figs/runtime.pdf} &
\includegraphics[height=4.1cm,trim={0.2cm 0 0.8cm 0},clip]{figs/iterations} &
\includegraphics[height=4.1cm,trim={0.0cm 0 0cm 0},clip]{figs/comms}\\
\end{tabular}
\end{small}
\fi
\vspace{-0.1in}
\caption{\textbf{(a) IL vs. RL:} Comparison of imitation learning and reinforcement learning on different number of agents;
\textbf{(b) Runtime:} Comparison MARVIN's runtime to that of the LKH solver;
\textbf{(c) No. iterations in the VIN module:} Evaluation of how the number of value iterations has on performance, and how the number of
    iterations generally scales for other value iteration models (GVIN);
\textbf{(d) Communication module design:} Comparison of our communication protocol to other communication protocol alternatives.
}
\label{fig:all}
\end{figure*}
\subsection{Results}
\vspace{-0.1in}
\paragraph{Comparisons to Baselines:} As shown in Table~\ref{tab:1}, our method has the best
performance across different numbers of agents and graph sizes. Notably, under 25 nodes and two
agents, which is the training setting, our method with RL achieves a total cost that is within 3\%
from that of the oracle. We found our model trained with both reinforcement learning and imitation
learning outperforms all competitor models. Overall, the model shows impressive generalization to
more agents and larger graph size, since we limit the training of the model with two agents and 25
nodes. We also note that deep learning-based solvers that perform well in the more traditional TSP
domain (AM, EAN) are unable to generalize well to our realistic benchmark. Specifically, the low
performance of these deep learning solvers can be attributed to their inability to cope with mapping
failures and nodes requiring multiple passes, which in turn is the result of their architecture
not being structured for this  problem formulation.
% architecture does not account for these realistic challenges, and the potential for revisits of the
% same node.
% \raquel{this last sentence is redundant with the previous one, or?. merge them}

\vspace{-0.1in}
\paragraph{Load distribution:} The cost of performing each traversal is very evenly spread out
amongs all of the agents. The maximum Gini coefficient for two agents observed on our evaluation set
was 0.169, with the mean coefficient being around 0.075.

\vspace{-0.1in}
\paragraph{Scale to number of agents and graph size:}  One
of the primary focuses of our work is to develop a model that scales well with the number of agents
and the size of the graph. Therefore, we evaluated our model's performance on increasingly large
graphs and observed how the performance varied with the number of agents.  As shown in Fig.~\ref{fig:scalability}, 
 the total cost increases marginally when we increase the total number of agents,
indicating good scalability in this respect, and that our method performs much better than the
current state of the art, as is represented by LKH.

% \raquel{we are missing the link to Fig.~\ref{fig:scalability} and talk aobut what we see on it}

We notice that models trained with RL appear to be able to generalize better when dealing with a
larger number of agents, shown in Fig.~\ref{fig:all}A. This could be explained by the fact that
supervised learning tries to exactly mimic the optimal strategy, which may not carry over when
dealing with more agents, and therefore generalizes less effectively.

We also evaluate how a model trained on only toy graphs with 25 nodes compares with a graph trained
only on toy graphs with 100 nodes.
%\raquel{why are you talking about toy graphs?}
%\quin{I did not generate enough realistic graphs that were 100 nodes in size to run on realistic
%graphs so this experiment was performed on the toy graphs to give a general sense of how our model
%is able to be trained on larger environments if needed}
As shown in Fig.~\ref{fig:retrained} ,the 100 node model scales much
better on larger graphs, but that  25 node version of the model is able to perform much closer to
true optimal when acting on smaller graphs.

% !TEX root = marvin.tex

\begin{figure}[t]
\begin{center}
\iflatexml
\includegraphics[width=6\columnwidth]{figs/sizes.png}
\else
\includegraphics[width=0.94\columnwidth]{figs/sizes.pdf}
\fi
\vspace{-0.2in}
\caption{The average traversal time (hrs) relative to the number of nodes in the graph
for policies trained exclusively on 25 node graphs and 100 node graphs
}
\vspace{-0.25in}
\label{fig:retrained}
\end{center}
\end{figure}

\vspace{-0.1in}
\paragraph{Runtime:} An  advantage that deep learning solutions have over conventional
solvers is their runtime. We compare the average runtime of our model versus that of the LKH3
solver in Fig.~\ref{fig:all}B. Note that our model is significantly faster than LKH3 on large
scale graphs with over 100 nodes.

\vspace{-0.1in}
\paragraph{Robustness to distribution shift:} We also evaluate the generalizability of our model to
different ``multiple pass'' distributions. In order to simulate realistic mapping failures, each
node must be revisited an unknown number of times before we say that it has been completely mapped.
The ``multiple pass'' distributions is the distribution from which these numbers are selected.
% \raquel{explain here what you mean by multiple pass}
Shown in Table~\ref{tab:multipass}, our model has a
consistent performance when we change the distribution at test time, despite being trained
exclusively on the uniform 1-3 distribution.

% !TEX root = marvin.tex
\begin{table}[t]
\begin{center}
\begin{small}
\resizebox{0.9\columnwidth}{!}{%
\begin{tabular}{c|c|ccc}
\toprule
Multi-pass distribution & Oracle & LKH3  & Ours (RL)     & Ours (IL)     \\
\midrule
Uniform 1 - 3           & 1.28   & 1.80  & \textbf{1.32} & 1.42          \\
Uniform 1 - 5           & 1.92   & 3.05  & \textbf{2.11} & 2.20          \\
Uniform 1 - 10          & 3.50   & 5.72  & \textbf{3.82} & 4.24          \\
TruncGaussian 1 - 3     & 1.51   & 2.52  & \textbf{1.75} & 1.79          \\
Either 2 or 4           & 1.72   & 2.49  & \textbf{1.84} & 2.01          \\
Only 3                  & 1.52   & 1.92  & \textbf{1.68} & \textbf{1.68} \\
Exp (mean = 2)          & 1.67   & 3.26  & \textbf{1.73} & 1.84          \\
\bottomrule
\end{tabular}
}
\end{small}
\vspace{-0.1in}
\caption{Model performance on different multi-pass distributions}
\vspace{-0.2in}
\label{tab:multipass}
\end{center}
\end{table}

\vspace{-0.1in}
\paragraph{Number of value iterations:}
In this experiment, we extend the number of iterations in our value iteration module to see if the
module can benefit from longer reasoning. We originally trained our model with 5 iteration, and find
that when we scale the number of iterations up to 10 during evaluation, the performance also further
increases, as is seen in Fig.~\ref{fig:all}C.

% !TEX root = marvin.tex
\begin{table}[t]
\begin{center}
\begin{small}
      \resizebox{0.9\columnwidth}{!}{%
\begin{tabular}{c|cccc}
\toprule
Method                              & Network Size    & Cost            & Gap             & Runtime         \\
\midrule
Concorde (Oracle)                   & -               & 4.22            & 0.00\%          & 40.1            \\
LKH3                                & -               & 4.22            & 0.00\%          & 159             \\
OR Tools                            & -               & 4.27            & 1.11\%          & 15.0            \\
\midrule
Random Insertion                    & -               & 4.44            & 5.12\%          & \textbf{2.31}   \\
Nearest Insertion                   & -               & 4.84            & 14.7\%          & 15.4            \\
Farthest Insertion                  & -               & 4.34            & 2.36\%          & 4.66            \\
Nearest Neighbour                   & -               & 5.02            & 19.0\%          & 14.2            \\
\midrule
AM (SS)                             & 28 MB           & 4.24            & 0.51\%          & 16.3            \\
AM (SS + SP)                        & 28 MB           & 4.23            & 0.13\%          & 552             \\
MARVIN                              & 0.04 MB           & 4.54            & 7.56\%          & 61.7            \\
MARVIN (SS)                         & 0.04 MB           & 4.32            & 2.38\%          & 18.9            \\
MARVIN (SS + SP)                    & 0.04 MB           & \textbf{4.23}   & \textbf{0.10\%} & 1714            \\
\bottomrule
\end{tabular}
}
\end{small}
\caption{Single agent TSP on synthetic graphs of size 25.
We abbreviate methods that make use of \emph{self-starting} with \textbf{SS}
and \emph{sampling} with \textbf{SP}.}
\label{tab:toy}
\end{center}
\vspace{-0.2in}
\end{table}

\vspace{-0.1in}
\paragraph{Toy TSP:} To validate that our model can also solve toy TSP problems and thoroughly be
compared with previous methods under their settings, we run a single-agent TSP benchmark with graphs
of size 25 and uniformly generated vertices in a 1 by 1 square, following \citep{am}. Random,
Nearest, and Farthest insertion, as well as nearest neighbour and AM are all taken from~\cite{am}.
Usually, in our problem setting the agent has no control over where it begins. However, since a
complete TSP tour is independent of its starting position, we also test the effect of letting our
model choose its starting position, which we denote as \emph{self-starting}. We also evaluate how
other conventional tour augmentation techniques affect our model's performance.
\emph{Sampling} takes random samples from the action space of each of the agents and chooses the
one with the lowest overall cost. When augmenting the method with trajectory sampling, we sample
from 1280 model-guided stochastic runs.
%\raquel{why that value? odd}
%\quin{okay to be honest, which I was running this experiment I meant to type in 1028 but I think I accidentally
%typed it in wrong so this ended up being the value I used since I didn't really think it mattered}
As shown in Table~\ref{tab:toy}, we find that even though our model performs worse than the state-of-the-art attention model when we simply take a single greedy trajectory, it is able to outperform it
when both models are augmented with trajectory sampling, getting within 0.10\% of the optimal. It is
also worth noting that our model is 800 times smaller than the best performer in terms of the number
of parameters.

\vspace{-0.1in}
\paragraph{Visualization of large scale mapping:} We also visualize our model navigating a swam of
agents on a large portion of Chicago. Here we perform a large scale autonomous mapping simulation
with a fleet of 20 vehicles on a graph of size 2426 nodes.
% Our visualization will be released very
% soon \mengye{Add blog link}. \raquel{this is very odd, either you have the blog link or you have some figure in the paper}
We generally observe that when compared to other deep learning
solutions, our model results in a much more thorough sweep, where it more rarely has to revisit
previously seen regions. We further observe that models trained with imitation learning  adopt more ``exploratory'' strategies, where
agents split from the main swarm to visit new regions. For more details on the implications of this strategy
please refer to the Appendix
% \raquel{explain why this is not good}

% !TEX root = marvin.tex
\begin{table}[t]
    \begin{center}
    \begin{small}
    \resizebox{0.85\columnwidth}{!}{%
    \begin{tabular}{c|ccc|c}
    \toprule
    Variant                &  Attn. & Dense adj. & LSTM         & Action Acc.    \\
    \midrule
    GVIN                   &             &            &             & 65.4\%          \\
    GAT                    & \checkmark  &            &             & 23.5\%          \\
    No LSTM                & \checkmark  & \checkmark &             & 71.7\%          \\
    Full                   & \checkmark  & \checkmark & \checkmark  & \textbf{75.8\%} \\
    \bottomrule
    \end{tabular}
    }
    \end{small}
    \caption{Action prediction accuracy on different value iteration module designs.
    All models are trained using imitation learning.}
    \vspace{-0.25in}
    \label{tab:accuracy}
    \end{center}
\end{table}
%\subsection{Ablation studies}
\vspace{-0.1in}
\paragraph{Value iteration module:}
We investigated various design choices of the value iteration module, shown in
Table~\ref{tab:accuracy}, where we train different modules using IL and test them in
terms of action prediction accuracy. As shown, GVIN, lacking the dense adjacency matrix and
attention mechanism, is significantly worse than our model, and adding an LSTM further improves the
performance by allowing an extended number of iterations of value reasoning.

\vspace{-0.1in}
\paragraph{Communication module:}
We investigated different potential designs of the communication module,
including CommNet style, MaxPooling, and the number of channels. As shown in Fig.~\ref{fig:all}D, we found that our attention based
modules performs significantly better. Furthermore, the performance is improved with more channels.



