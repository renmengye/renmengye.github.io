% !TEX root = marvin.tex
\vspace{-0.3in}
\begin{abstract}
In this paper we tackle the problem of routing multiple agents in a coordinated manner. This is a
complex problem that has a wide range of applications in fleet management to achieve a common goal,
such as mapping from a swarm of robots  and ride sharing. Traditional methods are typically not
designed for realistic environments which contain  sparsely connected graphs and unknown
traffic, and are often too slow in runtime to be practical. In contrast, we propose a graph neural
network based model that is able to perform multi-agent routing based on learned value iteration in
a sparsely connected graph with dynamically changing traffic conditions. Moreover, our learned
communication module enables the agents to coordinate online and adapt to changes more effectively.
We created a simulated environment to mimic realistic  mapping performed by autonomous vehicles with unknown minimum edge
coverage and traffic conditions;  Our approach significantly outperforms traditional solvers
both in terms of total cost and runtime. % in this challenges scenario.
We also show that our model trained with only two agents on
graphs with a maximum of 25 nodes can easily generalize to situations with more agents and/or nodes. %o five agents with a hundred
%nodes. \raquel{it was too specific, and thats not the point }
\end{abstract}