% !TEX root = ../main.tex
\begin{abstract}
\looseness=-1
Deep neural nets typically perform end-to-end backpropagation to learn the weights, a procedure that
creates synchronization constraints in the weight update step across layers and is not biologically
plausible. Recent advances in unsupervised contrastive representation learning invite the question
of whether a learning algorithm can also be made local, that is, the updates of lower layers do not
directly depend on the computation of upper layers. While Greedy InfoMax~\cite{e2e2e} separately
learns each block with a local objective, we found that it consistently hurts readout accuracy in
state-of-the-art unsupervised contrastive learning algorithms, possibly due to the greedy objective
as well as gradient isolation. In this work, we discover that by overlapping local blocks stacking
on top of each other, we effectively increase the decoder depth and allow upper blocks to implicitly
send feedbacks to lower blocks. This simple design closes the performance gap between local learning 
and end-to-end contrastive learning algorithms for the first time. Aside from standard ImageNet 
experiments, we also show results on complex downstream tasks such as object detection and instance 
segmentation directly using readout features.

%Code will be released upon acceptance.
\end{abstract}