% !TEX root = ../main.tex
\section*{Broader Impact}
\vspace{-0.1in}
Our work aims to make deep unsupervised representation learning more biologically plausible by
removing the reliance on end-to-end backpropagation, a step towards a better understanding of the
learning in our brain. This can potentially lead to solutions towards mental and psychological
illness. Our algorithm also lowers the GPU memory requirements and can be deployed with model
parallel configurations. This can potentially allow deep learning training to run on cheaper and
more energy efficient hardware, which would make a positive impact to combat climate change. We
acknowledge unknown risks can be brought by the development of AI technology; however, the
contribution of this paper has no greater risk than any other generic deep learning paper that
studies standard datasets such as ImageNet.


% Our work aims to bridge the gap between machine learning and human learning in terms of the
% adaptibility at test time of recognizing novel objects. As shown in our experiments, potential
% applications include more personalized assistive technology and general home robotics. Home robotics
% in many ways will fill in the gap instead of replacing humans. For example, as the society ages, the
% population needs additional caregiving at home, but this option is currently not available to
% everyone. Having a robot that can cook and do laundry at home can greatly alleviate the burden to
% many people. We also note that while our intention is good, just like all technologies, the
% algorithm we develop can also bring disadvantages to certain group, for example, people who cannot
% afford products sold by companies that have put in their resources to commercialize the
% technological advances made by us, and people or facilities that are targeted by autonomous
% surveillance and weapons. Whether the potential impact is positive or negative is unclear to us and
% we would like to leave it open to debate.