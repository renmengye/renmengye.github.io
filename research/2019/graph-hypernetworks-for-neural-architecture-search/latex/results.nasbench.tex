% !TEX root = top.tex
\subsection{NAS benchmarks}
\begin{table}[t]
\vspace{-0.7cm}
\caption{Comparison with image classifiers found by state-of-the-art NAS methods which employ a random search on CIFAR-10. Results shown are mean $\pm$ standard deviation.}
\label{table:Results1}
\footnotesize
\begin{center}
\begin{tabular}{ c c c c } 
Method & Search Cost (GPU days) & Param $\times 10^6$ & Accuracy   \\ 
\hline
SMASHv1 \citep{brock2017smash} &? & 4.6 & 94.5 \\
SMASHv2 \citep{brock2017smash} & 3 & 16.0 & 96.0\\
One-Shot Top (F=32) \citep{bender2018understanding} & 4  & 2.7 $\pm$ 0.3 & 95.5 $\pm$ 0.1\\
One-Shot Top (F=64) \citep{bender2018understanding} & 4 & 10.4 $\pm$ 1.0 & 95.9 $\pm$ 0.2\\
\hline
\hline
Random (F=32) & - & 4.6 $\pm$ 0.6 & 94.6 $\pm$ 0.3\\ 
GHN Top (F=32) &  0.42  & 5.1 $\pm$ 0.6 & 95.7 $\pm$ 0.1\\ 
\end{tabular}
\end{center}
%\footnotesize{$^*$ \cite{bender2018understanding} 16 GPU hours training + 80 GPU hours evaluation. }
\end{table}
\begin{table}[t]
\vspace{-0.5cm}
\caption{Comparison with image classifiers found by state-of-the-art NAS methods which employ advanced search methods on CIFAR-10. Results shown are mean $\pm$ standard deviation.}
\label{table:Results2}
\vspace{-0.2cm}
\footnotesize
\begin{center}
\begin{tabular}{ c c c c } 
Method & Search Cost (GPU days) & Param $\times 10^6$ & Accuracy   \\ 
\hline
NASNet-A  \citep{zoph2017learning} & 1800 & 3.3 & 97.35 \\
%AmoebaNet-A  + CutOut \cite{real2018regularized} & 1 GPU, 0.45 days & 4.6 & 97.11 \\
%AmoebaNet-B + CutOut \cite{real2018regularized} & 1 GPU, 0.45 days & 4.6 & 97.11 \\
ENAS Cell search  \citep{pham2018efficient} & 0.45  & 4.6 & 97.11 \\
DARTS (first order)  \citep{liu2018darts} &  1.5  & 2.9  & 97.06  \\
DARTS (second order) \citep{liu2018darts} & 4  & 3.4 & 97.17 $\pm$ 0.06\\
\hline
\hline
GHN Top-Best, 1K (F=32) & 0.84  & 5.7 & 97.16 $\pm$ 0.07 \\
%GHN Top-Best, 20K (F=32) & 4.17 & 4.7 & 97.24 $\pm$ 0.05 \\
\end{tabular}
\end{center}
\end{table}

% !TEX root = top.tex
\begin{table}[t]
\vspace{-0.5cm}
\caption{Comparison with image classifiers found by state-of-the-art NAS methods which employ advanced search methods on ImageNet-Mobile.}
\vspace{-0.2cm}
\label{table:Results3}
\footnotesize
\begin{center}
\begin{tabular}{ c c c c } 
Method & Search Cost (GPU days) & Param $\times 10^6$ & Accuracy   \\ 
\hline
%One-Shot Top (F=24) \citep{bender2018understanding} & 3.3+? $^*$ & 6.8 $\pm$ 0.9 & 70.7 $\pm$ 0.6\\
%One-Shot Top (F=32) \citep{bender2018understanding} & 3.3+? $^*$ & 11.9 $\pm$ 1.5 & 72.6 $\pm$ 0.4\\
NASNet-A  \citep{zoph2017learning} & 1800 & 5.3 & 74.0 \\
%NASNet-B  \citep{zoph2017learning} & 1800 & 5.3 & 72.8 \\
NASNet-C  \citep{zoph2017learning} & 1800 & 4.9 & 72.5 \\
AmoebaNet-A \citep{real2018regularized} & 3150 & 5.1 & 74.5  \\
%AmoebaNet-B \citep{real2018regularized} & 3150 & 5.3 & 74.0  \\
AmoebaNet-C \citep{real2018regularized} & 3150 & 6.4 & 75.7  \\
PNAS \citep{liu2017progressive} & 225 & 	5.1 &  74.2\\
DARTS (second order) \citep{liu2018darts} & 4  & 4.9 & 73.1\\
\hline
\hline
GHN Top-Best, 1K & 0.84  & 6.1  & 73.0  \\
%GHN Top-Best, 20K & 4.17  & 5.0  & 73.2  \\
\end{tabular}
\end{center}
\vspace{-0.5cm}
\end{table}

\subsubsection{CIFAR-10}
\label{section:cifar10}
We conduct our initial set of experiments on CIFAR-10 \citep{krizhevsky2009cifar}, which contains 10
object classes and 50,000 training images and 10,000 test images of size 32$\times$32$\times$3. We
use 5,000 images split from the training set as our validation set.

\vspace{-0.25cm}
\paragraph{Search space:} 
Following existing NAS methods, we choose to search for optimal blocks rather than the entire
network. Each block contains 17 nodes, with 8 possible operations. The final architecture is formed
by stacking 18 blocks. The spatial size is halved and the number of channels is doubled after blocks
6 and 12. These settings are all chosen following recent NAS methods
\citep{zoph2016neural,pham2018efficient,liu2018darts}, with details in the Appendix.

\vspace{-0.25cm}
\paragraph{Training:}
For the GNN module, we use a standard GRU cell \citep{cho14gru} with hidden size 32 and 2
layer MLP with hidden size 32 as the recurrent cell function $U$ and message function $M$
respectively. The shared hypernetwork $H \left(\cdot; \vvphi\right)$ is a 2-layer MLP with hidden
size 64. From the results of ablations studies in Section~\ref{section:ablations}, the GHN is
trained with blocks with $N=7$ nodes and $T=5$ propagations under the forward-backward scheme, using
the ADAM optimizer \citep{kingma2015adam}. Training details of the final selected architectures are
chosen to follow existing works and can be found in the Appendix.
\vspace{-0.25cm}
\paragraph{Evaluation:}
First, we compare to similar methods that use random search with a  hypernetwork or a one-shot model
as a surrogate search signal. We randomly sample 10 architectures and train until convergence for
our random baseline. Next, we randomly sample 1000 architectures, and select the top 10 performing
architectures with GHN generated weights, which we refer to as GHN Top. Our reported search cost
includes both the GHN training and evaluation phase. Shown in Table~\ref{table:Results1}, the GHN
achieves competitive results with nearly an order of magnitude reduction in search cost.

In Table~\ref{table:Results2}, we compare with methods which use more advanced search methods, such
as reinforcement learning and evolution. Once again, we sample 1000 architectures and use the GHN to
select the top 10. To make a fair comparison for random search, we train the top 10 for a short
period before selecting the best to train until convergence. The accuracy reported for GHN Top-Best
is the average of 5 runs  of the same final architecture. Note that all methods in
Table~\ref{table:Results2} use CutOut~\citep{devriescutout17}. GHN achieves very competitive results
with a simple random search algorithm, while only using a fraction of the total search cost. Using
advanced search methods with GHNs may bring further gains.

\subsubsection{ImageNet-Mobile}
We also run our GHN algorithm on the ImageNet dataset \citep{russakovsky2015imagenet}, which
contains 1.28 million training images. We report the top-1  accuracy on the 50,000 validation
images. Following existing literature, we conduct the ImageNet experiments in the mobile setting,
where the model is constrained to be under 600M FLOPS. We directly transfer the best architecture
block found in the CIFAR-10 experiments, using an initial convolution layer of stride 2 before
stacking 14 blocks with scale reduction at blocks 1, 2, 6 and 10. The total number of flops is
constrained by choosing the initial number of channels. We follow existing NAS methods on the
training procedure of the final architecture; details can be found in the Appendix. As shown in
Table \ref{table:Results3} the transferred block is competitive with other NAS methods which require
a far greater search cost.