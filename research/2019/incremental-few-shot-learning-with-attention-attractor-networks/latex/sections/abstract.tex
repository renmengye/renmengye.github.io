% !TEX root = ../main.tex
\begin{abstract}
Machine learning classifiers are often trained to recognize a set of pre-defined classes. However,
in many applications, it is often desirable to have the flexibility of learning additional concepts,
with limited data and without re-training on the full training set. This paper addresses this
problem, {\it \ourproblemsmall}, where a regular classification network has already been trained to
recognize a set of base classes, and several extra novel classes are being considered, each with
only a few labeled examples. After learning the novel classes, the model is then evaluated on the
overall classification performance on both base and novel classes. To this end, we propose a
meta-learning model, the Attention Attractor Network, which regularizes the learning of novel
classes. In each episode, we train a set of new weights to recognize novel classes until they
converge, and we show that the technique of recurrent back-propagation can back-propagate through
the optimization process and facilitate the learning of these parameters. We demonstrate that the
learned attractor network can help recognize novel classes while remembering old classes without the
need to review the original training set, outperforming various baselines.
\end{abstract}