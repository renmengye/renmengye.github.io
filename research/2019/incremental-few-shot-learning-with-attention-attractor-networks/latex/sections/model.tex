% !TEX root = ../main.tex
\section{Model}
In this section, we first define the setup of incremental few-shot learning, and then we introduce
our new model, the Attention Attractor Network, which attends to the set of base classes according
to the few-shot training data by using the attractor regularizing term. Figure~\ref{fig:model}
illustrates the high-level model diagram of our method.
\subsection{\ourproblem}
The outline of our meta-learning approach to incremental few-shot learning is: (1) We learn a fixed
feature representation and a classifier on a set of base classes; (2) In each training and testing
episode we train a novel-class classifier with our meta-learned regularizer; (3) We optimize our
meta-learned regularizer on combined novel and base classes classification, adapting it to perform
well in conjunction with the base classifier. Details of these stages follow.

\paragraph{Pretraining Stage:} We learn a base model for the regular supervised classification task
on dataset $\{(x_{a,i},y_{a,i})\}_{i=1}^{N_a}$ where $x_{a,i}$ is the $i$-th example from dataset
$\mathcal{D}_a$ and its labeled class $y_{a,i}\in\{1,2,...,K\}$. The purpose of this stage is to
learn both a good base classifier and a good representation. The parameters of the base classifier
are learned in this stage and will be fixed after pretraining. We denote the parameters of the top
fully connected layer of the base classifier $W_a \in \mathbb{R}^{D\times K}$ where $D$ is the
dimension of our learned representation.
\paragraph{Incremental Few-Shot Episodes:} A few-shot dataset $\mathcal{D}_b$ is presented, from
which we can sample few-shot learning episodes $\mathcal{E}$. Note that this can be the same data
source as the pretraining dataset $\mathcal{D}_a$, but sampled episodically. For each $N$-shot
$K'$-way episode, there are $K'$ novel classes disjoint from the base classes. Each novel class has
$N$ and $M$ images from the support set $S_b$ and the query set $Q_b$ respectively. Therefore, we
have $\mathcal{E} = (S_b, Q_b), S_b = (x_{b,i}^S, y_{b,i}^S)_{i=1}^{N \times K'}, Q_b = (x_{b,i}^Q,
y_{b,i}^Q)_{i=1}^{M
\times K'}$ where $y_{b,i} \in \{K+1,...,K+K'\}$. $S_b$ and $Q_b$ can be regarded as this episodes
training and validation sets. Each episode we learn a classifier on the support set $S_b$ whose
learnable parameters $W_b$ are called the \textit{fast weights} as they are only used during this
episode. To evaluate the performance on a joint prediction of both base and novel classes, i.e., a
$(K+K')$-way classification, a mini-batch $Q_a=\{(x_{a,i}, y_{a,i})\}_{i=1}^{M \times K}$ sampled
from $\mathcal{D}_a$ is also added to $Q_b$ to form $Q_{a+b} = Q_a \cup Q_b$. This means that the
learning algorithm, which only has access to samples from the novel classes $S_b$, is evaluated on
the \emph{joint} query set $Q_{a+b}$.
\paragraph{Meta-Learning Stage:} In meta-training, we iteratively sample few-shot episodes
$\mathcal{E}$ and try to learn the meta-parameters in order to minimize the joint prediction loss on
$Q_{a+b}$. In particular, we design a regularizer $R(\cdot, \theta)$ such that the \textit{fast
weights} are learned via minimizing the loss $\ell(W_b,S_b)+R(W_b, \theta)$ where $\ell(W_b,S_b)$ is
typically cross-entropy loss for few-shot classification. The meta-learner tries to learn
meta-parameters $\theta$ such that the optimal \textit{fast weights} $W_b^*$ w.r.t. the above loss
function performs well on $Q_{a+b}$. In our model, meta-parameters $\theta$ are encapsulated in our
attention attractor network, which produces regularizers for the fast weights in the few-shot
learning objective.

\paragraph{Joint Prediction on Base and Novel Classes:} We now introduce the details of our joint
prediction framework performed in each few-shot episode. First, we construct an episodic classifier,
\eg, a logistic regression (LR) model or a multi-layer perceptron (MLP), which takes the learned
image features as inputs and classifies them according to the few-shot classes.

During training on the support set $S_b$, we learn the \textit{fast weights} $W_{b}$ via minimizing
the following regularized cross-entropy objective, which we call the {\it episodic objective}:
\begin{equation}
\label{eq:general_form}
L^{S}(W_{b}, \theta) = - \frac{1}{NK'}\sum_{i=1}^{N K'} \sum_{c=K+1}^{K+K'}
y_{b,i,c}^S \log \hat{y}_{b,i,c}^S  + R(W_b,\theta).
\end{equation}
This is a general formulation and the specific functional form of the regularization term
$R(W_b,\theta)$ will be specified later. The predicted output $\hat{y}_{b,i}^S$ is obtained via,
$\hat{y}_{b,i}^S = \softmax(\left[W_{a}^\top x_{b,i}, h(x_{b,i}; W_b^\ast) \right])$, where
$h(x_{b,i})$ is our classification network and $W_{b}$ is the fast weights in the network. In the
case of LR, $h$ is a linear model: $h(x_{b,i}; W_{b}) = W_b^\top x_{b,i}$. $h$ can also be an MLP
for more expressive power.

During testing on the query set $Q_{a+b}$, in order to predict both base and novel classes, we
directly augment the softmax with the fixed base class weights $W_a$, $\hat{y}_{i}^Q =
\softmax(\left[W_{a}^\top x_{i}, h(x_{i}; W_b^\ast) \right])$, where ${W}_{b}^{\ast}$ are the
optimal parameters that minimize the regularized classification objective in
Eq.~(\ref{eq:general_form}).

\subsection{\ourmodel}
Directly learning the few-shot episode, e.g., by setting $R(W_b,\theta)$ to be zero or simple
weight decay, can cause catastrophic forgetting on the base classes. This is because $W_b$ which is
trained to maximize the correct novel class probability can dominate the base classes in the joint
prediction. In this section, we introduce the Attention Attractor Network to address this problem.
The key feature of our attractor network is the regularization term $R(W_b, \theta)$:
% \vskip -5mm
\begin{equation}
R(W_b, \theta) = 
\sum_{k^{\prime} = 1}^{K^{\prime}} 
(W_{b,k^{\prime}} - u_{k^{\prime}})^\top \diag(\exp(\gamma)) (W_{b,k^{\prime}} - u_{k^{\prime}}),
\end{equation}
where $u_{k^{\prime}}$ is the so-called \textit{attractor} and $W_{b,k^{\prime}}$ is the
$k^{\prime}$-th column of $W_b$. This sum of squared Mahalanobis distances from the attractors adds
a bias to the learning signal arriving solely from novel classes. Note that for a classifier such as
an MLP, one can extend this regularization term in a layer-wise manner. Specifically, one can have
separate attractors per layer, and the number of attractors equals the number of output dimension
of that layer.

To ensure that the model performs well on base classes, the attractors $u_{k^{\prime}}$ must contain
some information about examples from base classes. Since we can not directly access these base
examples, we propose to use the \textit{slow weights} to encode such information. Specifically, each
base class has a learned attractor vector $U_k$ stored in the memory matrix $U=[U_1,...,U_K]$. It is
computed as, $U_k = f_{\phi}(W_{a, k})$, where $f$ is a MLP of which the learnable parameters are
$\phi$. For each novel class $k^{\prime}$ its classifier is regularized towards its attractor
$u_{k^{\prime}}$ which is a weighted sum of $U_k$ vectors. Intuitively the weighting is an attention
mechanism where each novel class attends to the base classes according to the level of interference,
i.e. how prediction of new class $k'$ causes the forgetting of base class $k$.

For each class in the support set, we compute the cosine similarity between the average
representation of the class and  base weights $W_a$ then normalize using a softmax function
\begin{align}
a_{k^{\prime}, k} = 
\frac{\exp \left(\tau A(\frac{1}{N}\sum_{j} h_j \mathbbm{1}[y_{b,j} = 
k^{\prime}], W_{a, k}) \right)}
{\sum_{k} \exp \left(\tau A(\frac{1}{N}\sum_{j} h_j \mathbbm{1}[y_{b,j} = 
k^{\prime}], W_{a, k}) \right)},
\end{align}
where $A$ is the cosine similarity function, $h_j$ are the representations of the inputs in the
support set $S_b$ and $\tau$ is a learnable temperature scalar. $a_{k^{\prime},k}$ encodes a
normalized pairwise attention matrix between the novel classes and the base classes. The attention
vector is then used to compute a linear weighted sum of entries in the memory matrix $U$,
$u_{k^{\prime}} = \sum_k a_{k^{\prime}, k} U_k + U_0$, where $U_0$ is an embedding vector and serves
as a bias for the attractor.

% !TEX root = ../main.tex
\iflatexml
% \begin{wrapfigure}{R}{0.55\textwidth}
% \vspace{-0.25in}
% \begin{minipage}[t]{0.55\textwidth}
\begin{algorithm}
% \begin{small}
\caption{Meta Learning for {\ourproblem}}
\label{alg:energy}
\begin{algorithmic}[1]
\REQUIRE $\theta_0$, $\mathcal{D}_a$, $\mathcal{D}_b$, $h$
\ENSURE $\theta$
\STATE $\theta \gets \theta_0$;
\FOR{$t=1$ ... $T$}
\STATE $\{(x_b^S, y_b^S)\}, \{(x_b^Q,y_b^Q)\} \gets \text{GetEpisode}(\mathcal{D}_b)$;
\STATE $\{x_{a+b}^Q, y_{a+b}^Q\} \gets \text{GetMiniBatch}(\mathcal{D}_a) \cup \{(x_b^Q, y_b^Q)\}$;
\STATE $ $
\REPEAT
    \STATE $L^S \gets \frac{1}{NK'} \sum_i y_{b,i}^S \log \hat{y}_{b,i}^S + R(W_b; \theta)$;
    \STATE $W_b \gets \text{OptimizerStep}(W_b, \nabla_{W_b} L^S)$;
\UNTIL{$W_b$ \text{converges}}

\STATE $\hat{y}_{a+b,j}^Q \gets \softmax([W_a^\top x_{a+b, j}^Q, h( x_{a+b, j}^Q;W_b) ])$;
\STATE $L^Q \gets \frac{1}{2NK'} \sum_j y_{a+b,j}^Q \log \hat{y}_{a+b,j}^Q$;
\STATE \COMMENT{Backprop through the above optimization via RBP}
\STATE \COMMENT{A dummy gradient descent step}
\STATE $W_b' \gets W_b - \alpha \nabla_{W_b} L^S$;% \ \ \ \COMMENT{A dummy gradient descent step}
\STATE $J \gets \frac{\partial W_b'}{\partial W_b}$; $v \gets \frac{\partial L^Q}{\partial W_b}$; $g \gets v$;

\REPEAT
\STATE $v \gets J^\top v - \epsilon v$; $g \gets g + v$;
\UNTIL{$g$ \text{converges}}
\STATE ~\\
\STATE $\theta \gets \text{OptimizerStep}(\theta, g^\top \frac{\partial W_b^{\prime}}{\partial \theta})$
\ENDFOR
\end{algorithmic}
% \end{small}
\end{algorithm}
% \end{minipage}
% \vspace{-0.1in}
% \end{wrapfigure}
\else
\begin{wrapfigure}{R}{0.55\textwidth}
\vspace{-0.25in}
\begin{minipage}[t]{0.55\textwidth}
\begin{algorithm}[H]
\begin{small}
\caption{Meta Learning for {\ourproblem}}
\label{alg:energy}
\begin{algorithmic}[1]
\REQUIRE $\theta_0$, $\mathcal{D}_a$, $\mathcal{D}_b$, $h$
\ENSURE $\theta$
\STATE $\theta \gets \theta_0$;
\FOR{$t=1$ ... $T$}
\STATE $\{(x_b^S, y_b^S)\}, \{(x_b^Q,y_b^Q)\} \gets \text{GetEpisode}(\mathcal{D}_b)$;
\STATE $\{x_{a+b}^Q, y_{a+b}^Q\} \gets \text{GetMiniBatch}(\mathcal{D}_a) \cup \{(x_b^Q, y_b^Q)\}$;
\STATE $ $
\REPEAT
    \STATE $L^S \gets \frac{1}{NK'} \sum_i y_{b,i}^S \log \hat{y}_{b,i}^S + R(W_b; \theta)$;
    \STATE $W_b \gets \text{OptimizerStep}(W_b, \nabla_{W_b} L^S)$;
\UNTIL{$W_b$ \text{converges}}

\STATE $\hat{y}_{a+b,j}^Q \gets \softmax([W_a^\top x_{a+b, j}^Q, h( x_{a+b, j}^Q;W_b) ])$;
\STATE $L^Q \gets \frac{1}{2NK'} \sum_j y_{a+b,j}^Q \log \hat{y}_{a+b,j}^Q$;
\STATE ~\\
\COMMENT{Backprop through the above optimization via RBP}\\
\COMMENT{A dummy gradient descent step}
\STATE $W_b' \gets W_b - \alpha \nabla_{W_b} L^S$;% \ \ \ \COMMENT{A dummy gradient descent step}
\STATE $J \gets \frac{\partial W_b'}{\partial W_b}$; $v \gets \frac{\partial L^Q}{\partial W_b}$; $g \gets v$;

\REPEAT
\STATE $v \gets J^\top v - \epsilon v$; $g \gets g + v$;
\UNTIL{$g$ \text{converges}}
\STATE ~\\
\STATE $\theta \gets \text{OptimizerStep}(\theta, g^\top \frac{\partial W_b^{\prime}}{\partial \theta})$
\ENDFOR
\end{algorithmic}
\end{small}
\end{algorithm}
\end{minipage}
\vspace{-0.1in}
\end{wrapfigure}
\fi

Our design takes inspiration from attractor networks~\citep{attractor,localist}, where for each base
class one learns an ``attractor" that stores the relevant memory regarding that class. We call our
full model  ``dynamic attractors" as they may vary with each episode even after meta-learning. In
contrast if we only have the bias term $U_0$, i.e. a single attractor which is shared by all novel
classes, it will not change after meta-learning from one episode to the other. We call this model
variant the ``static attractor".

In summary, our meta parameters $\theta$ include $\phi$, $U_0$, $\gamma$ and $\tau$, which is on
 the same scale as as the number of paramters in $W_a$. It is important to note that $R(W_b,
 \theta)$ is convex w.r.t. $W_b$. Therefore, if we use the LR model as the classifier, the overall
 training objective on episodes in Eq. (\ref{eq:general_form}) is convex which implies that the
 optimum $W_b^*(\theta,S_b)$ is guaranteed to be unique and achievable. Here we emphasize that the
 optimal parameters $W_b^*$ are functions of parameters $\theta$ and few-shot samples $S_b$.

During meta-learning, $\theta$ are updated to minimize an expected loss of the query set $Q_{a+b}$
which contains both base and novel classes, averaging over all few-shot learning episodes,
\begin{equation}
\min_{\theta} ~~ \mathop{\mathbb{E}}_{\mathcal{E}} \left[ L^Q(\theta,S_b) \right] =
\mathop{\mathbb{E}}_{\mathcal{E}}\left[\sum_{j=1}^{M(K+K')} \sum_{c=1}^{K+K'} y_{j,c} \log
\hat{y}_{j,c}(\theta,S_b)\right],
\end{equation}
where the predicted class is
$
\hat{y}_{j}(\theta,S_b) = \softmax\left(\left[W_{a}^{\top} x_j, h \left( x_j;
W_{b}^{\ast}(\theta,S_b) \right) \right] \right)$.

\subsection{Learning via Recurrent Back-Propagation}
As there is no closed-form solution to the episodic objective (the optimization problem in Eq.
\ref{eq:general_form}), in each episode we need to minimize $L^S$ to obtain $W^*_b$ through an
iterative optimizer. The question is how to efficiently compute $\frac{\partial W_b^*}{\partial
\theta}$, \ie, back-propagating through the optimization. One option is to unroll the iterative
optimization process in the computation graph and use back-propagation through time (BPTT)~\citep{bptt}. 
However, the number of iterations for a gradient-based optimizer to converge can be on
the order of thousands, and BPTT can be computationally prohibitive. Another way is to use the
truncated BPTT~\citep{tbptt} (T-BPTT) which optimizes for $T$ steps of gradient-based optimization,
and is commonly used in meta-learning problems. However, when $T$ is small the training objective
could be significantly biased.

Alternatively, the recurrent back-propagation (RBP) algorithm~\citep{rbp2,rbp3,rbp} allows us to
back-propagate through the fixed point efficiently without unrolling the computation graph and
storing intermediate activations. Consider a vanilla gradient descent process on $W_b$ with step
size $\alpha$. The difference between two steps $\Phi$ can be written as $\Phi(W_b^{(t)}) =
W_b^{(t)} - F(W_b^{(t)})$, where $F(W_b^{(t)}) =W_b^{(t+1)} = W_b^{(t)} - \alpha \nabla
L^S(W_b^{(t)})$. Since $\Phi(W_b^{*}(\theta))$ is identically zero as a function of $\theta$,
using the implicit function theorem we have $\frac{\partial W_b^*}{\partial \theta} =
(I-J_{F,W_b^*}^\top)^{-1} \frac{\partial F}{\partial \theta}$, where $J_{F,W_b^*}$ denotes the
Jacobian matrix of the mapping $F$ evaluated at $W_b^*$. Algorithm~\ref{alg:energy} outlines the key
steps for learning the episodic objective using RBP in the incremental few-shot learning setting.
Note that the RBP algorithm implicitly inverts $(I-J^\top)$ by computing the matrix inverse vector
product, and has the same time complexity compared to truncated BPTT given the same number of
unrolled steps, but meanwhile RBP does not have to store intermediate activations.

\paragraph{Damped Neumann RBP}
To compute the matrix-inverse vector product $(I - J^\top)^{-1} v$, \citet{rbp} propose to use the
Neumann series: $(I-J^\top)^{-1}v  = \sum_{n=0}^\infty (J^\top)^{n}v \equiv \sum_{n=0}^\infty
v^{(n)}$. Note that $J^\top v$ can be computed by standard back-propagation. However, directly
applying the Neumann RBP algorithm sometimes leads to numerical instability. Therefore, we propose
to add a damping term $0 < \epsilon < 1$ to $I-J^\top$. This results in the following update:
$\tilde{v}^{(n)} = (J^\top - \epsilon I)^{n}v$. In practice, we found the damping term with
$\epsilon = 0.1$ helps alleviate the issue significantly.