\documentclass{article}
\pdfoutput=1

% Recommended, but optional, packages for figures and better typesetting:
% \usepackage{array}
\usepackage{microtype}
\usepackage{graphicx}
\usepackage{subfigure}
\usepackage{booktabs} % for professional tables
\usepackage[final,nonatbib]{neurips_2019}

\usepackage{amsmath,amsfonts,bm}
\usepackage{todonotes}
\usepackage{url}
\usepackage{algorithm,algorithmic}
\usepackage{bbm}
\usepackage{multirow}
\usepackage{wrapfig}
\usepackage{enumitem}
\usepackage{xspace}
\usepackage{eqparbox}
\usepackage{setspace}

% \usepackage{authblk}
\input{macros/math_commands.tex}
\renewcommand{\algorithmiccomment}[1]{\eqparbox{COMMENT}{// #1}}
\newcommand{\mengye}[1]{\textcolor{red}{Mengye: #1}}
\newcommand{\renjie}[1]{\textcolor{orange}{Renjie: #1}}
\newcommand{\tb}[1]{\textbf{#1}}
% hyperref makes hyperlinks in the resulting PDF.
% If your build breaks (sometimes temporarily if a hyperlink spans a page)
% please comment out the following usepackage line and replace
% \usepackage{icml2019} with \usepackage[nohyperref]{icml2019} above.

% Attempt to make hyperref and algorithmic work together better:
\newcommand{\theHalgorithm}{\arabic{algorithm}}
\usepackage{hyperref}
\newcommand{\ourmodel}{Attention Attractor Networks}
\newcommand{\ourmodelsmall}{attention attractor networks}
\newcommand{\ourproblem}{Incremental Few-Shot Learning}
\newcommand{\ourproblemsmall}{incremental few-shot learning}
\newcommand{\ua}{\uparrow}
\newcommand{\da}{\downarrow}
\newcommand{\D}{$\Delta \da$}
\newcommand{\Da}{$\Delta_a \da$}
\newcommand{\Db}{$\Delta_b \da$}

\newcommand{\rich}[1]{\textcolor{magenta}{Rich: #1}}

\hyphenpenalty=5000

\makeatletter
\DeclareRobustCommand\onedot{\futurelet\@let@token\@onedot}
\def\@onedot{\ifx\@let@token.\else.\null\fi\xspace}

\def\eg{\emph{e.g}\onedot} \def\Eg{\emph{E.g}\onedot}
\def\ie{\emph{i.e}\onedot} \def\Ie{\emph{I.e}\onedot}
\def\cf{\emph{c.f}\onedot} \def\Cf{\emph{C.f}\onedot}
\def\aka{a.k.a\onedot} \def\Aka{A.k.a\onedot}
\def\etc{\emph{etc}\onedot} \def\vs{\emph{vs}\onedot}
\def\wrt{w.r.t\onedot} \def\dof{d.o.f\onedot}
\def\etal{\emph{et al}\onedot}
\def\arxiv{1}
\makeatother
\begin{document}
\title{{\ourproblem} with {\ourmodel}}
\author{Mengye Ren$^{1,2,3}$, Renjie Liao$^{1,2,3}$, Ethan Fetaya$^{1,2}$, Richard S. Zemel$^{1,2}$\\
${}^1$University of Toronto, ${}^2$Vector Institute, ${}^3$Uber ATG\\
\texttt{\{mren, rjliao, ethanf, zemel\}@cs.toronto.edu}}
\maketitle
\vspace{-0.2in}
% !TEX root = ../main.tex

\ifarxiv
\vspace{-0.05in}
\fi
Semantic concepts are frequently defined by combinations of underlying
attributes. As mappings from attributes to classes are often simple,
attribute-based representations facilitate novel concept learning with zero or
few examples. A significant limitation of existing attribute-based learning
paradigms, such as zero-shot learning, is that the attributes are assumed to be
known and fixed. In this work we study the rapid learning of attributes that
were not previously labeled. Compared to standard few-shot learning of semantic
classes, in which novel classes may be defined by attributes that were relevant
at training time, learning new attributes imposes a stiffer challenge. We found
that supervised learning with training attributes does not generalize well to
new test attributes, whereas self-supervised pre-training brings significant
improvement. We further experimented with random splits of the attribute space
and found that predictability of test attributes provides an informative
estimate of a model's generalization ability.
\vspace{-0.1in}
% !TEX root = ../main.tex
\section{Introduction}
\vspace{-0.1in}
In machine learning, many paradigms exist for training and evaluating models: standard
train-then-evaluate, few-shot learning, incremental learning, continual learning, and so forth. None
of these paradigms well approximates the naturalistic conditions that humans and artificial agents
encounter as they wander within a physical environment. Consider, for example, learning and
remembering peoples' names in the course of daily life. We tend to see people in a given
environment---work, home, gym, etc. We tend to repeatedly revisit those environments, with different
environment base rates, nonuniform environment transition probabilities, and nonuniform base rates
of encountering a given person in a given environment. We need to recognize when we do not know a
person, and we need to learn to recognize them the next time we encounter them. We are not always
provided with a name, but we can learn in a semi-supervised manner. And every training trial is
itself an evaluation trial as we repeatedly use existing knowledge and acquire new knowledge. In
this article, we propose a novel paradigm, \emph{online contextualized few-shot learning}, that
approximates these naturalistic conditions, and we develop deep-learning architectures well suited
for this paradigm.

In traditional few-shot learning (FSL)~\citep{omniglot,matchingnet}, training is episodic. Within an
isolated episode, a set of new classes is introduced with a limited number of labeled examples per
class---the \textit{support}  set---followed by evaluation on an unlabeled \textit{query} set. While
this setup has inspired the development of a multitude of meta-learning algorithms which can be
trained to rapidly learn novel classes with a few labeled examples, the algorithms are focused
solely on the few classes introduced in the current episode; the classes learned are not carried
over to future episodes. Although incremental learning and continual learning
methods~\citep{icarl,rebalance} address the case where classes are carried over, the episodic
construction of these frameworks seems artificial: in our daily lives, we do not learn new objects
by grouping them with five other new objects, process them together, and then move on.

To break the rigid, artificial structure of continual and few-shot learning, we propose a new
continual few-shot learning setting where environments are revisited and the total number of novel
object classes increases over time. Crucially, model evaluation happens on each trial, very much
like the setup in online learning. When encountering a new class, the learning algorithm is expected
to indicate that the class is ``new,'' and it is then expected to recognize subsequent instances of
the class once a label has been provided.

When learning continually in such a dynamic environment, contextual information can guide learning
and remembering. Any structured sequence provides \emph{temporal context}: the instances encountered
recently are predictive of instances to be encountered next. In natural environments, \emph{spatial
context}---information in the current input weakly correlated with the occurrence of a particular
class---can be beneficial for retrieval as well. For example, we tend to see our  boss in an office
setting, not in a bedroom setting. Human memory retrieval benefits from both  spatial and temporal
context~\citep{Howard2017, foundationsmemory}. In our online few-shot learning setting, we provide
spatial context in the presentation of each instance and temporal structure to sequences, enabling
an agent to learn from both spatial and temporal context. Besides developing and experimenting on a
toy benchmark using handwritten characters~\citep{omniglot}, we also propose a new large-scale
benchmark for online contextualized few-shot learning derived from indoor panoramic
imagery~\citep{matterport}. In the toy benchmark, temporal context can be defined by the
co-occurrence of character classes. In the indoor environment, the context---temporal and
spatial---is a natural by-product as the agent wandering in between different rooms.

We propose a model that can exploit contextual information, called \emph{\ourmodel{}}
(\emph{\ourmodelshort{}}), which incorporates an RNN to encode contextual information and a separate
prototype memory to remember previously learned classes (see Figure~\ref{fig:mainmodel}). This model
obtains significant gains on few-shot classification performance compared to models that do not
retain a memory of the recent past. We compare to classic few-shot algorithms extended to an online
setting, and
\ourmodelshort{} consistently achieves the best performance.

\looseness=-1000
The main contributions of this paper are as follows. First, we define an \emph{online contextualized
few-shot learning (OC-FSL)} setting to mimic naturalistic human learning. Second, we build three
datasets: 1) {\it \ourchar{}} is based on handwritten characters from
Omniglot~\citep{omniglot}; 2) {\it \ourimg{}} is based on images from ImageNet~\citep{imagenet}; and 3) {\it \ourroom{}} is our new few-shot learning dataset
based on indoor imagery~\citep{matterport}, which resembles the visual experience of a wandering
agent. Third, we benchmark classic FSL methods and also explore our \ourmodelshort{} model, which
combines the strengths of RNNs for modeling temporal context and Prototypical Networks
\citep{protonet} for memory consolidation  and rapid learning.
% !TEX root = ../main.tex
\section{Related Work}
\textbf{Attention mechanism in deep learning:}
Human and other primate visual perception systems feature visual attention to reduce the complexity
of the scene and speed-up inference~\cite{neurobiology,saliencyvisattend}. Earlier studies in visual
saliency aimed to predict human gaze with no particular task in mind~\cite{predicthuman}. Attention
mechanisms nowadays are built in as  part of  end-to-end models to optimize towards specific tasks.
The attention modules are typically implemented as multiplicative gates to select features. This
schema has shown to improve performance and interpretability on downstream tasks such as object
recognition~\cite{visattend,attendrbm,resattn}, instance segmentation~\cite{recattend}, image
captioning~\cite{showattendtell}, question answering~\cite{coattend,san}, as well as other natural language processing applications~\cite{machinetrans,transformer,bert}. The visualization of
the end-to-end learned attention suggests that deep attention-based models have an intelligent
understanding of the inputs by focusing on the most informative parts of the input.

\textbf{Sparse activation in neural networks:}
Sparse coding models~\cite{sparsecoding} use an
overcomplete dictionary to achieve sparse activation in the feature space. In modern convolutional neural networks (CNNs), sparsity
is typically brought by the widespread use of ReLU activation functions, but these
are rather unstructured, and speed-up has only been shown on specially designed
hardware~\cite{cnvlutin,relusparse}. Structured spatial sparsity, on the other hand, can be made
efficient by using a sparse convolution operator~\cite{perforatedcnn,sbnet,submanifold}, which in turn allows the
network to shift its focus on more difficult parts of the
inputs~\cite{adaptivecomp,nopixelequal,sbnet,pag}. In self-driving, \cite{prioritize} proposed a ranking
function to prioritize computations that would have the most impact on motion planning. Weight
pruning~\cite{sparsecnn,netslim} is another popular way to achieve sparsity in the parameter space,
which is an orthogonal direction to our  method.

\textbf{Attention and loss weighting in multi-task learning:}
Our end-to-end self-driving network is an instance of multi-task learning as all three
tasks---perception, prediction and motion planning---are simultaneously solved by individual output
branches with shared features. It is common to use a summation of all the loss functions, but
sometimes there are conflicting objectives among the tasks. Prior literature in multi-task learning
has studied dynamic weighting towards different loss components, by using training signals such as
uncertainty~\cite{mtluncertain}, gradient norm~\cite{gradnorm}, difficulty
level~\cite{dynamicprioritize}, or entirely data-driven objectives~\cite{adaptiveweight,l2rw}. In
\cite{adaptiveweight,l2rw}, task and example weights are learned by optimizing the performance of the
main task. The attention mechanism has also been used in multi-task learning: in \cite{e2emtl}, a
network applies task-specific attention masks on shared features to encourage the outputs to be more
selective. Similar to dynamic loss weighting models~\cite{adaptiveweight}, we exploit the learned
attention towards weighting instance detection losses. Instead of using multiple attentions, as was
done in~\cite{e2emtl}, we use one single attention mask to optimize our main task: driving.

\textbf{Safety-driven learnable motion planning:}
One of the primary motivations of introducing attention into an end-to-end motion planning network is
to improve safety. Traditionally, safety for self-driving models was done in terms of
formal model checking and validation~\cite{formalsafety,combinatorialsafe,setsafety,failsafe,pnpsim}. More
recently, with the widely available driving data, imitation learning has been introduced in
self-driving to learn from cautious human driving~\cite{nmp,baidu,jointplt,pthree,dsdnet}. Safety has also been
considered in terms of explicitly learning a risk-sensitive measure from human
demonstration~\cite{riskirl,riskgail}. In our work, although safety is not explicitly encoded in our
loss function, we have experimentally verified that the sparse attention models are significantly
better at avoiding collisions.
% !TEX root = marvin.tex
\section{Problem Definition}

In this section we first provide a precise definition of the multi-agent mapping problem. We then
propose in the next section a decentralized deep neural network for coordinating a fleet of vehicles
to solve this mapping problem. Formally, given a strongly connected directed  graph $G(V,E)$
representing the road connectivity, we would like to produce a routing path for a set of $L$ agents
$\{p^{(i)}\}_{i=1}^L$ such that each vertex $v$ in $V$ is covered $M_v$ times in total across all
agents. We  consider the real-world  setting where 1) $M_v$ is unknown to all agents until the
number has been reached (i.e., only success/failure is revealed upon each action) and 2) only local
traffic information can be observed.

We consider a decentralized setting, where each agent gathers local observations and
information communicated from other agents, and outputs the route it needs to take in the next step.
Here we assume that each agent can broadcast to the rest of the fleet as this is possible with
today's communication technology. We also constrain  the policy of
each agent to be the same, making the system more robust to failure.

Let $a_t^{(i)}$ be the routing action taken by agent $i$ at time $t$, indicating the next node to
traverse. We define a  {\it route} as the sequence of actions $p^{(i)} = [a_0^{(i)}, \dots,
a_N]$, where each action represents an intermediate destination.
We refer the reader to
Table~\ref{tab:notation} for our notation.

The policy of a single agent $i$ can be formulated as a function of 1) the road network graph $G$; 2) local
environment observation $o_t^{(i)}$; 3) the communication messages sent by other agents
$\{\bc_t^{(j)}\}$; and 4) the state of the agent $s_t^{(i)}$.
Thus,
\begin{align}
\{a_t^{(i)}, \bc_t^{(i)}\} = f(G, o_t^{(i)}, \{\bc_{t-1}^{(j)}\}_{j=1}^L; s_t^{(i)}),
\end{align}

\section{Notation}
\label{app:notation}

\begin{table}[h]
    \centering
    \noindent\setlength\tabcolsep{4pt}\setlength{\extrarowheight}{5pt}%
    \begin{tabularx}{0.6\linewidth}{c|*{1}{>{\RaggedRight\arraybackslash}X}}
         &  \textbf{Description} \\\hline
        $\spaceX$ & The domain of the data, e.g. $\bbR^d$ \\
        $\spaceY$ & The range of the data, e.g. $\bbR$ \\
        $\spaceZ$ & The product space $\spaceX \times \spaceY$ \\
        $\calP$ & A collection of distributions over $\spaceZ$ \\
        $\calJ$ & A (finite) subset of distributions in $\calP$ \\
        $P$ & An element of $\calP$\\
        $P^k$ & The product distribution, whose samples correspond to $k$ independent draws from $P$\\
        $P_{1:M}$ & The product distribution, $\Pi_{i=1}^{M} P_i$, for $P_i \in \calP$\\
        $\bar{P}_{1:M}$ & The mixture distribution, $\frac{1}{M}\sum_{i=1}^M P_i$, for $P_i \in \calP$\\
        $\Omega$ & A metric space, containing parameters for each distribution\\
        $\theta$ & A functional, mapping distributions in $\calP$ to parameters in $\Omega$ \\
        $\estimator$ & An estimator $\estimator: \spaceZ^n \rightarrow \calF$\\ 
        $I(X;Y)$ & The mutual information between random variables $X$ and $Y$.\\
        $S, \testS$ & Denotes training datasets drawn \iid from some $P \in \calP$. Typically $S = \{z_1,\ldots,z_n\}$, $\testS = \{z'_1,\ldots,z'_k\}$\\
        $S_\calP$ & Denotes a meta-training set drawn \iid from $P_{1:M}$\\
        $[N]$, for $N \in \bbN$ & Indicates the set $\{1,\ldots,N\}$\\
        $\bball_p(r)$ & The $p$-norm ball of radius $r$, centered at $0$.
    \end{tabularx}
    \caption{Summary of notation used in this manuscript}
    \label{tab:notation}
\end{table}

Assuming that a traffic model $F$ produces the time needed to traverse a route, we would like our
multi-agent system to minimize the following objective:
\begin{align}
\min_{p^(i)}      &&& \sum_{i=1 \dots L} F(p^{(i)}), \\
\text{subject to} &&& \nonumber \sum_i M(p^{(i)}, v) \ge M_v, \ \ \forall v,
\end{align}
where $M(p, v)$ is the number of times node $v$ is visited in a route $p$.

\section{Multi-Agent Routing Value Iteration Network}

In this section, we describe our proposed approach to the multi-vehicle routing problem. Note that the model is running locally in each individual agent,
as this makes it scale well with the number of agents and be more robust to failures. There are two
main components of our approach.  First, the \textbf{communication module} (Fig.~\ref{fig:mainfig}C) works asynchronously to save
messages sent from other agents in a temporary memory unit, and retrieves the content based on an
attention mechanism at the agent level.
% \raquel{We now need to say that at each iteration of planning for this agent...}
Each time an agent needs to select a new destination,
% \raquel{this is each time it needs to select each action, not when it selects, as it makes it seem like the selection already happened}
this information is then sent to the value iteration module for
future planning. Second, the \textbf{value iteration module} (Fig.~\ref{fig:mainfig}B)
runs locally on each agent
% and exchanges information among nodes on the road network graph
and iteratively estimates the value of traveling to each node in the
road network graph for its next route
(Figure~\ref{fig:mainfig}A).
% \raquel{shouldnt this be a route, not an acction? and a route is selected at each time?}
% \raquel{previous sentence of "exchanging information, is  confusing". Rephrase}
Then an attention LSTM planning module iteratively refines the node
features for a fixed number of iterations, and outputs the value function for each node. The node
with the highest value will be considered as the next destination for the agent. We now describe the
value iteration module followed by the communication module.
% As opposed to conventional methods, we
% attempt to include the distance matrix as an explicit input in our architecture, thereby ensuring
% more meaningful information is encoded.
% \raquel{I'll not talk about the distance stuff here... as this is the high level paragraph, but do it when you talk about it}

% !TEX root = ../marvin.tex
\begin{table}[t]
\centering
\begin{small}
\resizebox{0.65\columnwidth}{!}{%
\begin{tabular}{ccc}
\toprule
Name                                   & Dim. & Type             \\
\midrule
Sum of in/out edge weights             & 2    & float            \\
\# of in/out edges                     & 2    & integer          \\
Agent at $v$                           & 1    & binary           \\
$v$ is unexplored                      & 1    & binary           \\
$v$ is fully covered                   & 1    & binary           \\
Dist. from cur. pos. to $v$            & 1    & float            \\
Traffic at $v$                         & 1    & float            \\
$v$ is adjacent                        & 1    & binary           \\
Communication vector                   & 16   & float            \\
\bottomrule
\end{tabular}
}
\end{small}
\caption{Graph input feature representation for node $v$}
\vspace{-0.25in}
\label{tab:node_feature}
\end{table}

\subsection{Value Iteration Module}
Our model operates on a strongly connected graph $G(V,E)$ representing the topology of the road
network. As shown in Fig.~\ref{fig:mainfig}, each street segment forms a node in the graph, and the goal for
each agent is to pick a node to be its next destination. % \raquel{I modified, its not each lane, but each street segment}
Given some
initial node features, our approach refines them for a fixed
number of  iterations of the graph neural network,  decodes the features into a scalar value function for
each node, and then selects the node with the maximum value to be our next destination (see
Fig.~\ref{fig:mainfig}). We now provide more details on each of these steps.
%Thus, the
%value iteration module computs the   ``value function'' of each node, and picks the node with the maximum value.


Let $\bX = \{\bx_1, \bx_2, ... , \bx_n\}$ be the set of initial node feature vectors with $n$ being the
total number of nodes and let $\bU = \{u_1, u_2, ... , u_n\}$ represent the input communication node features.
% \raquel{its odd to talk about processed here...} \raquel{shouldnt last row in table 2 be U? if yes, add it to the notation. Is U 16 D? }
We encode the node input features (see Table~\ref{tab:node_feature}) through a linear layer to serve as
initial features for the value iteration network:
\begin{align}
\bX^{(0)} &= (\bX \mathbin\Vert \bU) W_{\mathrm{enc}} + \bb_\mathrm{enc}.
\end{align}
% \raquel{this is the place to explain why you use a dense distance matrix, and how it is computed}



At each planning iteration $t$, we perform the following iterative update through an LSTM with an
attention module across neighboring nodes:
\begin{equation}
  \bX^{(k+1)} = \bX^{(k)} + \lstm(\mathrm{Att}(\bX^{(k)}, A); \bH^{(k)}),
\end{equation}
for $t=1 \dots K$ and $K$ is the total number of value iteration steps. $\bH^{(t)}$ is the hidden
state of the LSTM, which contains one state vector per node, and $A$ is the adjacency matrix.
As opposed to conventional methods where the binary adjacency matrix is used as the primary input to
the network, we use the Floyd-Warshall algorithm to compute the dense
distance matrix as an explicit input in our architecture, thereby ensuring
more meaningful information can be utilized by our model. In particular, the matrix produced by the Floyd-Warshall
algorithm encodes the pairwise minimum path
distance between any pair of nodes, $D_{i,j} = d(v_i,v_j)$, which we normalize to form our dense
adjacency matrix %. Let $A$ represent said dense distance matrix for the graph.
$A = \frac{D - \mu}{\sigma}$, where $\mu$ is the element-wise mean of $D$, and
$\sigma$ is the element-wise standard deviation. As shown in our experiments, using our dense adjacency
matrix results in significantly better planning than the binary connectivity matrix of GVIN~\cite{gvin}).

\paragraph{Graph attention layer:} Information exchange on the graph level happens in the attention
module ``$\mathrm{Att}$'' which is a transformer layer~\cite{gtn}, that takes in the node features
and the adjacency matrix, and outputs the transformed features. Specifically, we first compute the
key, query, and value vectors for each node:
\begin{align}
  \bQ^{(k)} &= \bX^{(k)} W_q + \bb_q, \\
  \bK^{(k)} &= \bX^{(k)} W_k + \bb_k, \\
  \bV^{(k)} &= \bX^{(k)} W_v + \bb_v.
\end{align}
We then compute the attention between each node and every other node to create an attention matrix
$A_{\text{att}} \in \mathbb{R}^{n \times n}$,
\begin{equation}
  A_{\text{att}} = \bQ^{(k)} \bK^{(k)\top}.
\end{equation}
We combine the graph adjacency matrix $A$ with the attention matrix $A_{\text{att}}$ to represent
edge features as follows:
\begin{equation}
  \tilde{A}^{(k)} = \softmax(g(A_{\text{att}}^{(k)}, A)),
\end{equation}
where $g$ is a learned multi-layer neural network.

% In the equation above, instead of using the graph binary adjacency matrix $A$, we propose to use a
% \textit{dense adjacency matrix} to encode more edge information in order to speed up the information
% exchange process for sparse graphs.  \raquel{this shoudl be said when you first define the A, which is before the equation. All this paragraph}
% Towards this goal, we first compute the pairwise minimum path
% distance between any pair of nodes. $D_{i,j} = d(v_i,v_j)$. We then normalize it to form our dense
% adjacency matrix. $A = \frac{D - \mu}{\sigma}$, where $\mu$ is the element-wise mean of $D$, and
% $\sigma$ is the element-wise standard deviation. As shown in our experiments, using our dense adjacency
% matrix results in significantly better planning than the binary connectivity matrix of GVIN~\cite{gvin}).

The new node values are computed by combining the values produced by all other nodes
according to the attention in the fused attention matrix. The output of the graph attention layer is
then fed to an LSTM module:
\begin{equation}
  \bX^{(k+1)} = \bX^{(k)} + \lstm(\tilde{A}^{(k)} \mathbf{V}^{(k)}; \bH^{(k)}).
\end{equation}

This full process is repeated for a fix number of iterations $k=1, \cdots, K$ before decoding. % \raquel{added this sentence, as this was not clear}

\paragraph{Value masking and decoding:}
After iterating the attention LSTM module for $K$ iterations, we use a linear layer to project the
features into a scalar value function for each node on the graph. We mask out the value of all nodes
that no longer need to be visited since they have been fully mapped, and  take a softmax over
all remaining nodes to get the action probabilities
\begin{align}
\pi(a_i; s_i) = \softmax(\bX^{(K)} W_{\mathrm{dec}} + \bb_{\mathrm{dec}}).
\end{align}
Finally, we take the node that has the maximum probability value to be the next destination. The full route
will be formed by connecting the current node and the destination by using a shortest path algorithm on the weighted
graph. Note that the weights are intended to represent the expected time required to travel from one road segment to the next,
and therefore are computed by dividing the length of the street segment  by the average speed of the vehicles traversing it.
% \raquel{we probably need to say something about how the weights are computed here}

\subsection{Communication Module}
Due to the partial observation nature of our realistic problem setup (\textit{e.g.,} traffic and
multiple revisits), it is beneficial to let the agents communicate their intended trajectories, thereby
encouraging more collaborative behaviours. % \raquel{their information is not just local. Why do you emphasize that? the observation is, but they have merge it with info of other agents, so its not local anymore}
Towards this goal, our proposed model also features an
attention-based communication module, where now  attention is performed over the agents, not the street segments. %\raquel{added clarification of street vs agent, to re-iterate teh difference}
Whenever an agent performs an action, it uses
$\bX^{(K)}$, the final encodings of the value iteration module,
to output the communication vector
% \raquel{this is the full communication, not the one ofr each node. Careful here. Instead talk about the dimensions, and why you pass something that has all the street nodes, e.g., to reflect the beliefs of each agent}
: $\bc^{(i)}$, which is then broadcasted to all agents. We express the communication vector as a set of node
features in order to reflect the structure of the street graph environment.
% \raquel{maybe explain again $\bX^{(K)}$ was after the iterations. Actually I'll not talk about $\bX^{(K)}$ here, but wait until you define things mathematically}
The most recent communication vector
from each sender is temporarily saved on the receiver end. When an agent decides to take a new
action, it applies an agent-level attention layer to aggregate information from its receiver inbox.

Let  $\bC_{\mathrm{in}} = \{\bc^{(1)}, \dots, \bc^{(L)}\} \in \mathbb{R}^{L \times nd}$,  be the messages that an agent receives  from other agents concatenated together,  where $L$ is the number of agents,
$n$ is the number of nodes and $d$ is the features dimension. The agent transforms the communication vectors
to produce a query and a value vector:
\begin{align}
  \bQ_{\mathrm{comm}} &= \bC_{\mathrm{in}} W_{q,{\mathrm{comm}}} + \bb_{q,\mathrm{comm}}, \\
  \bV_{\mathrm{comm}} &= \bC_{\mathrm{in}} W_{v,{\mathrm{comm}}} + \bb_{v,\mathrm{comm}}.
\end{align}
The communication vector last outputted by this given agent is also called upon to produce a key
vector:
\begin{equation}
  \bk_{i,{\mathrm{comm}}} = \bC_{\mathrm{in}, i} W_{k,\mathrm{comm}} + \bb_{k,\mathrm{comm}}.
\end{equation}
This key vector is then similarly dotted with the query vectors from all other agents to form a
learned linear combination of the communication vectors from all the other agents.
% \raquel{why do you need to put this in particulary? just say in the statement above that this includes its own communication vector. This way you save one equation}
%\quin{We phrase it in this way because the key vector is only computed for our current agent at each iteration }
 We can then compute the aggregated communication as
%communication features for the input to the model. We denote this aggregated communication as
$\bU_{i}$:
\begin{align}
\bU_{i} &= \sum_j \alpha_{i,j} \mathbf{V}_j,  \\
\mathbf{\alpha}_i   &= \softmax{(\mathbf{Q_{\mathrm{comm}}} \mathbf{k}_{i,{\mathrm{comm}}})}.
\end{align}
$\bU_{i}$ will then be used as part of the node feature inputs to the value iteration module for the
next step.
% \raquel{careful as this variable is not used in the equations above, and should be used. Go back and make notation compatible.}

\subsection{Learning}
Our proposed network can be trained end-to-end using either imitation learning or reinforcement
learning. Here we explore both possibilities. For imitation learning, we assume there is an
oracle that can solve these planning problems. Note that this relies on a fully observable
environment, and oftentimes the oracle solver will slow down the training process since we generate
a training graph for each rollout.
Alternatively, we also consider training the network using
reinforcement learning, which is more difficult to train but directly optimizes the final objective.
We now describe the learning algorithms in more details.

\paragraph{Imitation learning (IL):}
To generate the ground-truth $a^\star$ that we seek to imitate, we firstly provide an LKH3 solver
with global information about each problem to solve as a  fully observed environment. Based on the
groundtruth past trajectory, each agent tries to predict the next move $a$.  We train the agent
using ``teacher-forcing'' by minimizing the cross entropy loss for each action, summing across the
rollout. In teacher forcing, the agents are forced to perform the same actions as the ground truth rollout at
each timestep, and are penalized when their actions do not match that of their ``teacher''. The loss is averaged across a mini-batch.
\begin{align}
L = - \mathbb{E}[\sum_{t,i} \log \pi(a_t^{(i)\star}; s_t^{(i)}) ],
\end{align}
where $\pi(a; s)$ denotes the probability of taking action $a$ given state $s$.
% \raquel{maybe explain teaching forcing when you mention it, for people starting in the field }
% \raquel{shouldnt we take the - to be outside the expectation?}

\paragraph{Reinforcement learning (RL):}
While imitation learning is effective, expert demonstration may not always be available for realistic environments.  Instead, we can use
reinforcement learning. We use REINFORCE~\cite{reinforce} to train
the network using episodic reinforcement learning, and set the negative total cost of the fully
rolled out traversal to be the reward function, normalized across a mini-batch.
\begin{align}
r &= - \sum_{i} F(p^{(i)}), \quad
\tilde{r} = (r - \mu_r) / \sigma_r, \\
L &= - \mathbb{E}_{\pi}\tilde{r}, \ \
\nabla L = -\mathbb{E}_{\pi}[
\tilde{r} \sum_{t, i} \nabla \log \pi (a_t^{(i)}; s_t^{(i)} )
].
\end{align}

% !TEX root = ../main.tex
\section{Experiments}

\subsection{Datasets}

We evaluate the performance of our model on three datasets: two benchmark few-shot classification
datasets and a novel large-scale dataset that we hope will be useful for future few-shot learning
work.

\textbf{Omniglot} \citep{lake2011oneshot} is a dataset of 1,623 handwritten characters from 50
alphabets. Each character was drawn by 20 human subjects. We follow the few-shot setting proposed by
\citet{vinyals2016matchingnet}, in which the images are resized to $28 \times 28$ pixels and
rotations in multiples of 90$^\circ$ are applied, yielding 6,492 classes in total. These are split
into 4,112 training classes, 688 validation classes, and 1,692 testing classes.

\textbf{\textit{mini}ImageNet} \citep{vinyals2016matchingnet} is a modified version of the ILSVRC-12
dataset \citep{russakovsky2015imagenet}, in which 600 images for each of 100 classes were randomly
chosen to be part of the dataset. We rely on the class split used by \citet{ravi2017oneshot}. These
splits use 64 classes for training, 16 for validation, and 20 for test. All images are of size 84
$\times$ 84 pixels.

\textbf{\textit{tiered}ImageNet} is our proposed dataset for few-shot classification. Like
\textit{mini}Imagenet, it is a subset of ILSVRC-12. However, \textit{tiered}ImageNet represents a
larger subset of ILSVRC-12 (608 classes rather than 100 for \textit{mini}ImageNet). Analogous to
Omniglot, in which characters are grouped into alphabets, \textit{tiered}ImageNet groups classes
into broader categories corresponding to higher-level nodes in the ImageNet \citep{deng2009imagenet}
hierarchy. There are 34 categories in total, with each category containing between 10 and 30
classes. These are split into 20 training, 6 validation and 8 testing categories (details of the
dataset can be found in the supplementary material). This ensures that all of the training classes
are sufficiently distinct from the testing classes, unlike \textit{mini}ImageNet and other
alternatives such as \textit{rand}ImageNet proposed by  \citet{vinyals2016matchingnet}. For example,
``pipe organ'' is a training class and ``electric guitar'' is a test class in the
\citet{ravi2017oneshot} split of  \textit{mini}Imagenet, even though they are both musical
instruments. This scenario would not occur in \textit{tiered}ImageNet since ``musical instrument''
is a high-level category and as such is not split between training and test classes. This represents
a more realistic few-shot learning scenario since in general we cannot assume that test classes will
be similar to those seen in training. Additionally, the tiered structure of \textit{tiered}ImageNet
may be useful for few-shot learning approaches that can take advantage of hierarchical relationships
between classes. We leave such interesting extensions for future work.

\subsection{Adapting the Datasets for Semi-Supervised Learning}
For each dataset, we first create an additional split to separate the images of each class into
disjoint labeled and unlabeled sets. For Omniglot and {\it tiered}ImageNet we sampled 10\% of the
images of each class to form the labeled split. The remaining 90\% can only be used in the unlabeled
portion of episodes. For {\it mini}ImageNet we instead used 40\% of the data for the labeled split
and the remaining 60\% for the unlabeled, since we noticed that 10\% was too small to achieve
reasonable performance and avoid overfitting. We report the average classification scores over 10
random splits of labeled and unlabeled portions of the training set, with uncertainty computed in
standard error (standard deviation divided by the square root of the total number of splits).

We would like to emphasize that due to this labeled/unlabeled split, we are using strictly less
label information than in the previously-published work on these datasets. Because of this, we do
not expect our results to match the published numbers, which should instead be interpreted as an
upper-bound for the performance of the semi-supervised models defined in this work.

Episode construction then is performed as follows. For a given dataset, we create a training episode
by first sampling $N$ classes uniformly at random from the set of training classes ${\cal C}_{\rm
train}$. We then sample $K$ images from the labeled split of each of these classes to form the
support set, and $M$ images from the unlabeled split of each of these classes to form the unlabeled
set. Optionally, when including distractors, we additionally sample $H$ other classes from the set
of training classes and $M$ images from the unlabeled split of each to act as the distractors. These
distractor images are added to the unlabeled set along with the unlabeled images of the $N$ classes
of interest (for a total of $MN + MH$ images). The query portion of the episode is comprised of a
fixed number of images from the labeled split of each of the $N$ chosen classes. Test episodes are
created analogously, but with the $N$ classes (and optionally the $H$ distractor classes) sampled
from ${\cal C}_{\rm test}$. Note that we used $M=5$ for training and $M=20$ for testing, thus also
measuring the ability of the models to generalize to a larger unlabeled set size. We also used
$H=N=5$, i.e.\ used 5 classes for both the labeled classes and the disctractor classes.

In each dataset we compare our three semi-supervised models with two baselines. The first baseline,
referred to as ``Supervised'' in our tables, is an ordinary Prototypical Network that is trained in
a purely supervised way on the labeled split of each dataset. The second baseline, referred to as
``Semi-Supervised Inference'', uses the embedding function learned by this supervised Prototypical
Network, but performs semi-supervised refinement of the prototypes at inference time using a step of
Soft $k$-Means refinement. This is to be contrasted with our semi-supervised models that perform
this refinement both at training time and at test time, therefore learning a different embedding
function. We evaluate each model in two settings: one where all unlabeled examples belong to the
classes of interest, and a more challenging one that includes distractors. Details of the model
hyperparameters can be found in Appendix~\ref{sec:hyperparam} and our online repository\footnote{
Code available at
\url{https://github.com/renmengye/few-shot-ssl-public}}.

% !TEX root = top.tex
\section{Experiments}
In this section, we use our proposed GHN to search for the best CNN architecture for image
classification. First, we evaluate the GHN on the standard CIFAR \citep{krizhevsky2009cifar} and
ImageNet \citep{russakovsky2015imagenet} architecture search benchmarks. Next, we apply GHN on an
``anytime prediction'' task where we optimize the speed-accuracy tradeoff that is key for many
real-time applications. Finally, we benchmark the GHN's  predicted-performance correlation and
explore various factors in an ablation study.
% !TEX root = top.tex
\subsection{NAS benchmarks}
\input{results1}
\input{results2}
\input{results3}

\subsubsection{CIFAR-10}
\label{section:cifar10}
We conduct our initial set of experiments on CIFAR-10 \citep{krizhevsky2009cifar}, which contains 10
object classes and 50,000 training images and 10,000 test images of size 32$\times$32$\times$3. We
use 5,000 images split from the training set as our validation set.

\vspace{-0.25cm}
\paragraph{Search space:} 
Following existing NAS methods, we choose to search for optimal blocks rather than the entire
network. Each block contains 17 nodes, with 8 possible operations. The final architecture is formed
by stacking 18 blocks. The spatial size is halved and the number of channels is doubled after blocks
6 and 12. These settings are all chosen following recent NAS methods
\citep{zoph2016neural,pham2018efficient,liu2018darts}, with details in the Appendix.

\vspace{-0.25cm}
\paragraph{Training:}
For the GNN module, we use a standard GRU cell \citep{cho14gru} with hidden size 32 and 2
layer MLP with hidden size 32 as the recurrent cell function $U$ and message function $M$
respectively. The shared hypernetwork $H \left(\cdot; \vvphi\right)$ is a 2-layer MLP with hidden
size 64. From the results of ablations studies in Section~\ref{section:ablations}, the GHN is
trained with blocks with $N=7$ nodes and $T=5$ propagations under the forward-backward scheme, using
the ADAM optimizer \citep{kingma2015adam}. Training details of the final selected architectures are
chosen to follow existing works and can be found in the Appendix.
\vspace{-0.25cm}
\paragraph{Evaluation:}
First, we compare to similar methods that use random search with a  hypernetwork or a one-shot model
as a surrogate search signal. We randomly sample 10 architectures and train until convergence for
our random baseline. Next, we randomly sample 1000 architectures, and select the top 10 performing
architectures with GHN generated weights, which we refer to as GHN Top. Our reported search cost
includes both the GHN training and evaluation phase. Shown in Table~\ref{table:Results1}, the GHN
achieves competitive results with nearly an order of magnitude reduction in search cost.

In Table~\ref{table:Results2}, we compare with methods which use more advanced search methods, such
as reinforcement learning and evolution. Once again, we sample 1000 architectures and use the GHN to
select the top 10. To make a fair comparison for random search, we train the top 10 for a short
period before selecting the best to train until convergence. The accuracy reported for GHN Top-Best
is the average of 5 runs  of the same final architecture. Note that all methods in
Table~\ref{table:Results2} use CutOut~\citep{devriescutout17}. GHN achieves very competitive results
with a simple random search algorithm, while only using a fraction of the total search cost. Using
advanced search methods with GHNs may bring further gains.

\subsubsection{ImageNet-Mobile}
We also run our GHN algorithm on the ImageNet dataset \citep{russakovsky2015imagenet}, which
contains 1.28 million training images. We report the top-1  accuracy on the 50,000 validation
images. Following existing literature, we conduct the ImageNet experiments in the mobile setting,
where the model is constrained to be under 600M FLOPS. We directly transfer the best architecture
block found in the CIFAR-10 experiments, using an initial convolution layer of stride 2 before
stacking 14 blocks with scale reduction at blocks 1, 2, 6 and 10. The total number of flops is
constrained by choosing the initial number of channels. We follow existing NAS methods on the
training procedure of the final architecture; details can be found in the Appendix. As shown in
Table \ref{table:Results3} the transferred block is competitive with other NAS methods which require
a far greater search cost.
% !TEX root = top.tex
\subsection{Anytime Prediction}
In the real-time setting, the computational budget available can vary for each test case and cannot
be known ahead of time. This is formalized in anytime prediction, \citep{grubb2012speedboost}  the
setting in which for each test example $\rvx$, there is non-deterministic computational budget $B$
drawn from the joint distribution $P(\rvx, B)$. The goal is then to minimize the expected loss $L(f)
= \E\left[ L\left( f(\rvx), B \right)\right]_{P(\rvx, B)}$, where $f(\cdot)$ is the model and
$L(\cdot)$ is the loss for an $f(\cdot)$ that must produce a prediction within the budget $B$.

We conduct experiments on CIFAR-10. Our anytime search space consists of networks with 3 cells
containing 24, 16, and 8 nodes. Each node is given the additional properties: 1) the spatial size it
operates at and 2) if an early-exit classifier is attached to it. A node enforces its spatial size
by pooling or upsampling any input feature maps inputs that are of different scale. Note that while
a naive one-shot model would triple its size to include three different parameter sets at three
different scales, the GHN is negligibly affected by such a change. The GHN uses the area under the
predicted accuracy-FLOPS curve as its selection criteria. The search space, contains various
convolution and pooling operators. Training methodology of the final architectures are chosen to
match \cite{huang2017multi} and can be found in the Appendix.

Figure \ref{fig:test1} shows a comparison with the various methods presented by
\cite{huang2017multi}. Our experiments show that the best searched architectures can outperform the
current state-of-the-art human designed networks. We see the GHN is amenable to the changes proposed
above, and can find efficient architectures with a random search when used with a strong search
space.

\iflatexml
\begin{figure}
  \includegraphics[width=4\linewidth]{figures/anytime_compare.png}
  \captionof{figure}{Comparison with state-of-the-art\\ human-designed networks on CIFAR-10.}
  \label{fig:test1}
\end{figure}
\begin{figure}
  \includegraphics[width=4\linewidth]{figures/anytime_randoms.png}
  \captionof{figure}{Comparison between random 10 and\\ top 10 networks on CIFAR-10.}
  \label{fig:test2}
\end{figure}
\else
\begin{figure}[t]
\vspace{-0.5cm}
\centering
\begin{minipage}{.48\textwidth}
  \centering
  \includegraphics[width=0.8\linewidth]{figures/anytime_compare.pdf}
  \captionof{figure}{Comparison with state-of-the-art\\ human-designed networks on CIFAR-10.}
\label{table:Results4}
  \label{fig:test1}
\end{minipage}%
\begin{minipage}{.48\textwidth}
  \centering
  \includegraphics[width=0.8\linewidth]{figures/anytime_randoms.pdf}
  \captionof{figure}{Comparison between random 10 and\\ top 10 networks on CIFAR-10.}
  \label{fig:test2}
\end{minipage}
\end{figure}
\fi
% !TEX root = top.tex
\subsection{Predicted performance correlation (CIFAR-10)}
\begin{table}[t]
\caption{Benchmarking the correlation between the predicted and true performance of the GHN against SGD and a one-shot model baselines. Results are on CIFAR-10.}
\vspace{-0.2cm}
\label{table:correlation}
\small
\begin{center}
\begin{tabular}{ c c c c c} 
Method & \multicolumn{2}{c}{Computation cost}   & \multicolumn{2}{c}{Correlation}    \\ 
%\hline
 & Initial (GPU hours) & Per arch. (GPU seconds)  & Random-100 & Top-50   \\ 
\hline
SGD 10 Steps & - & 0.9 & 0.26 & -0.05\\
SGD 100 Steps & - & 9 & 0.59 & 0.06\\
SGD 200 Steps & - & 18 & 0.62 & 0.20 \\
SGD 1000 Steps & - & 90 & 0.77 & 0.26 \\
One-Shot & 9.8 & 0.06 & 0.58 & 0.31\\
\hline
\hline
GHN & 6.1 & 0.08 & 0.68 & 0.48
\end{tabular}
\end{center}
\end{table}

In this section, we evaluate  whether the parameters generated from GHN can be indicative of the
final performance. Our metric is the correlation between the accuracy of a model with trained
weights vs. GHN generated weights. We use a fixed set of 100 random architectures that have not been
seen by the GHN during training, and we train them for 50 epochs to obtain our ``ground-truth''
accuracy, and finally compare with the accuracy obtained from GHN generated weights. We report the
Pearson's R score on all 100 random architectures and the top 50 performing architectures (i.e.\
above average architectures). Since we are interested in searching for the best architecture,
obtaining a higher correlation on top performing architectures is more meaningful.

To evaluate the effectiveness of GHN, we further consider two baselines: 1) training a network with
SGD from scratch for a varying number of steps, and 2) our own implementation of the one-shot model
proposed by \citet{pham2018efficient}, where nodes store a set of shared parameters for each
possible operation. Unlike GHN, which is compatible with varying number of nodes, the one-shot model
must be trained with $N=17$ nodes to match the evaluation. The GHN is trained with $N=7$, $T=5$
using forward-backward propagation. These GHN parameters are selected based on the results found in
Section~\ref{section:ablations}.

Table \ref{table:correlation} shows performance correlation and search cost of SGD, the one-shot
model, and our GHN. Note that GHN clearly outperforms the one-shot model, showing the effectiveness
of dynamically predicting parameters based on graph topology. While it takes 1000 SGD steps to
surpasses GHN in the ``Random-100'' setting, GHN is still the strongest in the ``Top-50'' setting,
which is more important for architecture search. Moreover, compared to GHN, running 1000 SGD steps
for every random architecture is over 1000 times more computationally expensive. In contrast, GHN
only requires a pre-training stage of 6 hours, and afterwards, the trained GHN can be used to
efficiently evaluate a massive number of random architectures of different sizes.
% !TEX root = top.tex
\subsection{Ablation Studies (CIFAR-10)}
\label{section:ablations}

\iflatexml
\begin{figure}
  \includegraphics[width=4\linewidth]{figures/nodes.png}
  \caption{Vary number of nodes; $T=5$ , forward-backward}
  \label{fig:sfig2}
\end{figure}
\begin{figure}
  \includegraphics[width=4\linewidth]{figures/tsteps.png}
\caption{ GHN when varying the number of nodes and propagation scheme}
% \label{fig:ghn_hyps}
  \label{fig:sfig1}
\end{figure}
\else
\begin{figure}[t]
\vspace{-1.0cm}
 \begin{center}
\begin{subfigure}{.48\textwidth}
  \includegraphics[width=0.8\linewidth]{figures/nodes.pdf}
  \caption{Vary number of nodes; $T=5$ , forward-backward}
  \label{fig:sfig2}
\end{subfigure}
\begin{subfigure}{.48\textwidth}
  \centering
  \includegraphics[width=0.8\linewidth]{figures/tsteps.pdf}
  \caption{Vary propagation schemes, $N=7$ }
  \label{fig:sfig1}
\end{subfigure}%
\caption{ GHN when varying the number of nodes and propagation scheme}
\label{fig:ghn_hyps}
\end{center}
\vspace{-0.5cm}
\end{figure}
\fi

\vspace{-0.25cm}
\paragraph{Number of graph nodes:}
The GHN is compatible with varying number of nodes - graphs used in training need not be the same
size as the graphs used for evaluation. Figure~\ref{fig:sfig2} shows how GHN performance varies as a
function of the number of nodes employed during training - fewer nodes generally produces better
performance. While the GHN has difficulty learning on larger graphs, likely due to the vanishing
gradient problem, it can generalize well from just learning on smaller graphs. Note that all GHNs
are  tested with the full graph size ($N=17$ nodes).

\vspace{-0.25cm}
\paragraph{Number of propagation steps:}
We now compare  the forward-backward propagation scheme with the regular synchronous propagation
scheme. Note that $T=1$ synchronous step corresponds to one full forward-backward phase. As shown in
Figure~\ref{fig:sfig1},  the forward-backward scheme consistently outperforms the synchronous
scheme. More propagation steps also help improving the performance, with a diminishing return. While
the forward-backward scheme is less amenable to acceleration from parallelization due to its
sequential nature, it is possible to parallelize the evaluation phase across multiple GHNs when
testing the fitness of candidate architectures.

\begin{wraptable}[8]{r}{5.5cm}
\iflatexml
\else
\footnotesize
\fi
\vspace{-0.4cm}
\begin{center}
\begin{tabular}{ c c c c} 
SP & PE & \multicolumn{2}{c}{Correlation}    \\ 
 &&  Random-100 & Top-50   \\ 
\hline
\xmark & \xmark & 0.24 & 0.15\\
\xmark & \cmark  &  0.44 & 0.37\\
\cmark & \cmark  & 0.68 & 0.48 
\end{tabular}
\end{center}
\vspace{-0.1in}
\caption{Stacked GHN Correlation. SP denotes share parameters and PE denotes passing embeddings}
\label{table:stacked}
\end{wraptable}

\vspace{-0.25cm}
\paragraph{Stacked GHN for architectural motifs:}
We also evaluate different design choices of GHNs on representing architectural motifs. We compare
1) individual GHNs, each predicting one block independently, 
2) a stacked GHN where individual GHN's
   pass on their graph embedding without sharing parameters, 
3) a stacked GHN with shared parameters (our proposed approach). 
As shown in Table~\ref{table:stacked},  passing messages between GHN's is crucial, and sharing parameters produces better performance.


% !TEX root = ../main.tex
\section{Conclusion}

In this work, we propose a novel semi-supervised few-shot learning paradigm, where an unlabeled set
is added to each episode. We also extend the setup to more realistic situations where the unlabeled
set has classes not belonging to the labeled classes. To address the problem that current few-shot
classification dataset is too small for a labeled vs.\ unlabeled split and does not have
hierarchical levels of labels, we also introduce a new dataset, \textit{tiered}ImageNet. We propose
several novel extensions of Prototypical Networks, and they show consistent improvements under 
semi-supervised settings compared to our baselines. As future work, we are working on incorporating 
fast weights \citep{ba2016fw,FinnC2017} into our framework so that examples can have different
embedding representation given the contents in the episode.

\input{sections/acknowledgment}
{
\setstretch{0.93}
\bibliography{ref}
\bibliographystyle{abbrv}
}
\setstretch{1.0}
\appendix
\if\arxiv1
% !TEX root = ../main.tex
\section{Regular Few-Shot Classification}
We include standard 5-way few-shot classification results in Table~\ref{tab:fewshot1_novel}. As
mentioned in the main text, a simple logistic regression model can achieve competitive performance
on few-shot classification using pretrained features. Our full model shows similar performance on
regular few-shot classification. This confirms that the learned regularizer is mainly solving the
interference problem between the base and novel classes.
\vspace{0.1in}

\begin{table}[h!]
\begin{small}
\begin{center}
\caption{Regular 5-way few-shot classification on \textit{mini-ImageNet}.
Note that this is purely few-shot, with no base classes. Applying logistic regression on pretrained
features achieves performance on-par with other competitive meta-learning approaches. * denotes our
own implementation.}
\label{tab:fewshot1_novel}
% \begin{minipage}[c]{0.5\textwidth}
% \hfill
% \resizebox{\columnwidth}{!}{
% \begin{tabular}{|c|c|c|c|}
\begin{tabular}{lccc}
% \hline
\toprule
Model        & Backbone & 1-shot                & 5-shot                \\
% \hline\hline 
\midrule                                                             
MatchingNets \citep{matching} 
             & C64      & 43.60                 & 55.30                 \\
Meta-LSTM \citep{metalstm} 
           & C32      & 43.40 $\pm$ 0.77      & 60.20 $\pm$ 0.71      \\
MAML \citep{maml}
             & C64      & 48.70 $\pm$ 1.84      & 63.10 $\pm$ 0.92      \\
RelationNet \citep{relationnet} 
             & C64      & 50.44 $\pm$ 0.82      & 65.32 $\pm$ 0.70      \\
R2-D2  \citep{diffsolver} 
           & C256     & 51.20 $\pm$ 0.60      & 68.20 $\pm$ 0.60      \\
SNAIL \citep{mishra2017meta} 
           & ResNet   & 55.71 $\pm$ 0.99      & 68.88 $\pm$ 0.92      \\
ProtoNet \citep{proto} 
             & C64      & 49.42 $\pm$ 0.78      & 68.20 $\pm$ 0.66      \\
ProtoNet*  \citep{proto} 
             & ResNet   & 50.09 $\pm$ 0.41      & 70.76 $\pm$ 0.19      \\
LwoF \citep{lwof} 
             & ResNet   & 55.45 $\pm$ 0.89      & \tb{70.92} $\pm$ 0.35 \\
% LwoF*        & ResNet   & \textbf{56.97} $\pm$ 0.24 & 70.50 $\pm$ 0.36 \\
% \hline
\midrule
LR           & ResNet   & 55.40 $\pm$ 0.51      & 70.17 $\pm$ 0.46 \\
% LR +S        & ResNet   & 55.06 $\pm$ 0.52      & 70.32 $\pm$ 0.46 \\
Ours Full     & ResNet   & \tb{55.75} $\pm$ 0.51 & 70.14 $\pm$ 0.44 \\
% \hline
\bottomrule
\end{tabular}
% }
% \end{minipage}
\end{center}
\end{small}
\end{table}

\section{Visualization of Few-Shot Episodes}
We include more visualization of few-shot episodes in Figure~\ref{fig:moreviz}, highlighting the differences between our method and ``Dynamic Few-Shot Learning without Forgetting''~\citep{lwof}.
\begin{figure}[h!]
\centering
\iflatexml
\includegraphics[width=6\textwidth]{figures/attractor_progress_app.png}
\else
\begin{minipage}[c]{\textwidth}
\begin{small}
\begin{tabular}{cc}
\includegraphics[width=0.45\textwidth,trim={2.8cm 1cm 2.5cm 1cm},clip]{figures/attractor_progress_0.pdf} & 
\includegraphics[width=0.45\textwidth,trim={2.8cm 1cm 2.5cm 1cm},clip]{figures/lwof_progress_0.pdf}\\
\hline
\includegraphics[width=0.45\textwidth,trim={2.8cm 1cm 2.5cm 1cm},clip]{figures/attractor_progress_3_noleg.pdf} & 
\includegraphics[width=0.45\textwidth,trim={2.8cm 1cm 2.5cm 1cm},clip]{figures/lwof_progress_3_noleg.pdf}\\
\hline
\includegraphics[width=0.45\textwidth,trim={2.8cm 1cm 2.5cm 1cm},clip]{figures/attractor_progress_8_noleg.pdf} & 
\includegraphics[width=0.45\textwidth,trim={2.8cm 1cm 2.5cm 1cm},clip]{figures/lwof_progress_8_noleg.pdf}\\
(a) Ours & (b) LwoF \citep{lwof}
\end{tabular}
\end{small}
\end{minipage}
\fi
\caption{Visualization of 5-shot 64+5-way episodes on \textit{mini}-ImageNet using PCA.}
\label{fig:moreviz}
\end{figure}

\section{Visualization of Attention Attractors}
To further understand the attractor mechanism, we picked 5 semantic classes in
\textit{mini}-ImageNet and visualized their the attention attractors across 20 episodes, shown in
Figure~\ref{fig:attractorviz}. The attractors roughly form semantic clusters, whereas the static
attractor stays in the center of all attractors.
% !TEX root = ../supp.tex
\begin{figure}[h!]
\centering
\iflatexml
\includegraphics[width=4\textwidth]{figures/attractor_tsne.png}
\else
\includegraphics[width=0.6\textwidth,trim={1cm 0.5cm 1.6cm 1cm},clip]{figures/attractor_tsne.pdf}
\fi
\caption{Visualization of example features and attractors using t-SNE. This plot shows a 5-way
5-shot episode on \textit{mini}-ImageNet. 512-dimensional feature vectors and attractor vectors
are projected to a 2-dim space. Color represents the label class of the example. The static
attractor (\textcolor{teal}{teal}) appears at the center of the attention attractors, which roughly
form clusters based on the classes.}
\label{fig:attractorviz}
\end{figure}


\begin{table}
\centering
\caption{Full ablation results on 64+5-way {\it mini}-ImageNet}
\iflatexml
% \resizebox{0.8\textwidth}{!}{
\begin{tabular}{c|cccc|cccc}
\toprule
          & \multicolumn{4}{c|}{1-shot}           & \multicolumn{4}{c}{5-shot} \\
          & Acc. $\ua$            & \D         & \Da    &\Db     & Acc. $\ua$            & \D         & \Da    & \Db     \\
\midrule                                                                                                                
LR        & 52.74 $\pm$ 0.24      & -13.95     & -8.98  & -24.32 & 60.34 $\pm$ 0.20      & -13.60     & -10.81 & -15.97  \\
LR +S     & 53.63 $\pm$ 0.30      & -12.53     & -9.44  & -15.62 & 62.50 $\pm$ 0.30      & -11.29     & -13.84 & -8.75   \\
LR +A     & \tb{55.31} $\pm$ 0.32 & \tb{-11.72}& -12.72 & -10.71  & 63.00 $\pm$ 0.29      & -10.80     & -13.59 & -8.01   \\
\midrule                                                                                                                             
MLP       & 49.36 $\pm$ 0.29      & -16.78     & -8.95  & -24.61 & 60.85 $\pm$ 0.29      & -12.62     & -11.35 & -13.89  \\
MLP +S    & 54.46 $\pm$ 0.31      & -11.74     & -12.73 & -10.74 & 62.79 $\pm$ 0.31      & -10.77     & -12.61 & -8.80   \\
MLP +A    & 54.95 $\pm$ 0.30      & -11.84     & -12.81 & -10.87 & \tb{63.04} $\pm$ 0.30 & \tb{-10.66}& -12.55 & -8.77   \\
\bottomrule
\end{tabular}
% }
\else
\resizebox{0.8\textwidth}{!}{
\begin{tabular}{c|cccc|cccc}
\toprule
          & \multicolumn{4}{c|}{1-shot}           & \multicolumn{4}{c}{5-shot} \\
          & Acc. $\ua$            & \D         & \Da    &\Db     & Acc. $\ua$            & \D         & \Da    & \Db     \\
\midrule                                                                                                                
LR        & 52.74 $\pm$ 0.24      & -13.95     & -8.98  & -24.32 & 60.34 $\pm$ 0.20      & -13.60     & -10.81 & -15.97  \\
LR +S     & 53.63 $\pm$ 0.30      & -12.53     & -9.44  & -15.62 & 62.50 $\pm$ 0.30      & -11.29     & -13.84 & -8.75   \\
LR +A     & \tb{55.31} $\pm$ 0.32 & \tb{-11.72}& -12.72 & -10.71  & 63.00 $\pm$ 0.29      & -10.80     & -13.59 & -8.01   \\
\midrule                                                                                                                             
MLP       & 49.36 $\pm$ 0.29      & -16.78     & -8.95  & -24.61 & 60.85 $\pm$ 0.29      & -12.62     & -11.35 & -13.89  \\
MLP +S    & 54.46 $\pm$ 0.31      & -11.74     & -12.73 & -10.74 & 62.79 $\pm$ 0.31      & -10.77     & -12.61 & -8.80   \\
MLP +A    & 54.95 $\pm$ 0.30      & -11.84     & -12.81 & -10.87 & \tb{63.04} $\pm$ 0.30 & \tb{-10.66}& -12.55 & -8.77   \\
\bottomrule
\end{tabular}
}
\fi
\end{table}

\begin{table*}[t!]
\centering
\caption{Full ablation results on 200+5-way {\it tiered}-ImageNet}
\iflatexml
\begin{tabular}{c|cccc|cccc}
\toprule
          & \multicolumn{4}{c|}{1-shot}           & \multicolumn{4}{c}{5-shot} \\
          & Acc. $\ua$            & \D         & \Da    & \Db    & Acc. $\ua$            & \D         & \Da    & \Db     \\
\midrule                                                                                                              
LR        & 48.84 $\pm$ 0.23      & -10.44     & -11.65 & -9.24  & 62.08 $\pm$ 0.20      & -8.00      & -5.49  & -10.51  \\
LR +S     & 55.36 $\pm$ 0.32      & -6.88      & -7.21  & -6.55  & 65.53 $\pm$ 0.30      & -4.68      & -4.72  & -4.63   \\
LR +A     & 55.98 $\pm$ 0.32      & \tb{-6.07} & -6.64  & -5.51  & 65.58 $\pm$ 0.29      & \tb{-4.39} & -4.87  & -3.91   \\
\midrule                                                                                                                                           
MLP       & 41.22 $\pm$ 0.35      & -10.61     & -11.25 & -9.98  & 62.70 $\pm$ 0.31      & -7.44      & -6.05  & -8.82   \\
MLP +S    & \tb{56.16} $\pm$ 0.32 & -6.28      & -6.83  & -5.73  & \tb{65.80} $\pm$ 0.31 & -4.58      & -4.66  & -4.51   \\
MLP +A    & 56.11 $\pm$ 0.33      & 6.11       & -6.79  & -5.43  & 65.52 $\pm$ 0.31      & -4.48      & -4.91  & -4.05   \\
\bottomrule
\end{tabular}
\else
\resizebox{0.8\textwidth}{!}{
\begin{tabular}{c|cccc|cccc}
\toprule
          & \multicolumn{4}{c|}{1-shot}           & \multicolumn{4}{c}{5-shot} \\
          & Acc. $\ua$            & \D         & \Da    & \Db    & Acc. $\ua$            & \D         & \Da    & \Db     \\
\midrule                                                                                                              
LR        & 48.84 $\pm$ 0.23      & -10.44     & -11.65 & -9.24  & 62.08 $\pm$ 0.20      & -8.00      & -5.49  & -10.51  \\
LR +S     & 55.36 $\pm$ 0.32      & -6.88      & -7.21  & -6.55  & 65.53 $\pm$ 0.30      & -4.68      & -4.72  & -4.63   \\
LR +A     & 55.98 $\pm$ 0.32      & \tb{-6.07} & -6.64  & -5.51  & 65.58 $\pm$ 0.29      & \tb{-4.39} & -4.87  & -3.91   \\
\midrule                                                                                                                                           
MLP       & 41.22 $\pm$ 0.35      & -10.61     & -11.25 & -9.98  & 62.70 $\pm$ 0.31      & -7.44      & -6.05  & -8.82   \\
MLP +S    & \tb{56.16} $\pm$ 0.32 & -6.28      & -6.83  & -5.73  & \tb{65.80} $\pm$ 0.31 & -4.58      & -4.66  & -4.51   \\
MLP +A    & 56.11 $\pm$ 0.33      & 6.11       & -6.79  & -5.43  & 65.52 $\pm$ 0.31      & -4.48      & -4.91  & -4.05   \\
\bottomrule
\end{tabular}
}
\fi
\end{table*}

\section{Dataset Statistics}
In this section, we include more details on the datasets we used in our experiments.
% We include the dataset statistics in Table~\ref{tab:stats}. In \textit{mini}-ImageNet, we use the
% training set for both pretraining and meta-learning. For testing base class classification
% performance, we included the same val/test set as \citep{lwof}. Since the meta-training set is same
% as meta-learning, in each training episode, we masked out the 5 base classes in the base classifier,
% to ``pretend'' they are few-shot classes. In \textit{tiered}-ImageNet, we splits the original
% training set, Train-A and Train-B, for pretraining and meta-learning respectively.

\begin{table}[h]
\begin{small}
\caption{\textit{mini}-ImageNet and \textit{tiered}-ImageNet split statistics}
\vspace{-0.1in}
\label{tab:stats}
\begin{center}
\begin{tabular}{cc|crr|crr}
\toprule
&& \multicolumn{3}{c|}{\textit{mini}-ImageNet}& \multicolumn{3}{c}{\textit{tiered}-ImageNet} \\
Classes                & Purpose & Split         & N. Cls  & N. Img  & Split           & N. Cls   & N. Img \\
\midrule
\multirow{3}{*}{Base}  & Train   & Train-Train   & 64      & 38,400  & Train-A-Train   & 200      & 203,751   \\
                      & Val     & Train-Val     & 64      & 18,748  & Train-A-Val     & 200      & 25,460    \\
                      & Test    & Train-Test    & 64      & 19,200  & Train-A-Test    & 200      & 25,488    \\
\midrule
\multirow{3}{*}{Novel} & Train   & Train-Train   & 64      & 38,400  & Train-B         & 151      & 193,996   \\
                      & Val     & Val           & 16      & 9,600   & Val             & 97       & 124,261   \\
                      & Test    & Test          & 20      & 12,000  & Test            & 160      & 206,209   \\
\bottomrule
\end{tabular}
\end{center}
\end{small}
\vspace{-0.2in}
\end{table}

\subsection{Validation and testing splits for base classes}
In standard few-shot learning, meta-training, validation, and test set have disjoint sets of object
classes. However, in our incremental few-shot learning setting, to evaluate the model performance on
the base class predictions, additional splits of validation and test splits of the meta-training set
are required. Splits and dataset statistics are listed in Table~\ref{tab:stats}. For
\textit{mini}-ImageNet, \citet{lwof} released additional images for evaluating training set, namely
``Train-Val'' and ``Train-Test''. For \textit{tiered}-ImageNet, we split out $\approx$ 20\% of the
images for validation and testing of the base classes.

\subsection{Novel classes}
In \textit{mini}-ImageNet experiments, the same training set is used for both $\mathcal{D}_a$ and
$\mathcal{D}_b$. In order to pretend that the classes in the few-shot episode are novel, following
\citet{lwof}, we masked the base classes in $W_a$, which contains 64 base classes. In other words, we
essentially train for a 59+5 classification task. We found that under this setting, the progress
of meta-learning in the second stage is not very significant, since all classes have already been
seen before.

In \textit{tiered}-ImageNet experiments, to emulate the process of learning novel classes during the
second stage, we split the training classes into base classes (``Train-A'') with 200 classes and novel classes (``Train-B'') with 151 classes, just for meta-learning purpose.
During the first stage the classifier is trained using Train-A-Train data. In each meta-learning episode we sample few-shot examples from the novel classes (Train-B) and a query base set from Train-A-Val.  

{\bf 200 Base Classes (``Train-A''):}

{\tt n02128757, n02950826, n01694178, n01582220, n03075370, n01531178, n03947888, n03884397, n02883205, n03788195, n04141975, n02992529, n03954731, n03661043, n04606251, n03344393, n01847000, n03032252, n02128385, n04443257, n03394916, n01592084, n02398521, n01748264, n04355338, n02481823, n03146219, n02963159, n02123597, n01675722, n03637318, n04136333, n02002556, n02408429, n02415577, n02787622, n04008634, n02091831, n02488702, n04515003, n04370456, n02093256, n01693334, n02088466, n03495258, n02865351, n01688243, n02093428, n02410509, n02487347, n03249569, n03866082, n04479046, n02093754, n01687978, n04350905, n02488291, n02804610, n02094433, n03481172, n01689811, n04423845, n03476684, n04536866, n01751748, n02028035, n03770439, n04417672, n02988304, n03673027, n02492660, n03840681, n02011460, n03272010, n02089078, n03109150, n03424325, n02002724, n03857828, n02007558, n02096051, n01601694, n04273569, n02018207, n01756291, n04208210, n03447447, n02091467, n02089867, n02089973, n03777754, n04392985, n02125311, n02676566, n02092002, n02051845, n04153751, n02097209, n04376876, n02097298, n04371430, n03461385, n04540053, n04552348, n02097047, n02494079, n03457902, n02403003, n03781244, n02895154, n02422699, n04254680, n02672831, n02483362, n02690373, n02092339, n02879718, n02776631, n04141076, n03710721, n03658185, n01728920, n02009229, n03929855, n03721384, n03773504, n03649909, n04523525, n02088632, n04347754, n02058221, n02091635, n02094258, n01695060, n02486410, n03017168, n02910353, n03594734, n02095570, n03706229, n02791270, n02127052, n02009912, n03467068, n02094114, n03782006, n01558993, n03841143, n02825657, n03110669, n03877845, n02128925, n02091032, n03595614, n01735189, n04081281, n04328186, n03494278, n02841315, n03854065, n03498962, n04141327, n02951585, n02397096, n02123045, n02095889, n01532829, n02981792, n02097130, n04317175, n04311174, n03372029, n04229816, n02802426, n03980874, n02486261, n02006656, n02025239, n03967562, n03089624, n02129165, n01753488, n02124075, n02500267, n03544143, n02687172, n02391049, n02412080, n04118776, n03838899, n01580077, n04589890, n03188531, n03874599, n02843684, n02489166, n01855672, n04483307, n02096177, n02088364.}

{\bf 151 Novel Classes (``Train-B''):}

{\tt n03720891, n02090379, n03134739, n03584254, n02859443, n03617480, n01677366, n02490219, n02749479, n04044716, n03942813, n02692877, n01534433, n02708093, n03804744, n04162706, n04590129, n04356056, n01729322, n02091134, n03788365, n01739381, n02727426, n02396427, n03527444, n01682714, n03630383, n04591157, n02871525, n02096585, n02093991, n02013706, n04200800, n04090263, n02493793, n03529860, n02088238, n02992211, n03657121, n02492035, n03662601, n04127249, n03197337, n02056570, n04005630, n01537544, n02422106, n02130308, n03187595, n03028079, n02098413, n02098105, n02480855, n02437616, n02123159, n03803284, n02090622, n02012849, n01744401, n06785654, n04192698, n02027492, n02129604, n02090721, n02395406, n02794156, n01860187, n01740131, n02097658, n03220513, n04462240, n01737021, n04346328, n04487394, n03627232, n04023962, n03598930, n03000247, n04009552, n02123394, n01729977, n02037110, n01734418, n02417914, n02979186, n01530575, n03534580, n03447721, n04118538, n02951358, n01749939, n02033041, n04548280, n01755581, n03208938, n04154565, n02927161, n02484975, n03445777, n02840245, n02837789, n02437312, n04266014, n03347037, n04612504, n02497673, n03085013, n02098286, n03692522, n04147183, n01728572, n02483708, n04435653, n02480495, n01742172, n03452741, n03956157, n02667093, n04409515, n02096437, n01685808, n02799071, n02095314, n04325704, n02793495, n03891332, n02782093, n02018795, n03041632, n02097474, n03404251, n01560419, n02093647, n03196217, n03325584, n02493509, n04507155, n03970156, n02088094, n01692333, n01855032, n02017213, n02423022, n03095699, n04086273, n02096294, n03902125, n02892767, n02091244, n02093859, n02389026.}

\fi
\end{document}