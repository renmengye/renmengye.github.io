% !TEX root = ../main.tex
\begin{figure*}[t]
\centering
\iflatexml
\includegraphics[width=6\textwidth]{figures/mainresult.pdf}
\else
\includegraphics[width=0.95\textwidth]{figures/mainresult.pdf}
\caption{\small Planning performance of our learned sparse attention model compared to other baselines at varying computation budgets (lower is better on both metrics). \textbf{Left}: \ourdata{}; \textbf{Right}: nuScenes. Note that all models except for \textit{NMP} use the same SA-NMP backbone, which can be scaled by changing the depth and width, allowing the computation of \textit{Dense SA-NMP} and \textit{SA-NMP+Learned Attention} to be varied.
}
\label{fig:mainresult}
\end{figure*}

\section{Experimental Evaluation}
We evaluated on a 
real-world driving dataset (\ourdata{}), training on over 1 million frames from 5,000 scenarios and validating on 5,000 frames from 500 scenarios, using both LiDAR and HD-maps.
We also evaluated on  nuScenes v1.0~\cite{nuscenes}, a large-scale public dataset, with a
training set of over 200,000 frames and a test set of 5,000 frames. Due to the inaccurate localization they provide, we omitted HDMaps and only used LiDAR~\cite{pnpnet}.

\subsection{Implementation Details and Metrics}
\label{sec:impl}

\textbf{Training:}
To jointly train SA-NMP with attention, we use pretrained
weights for the backbone and headers from training a SA-NMP
without attention (dense) for two epochs. 
We train all our models with batch size 5 across 16 GPUs in parallel using the Adam \cite{adam} optimizer. We use an initial learning rate of $1 \times 10^{-4}$, and decay of 0.1 at 1.0 and 1.6 epoch(s), for a total of 2.0 epochs.

\textbf{Evaluation:}
To evaluate driving and safety performance, we focus on the following planning metrics which are accumulated over all 6 future timesteps (3s): \textit{Planning L2} is the L2 distance between waypoints of the predicted future ego trajectory and those of the ground-truth trajectory (characterized by human driving). \textit{Collision rate} is the frequency of collisions between the planned ego trajectory and the ground truth trajectories of other actors in the scene. \textit{Lane violation rate} measures the number of lane boundary violations by the planned ego trajectory. 
We do not evaluate this on nuScenes due to the inaccurate localization they provide,

\textbf{Baselines:}
We compare our learned attention to baselines that are end-to-end trained using static attention masks obtained from priors. 
\textit{Road Mask} covers the entire road as provided from the map data. \textit{Vehicle Mask} strictly covers all detections in the input space, obtained from a PSPNet \cite{pspnet} trained for segmentation. \textit{Proximity Mask} is a circular radius around the ego vehicle. \textit{Dense} is
not using sparse attention.

% !TEX root = marvin.tex
  \begin{table*}[t]
  \begin{center}
    \centering
      \resizebox{0.9\textwidth}{!}{%
      \begin{tabular}{c|ccc|ccc|ccc|ccc}
      \toprule
                  & \multicolumn{3}{c}{n = 25, 1 Agent}                & \multicolumn{3}{c}{n = 25, 2 Agents}                   & \multicolumn{3}{c}{n = 50, 2 Agents}            & \multicolumn{3}{c}{n = 100, 5 Agents}              \\
      Method      & Cost          & Gap             & Runtime          & Cost           & Gap             & Runtime             & Cost          & Gap             & Runtime       & Cost          & Gap             & Runtime          \\
      \midrule
      \midrule
      Oracle      & 1.16          & 0.00\%          & 71.3             & 1.28           & 0.00\%          & 438                 & 1.85          & 0.00\%          & 902           & 3.19          & 0.00\%          & 2430             \\
      \midrule
      Random      & 4.45          & 284.9\%         & \textbf{3.15}    & 4.47           & 249.4\%         & \textbf{1.50}       & 8.25          & 345.2\%         & \textbf{1.82} & 18.9          & 492.2\%         & \textbf{2.83}    \\
      Greedy      & 2.12          & 73.7\%          & 3.37             & 2.33           & 81.8\%          & 2.11                & 3.55          & 91.5\%          & 3.57          & 10.4          & 227.0\%         & 22.3             \\
      LKH3        & 1.26          & 8.84\%          & 71.2             & 1.80           & 40.5\%          & 438                 & 2.54          & 37.3\%          & 902           & 6.14          & 92.5\%          & 2430             \\
      \midrule
      GVIN        & 1.37          & 18.8\%          & 52.5             & 1.48           & 15.9\%          & 44.2                & 2.45          & 32.1\%          & 63.4          & 5.41          & 69.6\%          & 48.6             \\
      GAT         & 1.53          & 32.5\%          & 43.0             & 1.56           & 21.6\%          & 29.1                & 2.58          & 39.7\%          & 38.0          & 5.43          & 70.2\%          & 38.2             \\
      AM          & 4.90          & 322.4\%         & 161              & -              & -               & -                   & -             & -               & -             & -             & -               & -                \\
      EAN         & 2.89          & 145.8\%         & 212              & -              & -               & -                   & -             & -               & -             & -             & -               & -                \\
      MARVIN (IL) & 1.37          & 18.0\%          & 62.8             & 1.42           & 11.3\%          & 66.6                & 2.21          & 19.0\%          & 71.5          & \textbf{4.36} & \textbf{36.7\%} & 72.8 \\
      MARVIN (RL) & \textbf{1.25} & \textbf{8.17\%} & 62.8             & \textbf{1.32}  & \textbf{2.87\%} & 56.6                & \textbf{2.12} & \textbf{14.5\%} & 71.4          & 4.62          & 44.9\%          & 72.8 \\
      \bottomrule
      \end{tabular}%
      }
      \caption{Average graph traversal cost on realistic graphs; Time cost in hours; Runtime in milliseconds.}
        \label{tab:1}
        \vspace{-0.2in}
        \end{center}
    \end{table*}

