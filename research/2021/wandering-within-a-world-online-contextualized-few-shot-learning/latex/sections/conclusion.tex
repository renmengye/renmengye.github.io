% !TEX root = ../main.tex
\vspace{-0.1in}
\section{Conclusion}
\vspace{-0.1in}
We proposed online contextualized few-shot learning, OC-FSL, a paradigm for machine learning that
emulates a human or artificial agent interacting with a physical world. It combines multiple
properties to create a challenging learning task: every input must be classified or flagged as
novel, every input is also used for training, semi-supervised learning can potentially improve
performance, and the temporal distribution of inputs is non-IID and comes from a generative model in
which  input and class distributions are conditional on a latent environment with Markovian
transition probabilities. We proposed the \ourroom{} dataset to simulate an agent wandering within a
physical world. We also proposed a new model, CPM, which uses an RNN to extract spatiotemporal
context from the input stream and to provide control settings to a prototype-based FSL model. In the
context of naturalistic domains like \ourroom{}, CPM is able to leverage contextual information to
attain performance unmatched by other state-of-the-art FSL methods.