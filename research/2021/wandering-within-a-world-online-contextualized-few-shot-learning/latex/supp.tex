\documentclass{article}

% if you need to pass options to natbib, use, e.g.:
%     \PassOptionsToPackage{numbers, compress}{natbib}
% before loading neurips_2020

% ready for submission
% \usepackage{neurips_2020}

% to compile a preprint version, e.g., for submission to arXiv, add add the
% [preprint] option:
%     \usepackage[preprint]{neurips_2020}

% to compile a camera-ready version, add the [final] option, e.g.:
%     \usepackage[final]{neurips_2020}

% to avoid loading the natbib package, add option nonatbib:
\usepackage{iclr2021_conference,times}
% Optional math commands from https://github.com/goodfeli/dlbook_notation.
\input{math_commands.tex}

\usepackage[utf8]{inputenc} % allow utf-8 input
\usepackage[T1]{fontenc}    % use 8-bit T1 fonts
\usepackage{hyperref}       % hyperlinks
\usepackage{url}            % simple URL typesetting
\usepackage{booktabs}       % professional-quality tables
\usepackage{amsfonts}       % blackboard math symbols
\usepackage{nicefrac}       % compact symbols for 1/2, etc.
\usepackage{microtype}      % microtypography
\usepackage{setspace}
\usepackage{graphicx}
\usepackage{tikz}
\usepackage{multirow}
\usepackage{amsmath}
\usepackage{bbm}
\usepackage{xcolor}
\usepackage{pifont}% http://ctan.org/pkg/pifont
\usepackage{enumitem}
\usepackage[font=small,labelfont=bf]{caption}
\usepackage{sidecap}

% !TEX root = main.tex
\DeclareMathOperator*{\argmax}{arg\,max}
\DeclareMathOperator*{\argmin}{arg\,min}

\newcommand{\attn}{A}
\newcommand{\temp}{K}
\newcommand{\sattn}{\tilde{A}}
% \newcommand{\cL}{\mathcal{L}}
% \newcommand{\cLcla}{\mathcal{L}_{\text{class}}}
% \newcommand{\cLreg}{\mathcal{L}_{\text{reg}}}
% \newcommand{\cLclaij}{\mathcal{L}_{\text{class},i,j}}
% \newcommand{\cLregij}{\mathcal{L}_{\text{reg},i,j}}
% \newcommand{\cLpln}{\mathcal{L}_{\text{plan}}}
% \newcommand{\cLatn}{\mathcal{L}_{\attn}}

\newcommand{\cC}{C}
\newcommand{\cL}{L}
\newcommand{\cLcla}{L_{\text{cls}}}
\newcommand{\cLreg}{L_{\text{reg}}}
\newcommand{\cLclaij}{L_{\text{cls},i,j}}
\newcommand{\cLregij}{L_{\text{reg},i,j}}
\newcommand{\cLpln}{L_{\text{plan}}}
\newcommand{\cLatn}{L_{\attn}}

\newcommand{\lcla}{\lambda_{\text{cls}}}
\newcommand{\lreg}{\lambda_{\text{reg}}}
\newcommand{\lpln}{\lambda_{\text{plan}}}
\newcommand{\lalp}{\lambda_{A}}
\newcommand{\ind}{\mathbbm{1}}
\newcommand{\bX}{\mathbf{X}}
\newcommand{\Inp}{X}
\newcommand{\bx}{\mathbf{x}}

\DeclareMathOperator*{\sigmoid}{\mathrm{sigmoid}}
\newcommand{\mengye}[1]{{\textcolor{orange}{[MR: #1]}}}
\newcommand{\raquel}[1]{{\color{red}{[Raquel: #1]}}}
\newcommand{\bob}[1]{{\color{blue}{#1}}}
\newcommand{\mengyeedit}[1]{{\textcolor{orange}{[#1]}}}
\newcommand{\ourdata}{Drive4D}


\newcommand{\MR}[1]{{\color{orange}MR: #1}}
% \newcommand{\MRedit}[1]{{\color{orange} [#1]}}
\newcommand{\MRedit}[1]{{\color{orange} #1}}
\newcommand{\MCM}[1]{{\color{blue} #1}}
\newcommand{\RZ}[1]{{\color{magenta}RZ: #1}}

\title{
% Continual Few-Shot Learning with \\Prototypical Contextual Memory Networks
% A Wandering Few-Shot Learner with Temporal Contextual Memory
% Roaming Room to Room: \\Online Contextualized Few-Shot Learning
Wandering Within a World: \\Online Contextualized Few-Shot Learning \\
Supplementary Materials
}

%% NOTE: when we make an arxiv version, please uncomment acknowledgement section below
\author{
  David S.~Hippocampus\\
  \thanks{Use footnote for providing further information
    about author (webpage, alternative address)---\emph{not} for acknowledging
    funding agencies.} \\
  Department of Computer Science\\
  Cranberry-Lemon University\\
  Pittsburgh, PA 15213 \\
  \texttt{hippo@cs.cranberry-lemon.edu}
}

\begin{document}

\maketitle

\begin{abstract}
    In this document we include additional dataset statistics and experimental details. We also include additional visualization to better understand the embeddings and control parameters learned by our CPM model. Note that separate video visualizations can be found in the supplementary folder.
\end{abstract}

% \section{Dataset Statistics}

% \paragraph{\ourchar{} details:}
% For the \ourchar experiments, we use sequences with maximum 150 images, from 5 environments.
% For individual environment, we use a Chinese restaurant process to sample the class distribution. In particular, the probability of sampling a new class is:
% \begin{align}
% p_\text{new} = \frac{k \alpha + \theta}{m + \theta},
% \end{align}
% where $k$ is the number of classes that we have already sampled in the environment, and $m$ is the
% total number of instances we have in the environment. $\alpha$ is set to 0.2 and $\theta$ is set to
% 1.0 for all of our experiments.

% The switching between environments is implemented by a Markov switching process. At each step in the
% sequence there is a constant probability $p_\text{switch}$ that switch to another environment. For
% all the experiments, we set $p_\text{switch}$ to 0.2.

% \paragraph{Additional \ourroom{} statistics:} 
% Statistics of the \ourroom{} is included in Table~\ref{tab:dataset_stats}, in comparison to other
% few-shot and continual learning datasets. Note that since \ourroom{} is collected from a simulated
% environment, with 90 indoor worlds consists of 1.2K panorama images and 1.22M video frames. The
% dataset contains about 6.9K random walk sequences with a maximum of 200 frames per sequence. For
% training we randomly crop 100 frames to form a training sequence. There are 7.0K unique instance
% classes.

% Plots of additional statistics of \ourroom{} are shown in Figure~\ref{fig:additionalstats}. In
% addition to the ones shown in the main paper, the instance and viewpoint also follows long tail
% distributions. The number of objects in each frame follow an exponential distribution.

% \paragraph{Semisupervised labels:}
% Here we describe how we sample the labeled vs. unlabeled flag for each example in the semisupervised
% sequences in both \ourchar{} and \ourroom{} datasets. Due to the imbalance in our class distribution
% (from both the Chinese restaurant process and real data collection), directly masking the label may
% bias the model to ignore the rare seen classes. Ideally, we would like to preserve at least one
% labeled example for each class. Therefore, we designed the following procedure.

% First, for each class $k$, suppose $m_k$ is the number of examples in the sequence that belong to
% the class. Let $\alpha$ be the target label ratio. Then the class-specific label ratio $\alpha_k$
% is:
% \begin{align}
% \alpha_k = (1 - \alpha) \exp(-0.5 (m_k - 1)) + \alpha.
% \label{eq:semisup}
% \end{align}
% We then for each class $k$, we sample a binary Bernoulli sequence based on $\Ber(\alpha_k)$. If a
% class has all zeros in the Bernoulli sequence, we flip the flag of one of the instances to 1 to make
% sure there is at least one labeled instance for each class.

% \paragraph{Dataset splits}
% We include details about our dataset splits in Table~\ref{tab:omniglotsplit} and
% \ref{tab:matterportsplit}.

% \section{Experiment Details}
% \paragraph{Network architecture:}
% For the \ourchar{} experiment we used the common 4-layer CNN for few-shot learning with 64 channels
% in each layer, resulting in a 64-d feature vector~\cite{protonet}. For the \ourroom{} experiment we
% resize the input to 120$\times$160 and we use the ResNet-12 architecture~\cite{tadam} with
% \{32,64,128,256\} channels per block. To represent the feature of the input image with an attention
% mask, we concatenate the global average pooled feature with the attention ROI feature, resulting in
% a 512d feature vector. For the contextual RNN, in both experiments we used an LSTM~\cite{lstm} with
% a 256d hidden state. 

% We use a linear layer to map from the output of the RNN to the features and control variables. We
% obtain $\gamma^{r,w}$ by adding 1.0 to the linear layer output and then applying the softplus
% activation. The bias units for $\beta^{r,w}$ are initialized to 10.0 for all of our experiments. We
% also apply the softplus activation to $\bm$ from the linear layer output.

% % % !TEX root = ../main.tex
% \begin{table}[h]
% \begin{small}
% \caption{\textit{mini}-ImageNet and \textit{tiered}-ImageNet split statistics}
% \label{tab:stats}
% \begin{center}
% \begin{tabular}{cc|crr|crr}
% && \multicolumn{3}{c|}{\textit{mini}-ImageNet}& \multicolumn{3}{c}{\textit{tiered}-ImageNet} \\
% Classes                & Purpose & Split         & N. Cls  & N. Img  & Split           & N. Cls   & N. Img \\
% \hline
% \hline
% \multirow{3}{*}{Base}  & Train   & Train-Train   & 64      & 38,400  & Train-A-Train   & 200      & 203,751   \\
%                       & Val     & Train-Val     & 64      & 18,748  & Train-A-Val     & 200      & 25,460    \\
%                       & Test    & Train-Test    & 64      & 19,200  & Train-A-Test    & 200      & 25,488    \\
% \hline
% \multirow{3}{*}{Novel} & Train   & Train-Train   & 64      & 38,400  & Train-B         & 151      & 193,996   \\
%                       & Val     & Val           & 16      & 9,600   & Val             & 97       & 124,261   \\
%                       & Test    & Test          & 20      & 12,000  & Test            & 160      & 206,209   \\
% \end{tabular}
% \end{center}
% \end{small}
% \end{table}


% \begin{figure}
% \centering
% % \fbox{
% \includegraphics[width=0.95\textwidth,trim={4.5cm 3cm 4.2cm 10cm},clip]{figures/combined_catv2_counts.pdf}
% % }
% \caption{Additional statistics about our \ourroom{} dataset.}
% \label{fig:additionalstats}
% \end{figure}

% % !TEX root = ../main.tex
\begin{table}[t]
\iflatexml
    \begin{tabular}{clll}
    \toprule

    \mr{11}{Train} &
    \texttt{Angelic} &
    \texttt{Grantha} &
    \texttt{N Ko}\\
    & 
    \texttt{Aurek-Besh} &
    \texttt{Japanese (hiragana)} &
    \texttt{Malay}
    \\
    & 
    \texttt{Asomtavruli} &
    \texttt{Sanskrit} &
    \texttt{Ojibwe}
    \\
    & 
    \texttt{Korean} &
    \texttt{Arcadian} &
    \texttt{Greek}
    \\
    & 
    \texttt{Alphabet of the Magi} &
    \texttt{Blackfoot} &
    \texttt{Futurama}
    \\
    & 
    \texttt{Tagalog} &
    \texttt{Anglo-Saxon Futhorc} &
    \texttt{Braille}
    \\
    & 
    \texttt{Cyrillic} &
    \texttt{Burmese} &
    \texttt{Avesta}
    \\
    & 
    \texttt{Gujarati} &
    \texttt{Ge ez} &
    \texttt{Syriac (Estrangelo)}
    \\
    & 
    \texttt{Atlantean} &
    \texttt{Japanese (katakana)} &
    \texttt{Balinese}
    \\
    & 
    \texttt{Atemayar Qelisayer} &
    \texttt{Glagolitic} &
    \texttt{Tifinagh}
    \\
    & 
    \texttt{Latin} &
    \texttt{Inuktitut} &
    \\
    \midrule
    \mr{2}{Val} &
    \texttt{Hebrew} &
    \texttt{Mkhedruli} &
    \texttt{Armenian}\\
    & 
    \texttt{Early Aramaic} &
    \texttt{Bengali} &
    \\
    \midrule

    \mr{5}{Test} & 
    \texttt{Gurmukhi} &
    \texttt{Kannada} & 
    \texttt{Keble} \\
    &
    \texttt{Malayalam} &
    \texttt{Manipuri} &
    \texttt{Mongolian} 
    \\
    &
    \texttt{Old Church Slavonic} &
    \texttt{Oriya} &
    \texttt{Syriac (Serto)} \\
    &
    \texttt{Sylheti} &
    \texttt{Tengwar} &
    \texttt{Tibetan}\\
    &
    \texttt{ULOG}
    \\
    \bottomrule
    \end{tabular}
     \caption{\textbf{Split information for {\it \ourchar{}}}. Each column is an alphabet and we include all the characters in the alphabet in the split. Rows are continuation of lines.}
    \label{tab:omniglotsplit}
\else
    \vspace{-0.5in}
     \caption{\textbf{Split information for {\it \ourchar{}}}. Each column is an alphabet and we include all the characters in the alphabet in the split. Rows are continuation of lines.}
    \begin{center}
    \begin{small}
    \label{tab:omniglotsplit}
    \begin{tabular}{clll}
    \toprule

    \mr{11}{Train} &
    \texttt{Angelic} &
    \texttt{Grantha} &
    \texttt{N Ko}\\
    & 
    \texttt{Aurek-Besh} &
    \texttt{Japanese (hiragana)} &
    \texttt{Malay}
    \\
    & 
    \texttt{Asomtavruli} &
    \texttt{Sanskrit} &
    \texttt{Ojibwe}
    \\
    & 
    \texttt{Korean} &
    \texttt{Arcadian} &
    \texttt{Greek}
    \\
    & 
    \texttt{Alphabet of the Magi} &
    \texttt{Blackfoot} &
    \texttt{Futurama}
    \\
    & 
    \texttt{Tagalog} &
    \texttt{Anglo-Saxon Futhorc} &
    \texttt{Braille}
    \\
    & 
    \texttt{Cyrillic} &
    \texttt{Burmese} &
    \texttt{Avesta}
    \\
    & 
    \texttt{Gujarati} &
    \texttt{Ge ez} &
    \texttt{Syriac (Estrangelo)}
    \\
    & 
    \texttt{Atlantean} &
    \texttt{Japanese (katakana)} &
    \texttt{Balinese}
    \\
    & 
    \texttt{Atemayar Qelisayer} &
    \texttt{Glagolitic} &
    \texttt{Tifinagh}
    \\
    & 
    \texttt{Latin} &
    \texttt{Inuktitut} &
    \\
    \midrule
    \mr{2}{Val} &
    \texttt{Hebrew} &
    \texttt{Mkhedruli} &
    \texttt{Armenian}\\
    & 
    \texttt{Early Aramaic} &
    \texttt{Bengali} &
    \\
    \midrule

    \mr{5}{Test} & 
    \texttt{Gurmukhi} &
    \texttt{Kannada} & 
    \texttt{Keble} \\
    &
    \texttt{Malayalam} &
    \texttt{Manipuri} &
    \texttt{Mongolian} 
    \\
    &
    \texttt{Old Church Slavonic} &
    \texttt{Oriya} &
    \texttt{Syriac (Serto)} \\
    &
    \texttt{Sylheti} &
    \texttt{Tengwar} &
    \texttt{Tibetan}\\
    &
    \texttt{ULOG}
    \\
    \bottomrule
    \end{tabular}
    \end{small}
    \end{center}
\fi
\end{table}

% % !TEX root = ../main.tex
\iflatexml
\begin{table}[t]
\begin{tabular}{cc}
\toprule
\mr{12}{Train} &
\texttt{
r1Q1Z4BcV1o
JmbYfDe2QKZ
29hnd4uzFmX
ULsKaCPVFJR
E9uDoFAP3SH
}\\
&
\texttt{
8WUmhLawc2A
Uxmj2M2itWa
mJXqzFtmKg4
V2XKFyX4ASd
EU6Fwq7SyZv
}\\
&
\texttt{
gYvKGZ5eRqb
gxdoqLR6rwA
YFuZgdQ5vWj
gTV8FGcVJC9
sT4fr6TAbpF
}\\
&
\texttt{
VVfe2KiqLaN
fzynW3qQPVF
WYY7iVyf5p8
VFuaQ6m2Qom
YmJkqBEsHnH
}\\
&
\texttt{
2t7WUuJeko7
pLe4wQe7qrG
cV4RVeZvu5T
XcA2TqTSSAj
ur6pFq6Qu1A
}\\
&
\texttt{
1pXnuDYAj8r
b8cTxDM8gDG
x8F5xyUWy9e
X7HyMhZNoso
aayBHfsNo7d
}\\
&
\texttt{
TbHJrupSAjP
sKLMLpTHeUy
2azQ1b91cZZ
2n8kARJN3HM
Vvot9Ly1tCj
}\\
&
\texttt{
S9hNv5qa7GM
EDJbREhghzL
qoiz87JEwZ2
q9vSo1VnCiC
Vt2qJdWjCF2
}\\
&
\texttt{
VzqfbhrpDEA
D7G3Y4RVNrH
ZMojNkEp431
uNb9QFRL6hY
5LpN3gDmAk7
}\\
&
\texttt{
rqfALeAoiTq
e9zR4mvMWw7
yqstnuAEVhm
zsNo4HB9uLZ
JF19kD82Mey
}\\
&
\texttt{
759xd9YjKW5
wc2JMjhGNzB
rPc6DW4iMge
jh4fc5c5qoQ
HxpKQynjfin
}\\
&
\texttt{
GdvgFV5R1Z5
kEZ7cmS4wCh
vyrNrziPKCB
D7N2EKCX4Sj
PX4nDJXEHrG
}\\

\midrule

\mr{2}{Val} &
\texttt{
s8pcmisQ38h
dhjEzFoUFzH
RPmz2sHmrrY
1LXtFkjw3qL
8194nk5LbLH
}\\
&
\texttt{
jtcxE69GiFV
QUCTc6BB5sX
p5wJjkQkbXX
JeFG25nYj2p
82sE5b5pLXE
}\\
\midrule

\mr{4}{Test} & 
\texttt{
oLBMNvg9in8
r47D5H71a5s
Z6MFQCViBuw
YVUC4YcDtcY
pRbA3pwrgk9
}\\
&
\texttt{
SN83YJsR3w2
gZ6f7yhEvPG
ac26ZMwG7aT
7y3sRwLe3Va
B6ByNegPMKs
}\\
&
\texttt{
UwV83HsGsw3
VLzqgDo317F
17DRP5sb8fy
pa4otMbVnkk
5ZKStnWn8Zo
}\\
&
\texttt{
PuKPg4mmafe
Pm6F8kyY3z2
i5noydFURQK
ARNzJeq3xxb
5q7pvUzZiYa
}\\
\bottomrule
\end{tabular}
\caption{\textbf{Split information for {\it \ourroom{}}}. Each column is the ID of an indoor world. Rows are continuation of the lines.}
\label{tab:matterportsplit}
\end{table}

\begin{table}[t]
\begin{tabular}{cccc}
\toprule
Split & Worlds & Sequences & Frames \\
\midrule
Train & 60     & 4,699      &   823,444     \\
Val   & 20     &  725         &  125,823  \\
Test  & 10     &  1,547      &  271,335      \\
\midrule
Total & 90     & 6,971   & 1,220,602 \\
\bottomrule
\end{tabular}
\caption{{\it \ourroom{}} dataset split size}
\label{tab:matterportsplitsize}
\end{table}
\else
    \begin{table}[t]
    \caption{\textbf{Split information for {\it \ourroom{}}}. Each column is the ID of an indoor world. Rows are continuation of the lines.}
    \label{tab:matterportsplit}
    % \vspace{-0.5in}
    \begin{center}
    \begin{small}
    \begin{tabular}{cc}
    \toprule
    \mr{12}{Train} &
    \texttt{
    r1Q1Z4BcV1o
    JmbYfDe2QKZ
    29hnd4uzFmX
    ULsKaCPVFJR
    E9uDoFAP3SH
    }\\
    &
    \texttt{
    8WUmhLawc2A
    Uxmj2M2itWa
    mJXqzFtmKg4
    V2XKFyX4ASd
    EU6Fwq7SyZv
    }\\
    &
    \texttt{
    gYvKGZ5eRqb
    gxdoqLR6rwA
    YFuZgdQ5vWj
    gTV8FGcVJC9
    sT4fr6TAbpF
    }\\
    &
    \texttt{
    VVfe2KiqLaN
    fzynW3qQPVF
    WYY7iVyf5p8
    VFuaQ6m2Qom
    YmJkqBEsHnH
    }\\
    &
    \texttt{
    2t7WUuJeko7
    pLe4wQe7qrG
    cV4RVeZvu5T
    XcA2TqTSSAj
    ur6pFq6Qu1A
    }\\
    &
    \texttt{
    1pXnuDYAj8r
    b8cTxDM8gDG
    x8F5xyUWy9e
    X7HyMhZNoso
    aayBHfsNo7d
    }\\
    &
    \texttt{
    TbHJrupSAjP
    sKLMLpTHeUy
    2azQ1b91cZZ
    2n8kARJN3HM
    Vvot9Ly1tCj
    }\\
    &
    \texttt{
    S9hNv5qa7GM
    EDJbREhghzL
    qoiz87JEwZ2
    q9vSo1VnCiC
    Vt2qJdWjCF2
    }\\
    &
    \texttt{
    VzqfbhrpDEA
    D7G3Y4RVNrH
    ZMojNkEp431
    uNb9QFRL6hY
    5LpN3gDmAk7
    }\\
    &
    \texttt{
    rqfALeAoiTq
    e9zR4mvMWw7
    yqstnuAEVhm
    zsNo4HB9uLZ
    JF19kD82Mey
    }\\
    &
    \texttt{
    759xd9YjKW5
    wc2JMjhGNzB
    rPc6DW4iMge
    jh4fc5c5qoQ
    HxpKQynjfin
    }\\
    &
    \texttt{
    GdvgFV5R1Z5
    kEZ7cmS4wCh
    vyrNrziPKCB
    D7N2EKCX4Sj
    PX4nDJXEHrG
    }\\

    \midrule

    \mr{2}{Val} &
    \texttt{
    s8pcmisQ38h
    dhjEzFoUFzH
    RPmz2sHmrrY
    1LXtFkjw3qL
    8194nk5LbLH
    }\\
    &
    \texttt{
    jtcxE69GiFV
    QUCTc6BB5sX
    p5wJjkQkbXX
    JeFG25nYj2p
    82sE5b5pLXE
    }\\
    \midrule

    \mr{4}{Test} & 
    \texttt{
    oLBMNvg9in8
    r47D5H71a5s
    Z6MFQCViBuw
    YVUC4YcDtcY
    pRbA3pwrgk9
    }\\
    &
    \texttt{
    SN83YJsR3w2
    gZ6f7yhEvPG
    ac26ZMwG7aT
    7y3sRwLe3Va
    B6ByNegPMKs
    }\\
    &
    \texttt{
    UwV83HsGsw3
    VLzqgDo317F
    17DRP5sb8fy
    pa4otMbVnkk
    5ZKStnWn8Zo
    }\\
    &
    \texttt{
    PuKPg4mmafe
    Pm6F8kyY3z2
    i5noydFURQK
    ARNzJeq3xxb
    5q7pvUzZiYa
    }\\
    \bottomrule
    \end{tabular}
    \end{small}
    \end{center}
    \end{table}

    \begin{table}[t]
    \vspace{-0.5in}
    \caption{{\it \ourroom{}} dataset split size}
    \label{tab:matterportsplitsize}
    \begin{center}
    \begin{small}
    \begin{tabular}{cccc}
    \toprule
    Split & Worlds & Sequences & Frames \\
    \midrule
    Train & 60     & 4,699      &   823,444     \\
    Val   & 20     &  725         &  125,823  \\
    Test  & 10     &  1,547      &  271,335      \\
    \midrule
    Total & 90     & 6,971   & 1,220,602 \\
    \bottomrule
    \end{tabular}
    \end{small}
    \end{center}
    \end{table}
\fi


% \paragraph{Training details:}
% We use the Adam optimizer~\cite{adam} with initial learning rate 1e-3 for all experiments. For
% Omniglot we train the network for 40k steps with batch size of 32 with maximum sequence length 150
% across 2 GPUs and learning rate decay by 0.1$\times$ at 20k and 30k steps. For Matterport 3D we
% train for 20k steps with batch size 8 with maximum sequence length 100 across 4 GPUs and learning
% rate decay by 0.1$\times$ at 8k and 16k steps. We use BCE coeffcient $\lambda=1$  for all
% experiments. In semisupervised experiments, around 30\% examples are labeled ($\alpha = 0.3$, see
% Eq.~\ref{eq:semisup}).

% \paragraph{Data augmentation details:}
% For \ourchar{}, we pad the 28$\times$28 image to 32$\times$32 and then apply random cropping.

% For \ourroom{}, we apply random cropping in the time dimension to get a chunk of 100 frames per input example. We also apply random dropping of 5\% of the frames. We pad the 120$\times$160 images to 126 $\times$ 168 and apply random cropping in each image frame. We also randomly flip the order of the sequence (going forward or backward).

% \section{Additional Visualization of Experimental Results}

% \paragraph{Video visualization:}
% We include video visualization of \ourroom{} sequences in the supplementary zip folder. The class label is shown on the top left corner, and the CPM model prediction is right below the class label. Labeled objects are shown with red solid boxes, and unlabeled ones are shown with gray dashed boxes. Correct model predictions are colored in green whereas wrong ones are colored in red. 

% \paragraph{Prediction accuracy vs. time:}
% Figure~\ref{fig:acctimefull} shows the prediction accuracy of closed-set classes over time. We
% included supervised settings in addition to the unsupervised settings in the main paper.

% % !TEX root = ../main.tex
\begin{figure}[t]
\vspace{-0.5in}
\centering
\iflatexml
\includegraphics[width=6\linewidth]{figures/acctime_full.png}
\else
\setlength{\tabcolsep}{0pt}
\begin{tabular}{cccc}
\includegraphics[height=2.4cm,trim={0.3cm 0cm 0.5cm 0},clip]{figures/omniglot-nossl-time.pdf}
&
\includegraphics[height=2.4cm,trim={2cm 0cm 0cm 0},clip]{figures/omniglot-ssl-time.pdf}
&
\includegraphics[height=2.4cm,trim={1cm 0cm 0.5cm 0},clip]{figures/matterport-nossl-time.pdf}
&
\includegraphics[height=2.4cm,trim={2cm 0cm 0cm 0},clip]{figures/matterport-ssl-time.pdf}
\\
\end{tabular}
\vspace{-0.25in}
\fi
\caption{\textbf{Few-shot classification accuracy over time.} \textbf{Left:} \ourchar{}.
\textbf{Right:} \ourroom{}. \textbf{Top:} Supervised. \textbf{Bottom:} Semi-supervised. An offline
logistic regression (Offline LR) baseline is also included, using pretrained ProtoNet features. It
is trained on all labeled examples except for the one at the current time step.}
\label{fig:acctimefull}
% \vspace{-0.25in}
\end{figure}


% \paragraph{Embedding visualization:}
% Figure~\ref{fig:tsne} shows the learned embedding of each examples in Online ProtoNet vs. our CPM
% model in \ourchar{} sequences, where colors indicate the environment ID. In Online ProtoNet, the
% example features does not reflect the temporal context, and as a result, colors are scattered across
% the space. By contrast, in the CPM embedding visualization, colors are clustered together and we see
% a smoother transition of environments in the embedding space.
% % !TEX root = ../supp.tex
\begin{figure}[t]
\centering
% \vspace{-0.2in}
\iflatexml
    \includegraphics[width=6\linewidth]{figures/omniglot-tsne.png}
\else
    \begin{tabular}{cc}
    Online ProtoNet & CPM (Ours) \\
    \includegraphics[height=3.8cm, trim={2.5cm 1cm 2cm 1cm}, clip]{figures/omniglot-protonet-tsne/tsne-003.pdf}
    &
    \includegraphics[height=3.8cm, trim={2.5cm 1cm 2cm 1cm}, clip]{figures/omniglot-cpm-tsne/tsne-003.pdf}
    \\
    \includegraphics[height=3.8cm, trim={2.5cm 1cm 2cm 1cm}, clip]{figures/omniglot-protonet-tsne/tsne-008.pdf}
    &
    \includegraphics[height=3.8cm, trim={2.5cm 1cm 2cm 1cm}, clip]{figures/omniglot-cpm-tsne/tsne-008.pdf}
    \end{tabular}
\fi
\caption{\textbf{Embedding space visualization of \ourchar{} sequences using t-SNE~\citep{tsne}}. Different color
denotes different environments. Text labels (relative to each environment) are annotated beside the
scatter points. Unlabeled examples shown in smaller circles with lighter colors. \textbf{Left:}
Online ProtoNet; \textbf{Right:} CPM. The embeddings learned CPM model shows a smoother transition
of classes based on their temporal environments.}
\label{fig:tsne}
\end{figure}


% \paragraph{Control parameters vs. time:}
% Finally we visualize the control parameter values predicted by the RNN in
% Figure~\ref{fig:betagamma}. We verify that we indeed need two sets of $\beta$ and $\gamma$ for read
% and write operations separately as they learn different values. $\beta^w$ is smaller than $\beta^r$
% which means that the network is more conservative when writing to prototypes. $\gamma^w$ grows
% larger over time, which means that the network prefers a softer slope when writing
% to prototypes since in the later stage the prototype memory has already stored enough content and it can grow faster, whereas in the earlier stage, the prototype memory is more conservative to avoid embedding vectors to be assigned to wrong clusters.

% % !TEX root = ../supp.tex
\begin{figure}
\centering
% \vspace{-0.1in}
\iflatexml
\includegraphics[width=6\linewidth]{figures/beta-gamma.png}
\else
\begin{tabular}{cc}
\includegraphics[height=4.0cm,trim={0.3cm 0cm 0.5cm 0},clip]{figures/omniglot-beta.pdf}
\quad
&
\includegraphics[height=4.0cm,trim={0.3cm 0cm 0cm 0},clip]{figures/matterport-beta.pdf}
\\
\includegraphics[height=4.0cm,trim={0.3cm 0cm 0.5cm 0},clip]{figures/omniglot-gamma.pdf}
\quad
&
\includegraphics[height=4.0cm,trim={0.3cm 0cm 0cm 0},clip]{figures/matterport-gamma.pdf}
\\
\end{tabular}
\vspace{-0.1in}
\fi
\caption{\textbf{CPM control parameters ($\beta^{r,w}, \gamma^{r,w}$) vs. time.}
\textbf{Left:} \ourchar{} sequences; \textbf{Right:} \ourroom{} sequences; \textbf{Top:}
$\beta^{r,w}$ the threshold parameter; \textbf{Bottom:} $\gamma^{r,w}$ the temperature parameter.}
\label{fig:betagamma}
\end{figure}

% !TEX root = ../main.tex
\section{Regular Few-Shot Classification}
We include standard 5-way few-shot classification results in Table~\ref{tab:fewshot1_novel}. As
mentioned in the main text, a simple logistic regression model can achieve competitive performance
on few-shot classification using pretrained features. Our full model shows similar performance on
regular few-shot classification. This confirms that the learned regularizer is mainly solving the
interference problem between the base and novel classes.
\vspace{0.1in}

\begin{table}[h!]
\begin{small}
\begin{center}
\caption{Regular 5-way few-shot classification on \textit{mini-ImageNet}.
Note that this is purely few-shot, with no base classes. Applying logistic regression on pretrained
features achieves performance on-par with other competitive meta-learning approaches. * denotes our
own implementation.}
\label{tab:fewshot1_novel}
% \begin{minipage}[c]{0.5\textwidth}
% \hfill
% \resizebox{\columnwidth}{!}{
% \begin{tabular}{|c|c|c|c|}
\begin{tabular}{lccc}
% \hline
\toprule
Model        & Backbone & 1-shot                & 5-shot                \\
% \hline\hline 
\midrule                                                             
MatchingNets \citep{matching} 
             & C64      & 43.60                 & 55.30                 \\
Meta-LSTM \citep{metalstm} 
           & C32      & 43.40 $\pm$ 0.77      & 60.20 $\pm$ 0.71      \\
MAML \citep{maml}
             & C64      & 48.70 $\pm$ 1.84      & 63.10 $\pm$ 0.92      \\
RelationNet \citep{relationnet} 
             & C64      & 50.44 $\pm$ 0.82      & 65.32 $\pm$ 0.70      \\
R2-D2  \citep{diffsolver} 
           & C256     & 51.20 $\pm$ 0.60      & 68.20 $\pm$ 0.60      \\
SNAIL \citep{mishra2017meta} 
           & ResNet   & 55.71 $\pm$ 0.99      & 68.88 $\pm$ 0.92      \\
ProtoNet \citep{proto} 
             & C64      & 49.42 $\pm$ 0.78      & 68.20 $\pm$ 0.66      \\
ProtoNet*  \citep{proto} 
             & ResNet   & 50.09 $\pm$ 0.41      & 70.76 $\pm$ 0.19      \\
LwoF \citep{lwof} 
             & ResNet   & 55.45 $\pm$ 0.89      & \tb{70.92} $\pm$ 0.35 \\
% LwoF*        & ResNet   & \textbf{56.97} $\pm$ 0.24 & 70.50 $\pm$ 0.36 \\
% \hline
\midrule
LR           & ResNet   & 55.40 $\pm$ 0.51      & 70.17 $\pm$ 0.46 \\
% LR +S        & ResNet   & 55.06 $\pm$ 0.52      & 70.32 $\pm$ 0.46 \\
Ours Full     & ResNet   & \tb{55.75} $\pm$ 0.51 & 70.14 $\pm$ 0.44 \\
% \hline
\bottomrule
\end{tabular}
% }
% \end{minipage}
\end{center}
\end{small}
\end{table}

\section{Visualization of Few-Shot Episodes}
We include more visualization of few-shot episodes in Figure~\ref{fig:moreviz}, highlighting the differences between our method and ``Dynamic Few-Shot Learning without Forgetting''~\citep{lwof}.
\begin{figure}[h!]
\centering
\iflatexml
\includegraphics[width=6\textwidth]{figures/attractor_progress_app.png}
\else
\begin{minipage}[c]{\textwidth}
\begin{small}
\begin{tabular}{cc}
\includegraphics[width=0.45\textwidth,trim={2.8cm 1cm 2.5cm 1cm},clip]{figures/attractor_progress_0.pdf} & 
\includegraphics[width=0.45\textwidth,trim={2.8cm 1cm 2.5cm 1cm},clip]{figures/lwof_progress_0.pdf}\\
\hline
\includegraphics[width=0.45\textwidth,trim={2.8cm 1cm 2.5cm 1cm},clip]{figures/attractor_progress_3_noleg.pdf} & 
\includegraphics[width=0.45\textwidth,trim={2.8cm 1cm 2.5cm 1cm},clip]{figures/lwof_progress_3_noleg.pdf}\\
\hline
\includegraphics[width=0.45\textwidth,trim={2.8cm 1cm 2.5cm 1cm},clip]{figures/attractor_progress_8_noleg.pdf} & 
\includegraphics[width=0.45\textwidth,trim={2.8cm 1cm 2.5cm 1cm},clip]{figures/lwof_progress_8_noleg.pdf}\\
(a) Ours & (b) LwoF \citep{lwof}
\end{tabular}
\end{small}
\end{minipage}
\fi
\caption{Visualization of 5-shot 64+5-way episodes on \textit{mini}-ImageNet using PCA.}
\label{fig:moreviz}
\end{figure}

\section{Visualization of Attention Attractors}
To further understand the attractor mechanism, we picked 5 semantic classes in
\textit{mini}-ImageNet and visualized their the attention attractors across 20 episodes, shown in
Figure~\ref{fig:attractorviz}. The attractors roughly form semantic clusters, whereas the static
attractor stays in the center of all attractors.
% !TEX root = ../supp.tex
\begin{figure}[h!]
\centering
\iflatexml
\includegraphics[width=4\textwidth]{figures/attractor_tsne.png}
\else
\includegraphics[width=0.6\textwidth,trim={1cm 0.5cm 1.6cm 1cm},clip]{figures/attractor_tsne.pdf}
\fi
\caption{Visualization of example features and attractors using t-SNE. This plot shows a 5-way
5-shot episode on \textit{mini}-ImageNet. 512-dimensional feature vectors and attractor vectors
are projected to a 2-dim space. Color represents the label class of the example. The static
attractor (\textcolor{teal}{teal}) appears at the center of the attention attractors, which roughly
form clusters based on the classes.}
\label{fig:attractorviz}
\end{figure}


\begin{table}
\centering
\caption{Full ablation results on 64+5-way {\it mini}-ImageNet}
\iflatexml
% \resizebox{0.8\textwidth}{!}{
\begin{tabular}{c|cccc|cccc}
\toprule
          & \multicolumn{4}{c|}{1-shot}           & \multicolumn{4}{c}{5-shot} \\
          & Acc. $\ua$            & \D         & \Da    &\Db     & Acc. $\ua$            & \D         & \Da    & \Db     \\
\midrule                                                                                                                
LR        & 52.74 $\pm$ 0.24      & -13.95     & -8.98  & -24.32 & 60.34 $\pm$ 0.20      & -13.60     & -10.81 & -15.97  \\
LR +S     & 53.63 $\pm$ 0.30      & -12.53     & -9.44  & -15.62 & 62.50 $\pm$ 0.30      & -11.29     & -13.84 & -8.75   \\
LR +A     & \tb{55.31} $\pm$ 0.32 & \tb{-11.72}& -12.72 & -10.71  & 63.00 $\pm$ 0.29      & -10.80     & -13.59 & -8.01   \\
\midrule                                                                                                                             
MLP       & 49.36 $\pm$ 0.29      & -16.78     & -8.95  & -24.61 & 60.85 $\pm$ 0.29      & -12.62     & -11.35 & -13.89  \\
MLP +S    & 54.46 $\pm$ 0.31      & -11.74     & -12.73 & -10.74 & 62.79 $\pm$ 0.31      & -10.77     & -12.61 & -8.80   \\
MLP +A    & 54.95 $\pm$ 0.30      & -11.84     & -12.81 & -10.87 & \tb{63.04} $\pm$ 0.30 & \tb{-10.66}& -12.55 & -8.77   \\
\bottomrule
\end{tabular}
% }
\else
\resizebox{0.8\textwidth}{!}{
\begin{tabular}{c|cccc|cccc}
\toprule
          & \multicolumn{4}{c|}{1-shot}           & \multicolumn{4}{c}{5-shot} \\
          & Acc. $\ua$            & \D         & \Da    &\Db     & Acc. $\ua$            & \D         & \Da    & \Db     \\
\midrule                                                                                                                
LR        & 52.74 $\pm$ 0.24      & -13.95     & -8.98  & -24.32 & 60.34 $\pm$ 0.20      & -13.60     & -10.81 & -15.97  \\
LR +S     & 53.63 $\pm$ 0.30      & -12.53     & -9.44  & -15.62 & 62.50 $\pm$ 0.30      & -11.29     & -13.84 & -8.75   \\
LR +A     & \tb{55.31} $\pm$ 0.32 & \tb{-11.72}& -12.72 & -10.71  & 63.00 $\pm$ 0.29      & -10.80     & -13.59 & -8.01   \\
\midrule                                                                                                                             
MLP       & 49.36 $\pm$ 0.29      & -16.78     & -8.95  & -24.61 & 60.85 $\pm$ 0.29      & -12.62     & -11.35 & -13.89  \\
MLP +S    & 54.46 $\pm$ 0.31      & -11.74     & -12.73 & -10.74 & 62.79 $\pm$ 0.31      & -10.77     & -12.61 & -8.80   \\
MLP +A    & 54.95 $\pm$ 0.30      & -11.84     & -12.81 & -10.87 & \tb{63.04} $\pm$ 0.30 & \tb{-10.66}& -12.55 & -8.77   \\
\bottomrule
\end{tabular}
}
\fi
\end{table}

\begin{table*}[t!]
\centering
\caption{Full ablation results on 200+5-way {\it tiered}-ImageNet}
\iflatexml
\begin{tabular}{c|cccc|cccc}
\toprule
          & \multicolumn{4}{c|}{1-shot}           & \multicolumn{4}{c}{5-shot} \\
          & Acc. $\ua$            & \D         & \Da    & \Db    & Acc. $\ua$            & \D         & \Da    & \Db     \\
\midrule                                                                                                              
LR        & 48.84 $\pm$ 0.23      & -10.44     & -11.65 & -9.24  & 62.08 $\pm$ 0.20      & -8.00      & -5.49  & -10.51  \\
LR +S     & 55.36 $\pm$ 0.32      & -6.88      & -7.21  & -6.55  & 65.53 $\pm$ 0.30      & -4.68      & -4.72  & -4.63   \\
LR +A     & 55.98 $\pm$ 0.32      & \tb{-6.07} & -6.64  & -5.51  & 65.58 $\pm$ 0.29      & \tb{-4.39} & -4.87  & -3.91   \\
\midrule                                                                                                                                           
MLP       & 41.22 $\pm$ 0.35      & -10.61     & -11.25 & -9.98  & 62.70 $\pm$ 0.31      & -7.44      & -6.05  & -8.82   \\
MLP +S    & \tb{56.16} $\pm$ 0.32 & -6.28      & -6.83  & -5.73  & \tb{65.80} $\pm$ 0.31 & -4.58      & -4.66  & -4.51   \\
MLP +A    & 56.11 $\pm$ 0.33      & 6.11       & -6.79  & -5.43  & 65.52 $\pm$ 0.31      & -4.48      & -4.91  & -4.05   \\
\bottomrule
\end{tabular}
\else
\resizebox{0.8\textwidth}{!}{
\begin{tabular}{c|cccc|cccc}
\toprule
          & \multicolumn{4}{c|}{1-shot}           & \multicolumn{4}{c}{5-shot} \\
          & Acc. $\ua$            & \D         & \Da    & \Db    & Acc. $\ua$            & \D         & \Da    & \Db     \\
\midrule                                                                                                              
LR        & 48.84 $\pm$ 0.23      & -10.44     & -11.65 & -9.24  & 62.08 $\pm$ 0.20      & -8.00      & -5.49  & -10.51  \\
LR +S     & 55.36 $\pm$ 0.32      & -6.88      & -7.21  & -6.55  & 65.53 $\pm$ 0.30      & -4.68      & -4.72  & -4.63   \\
LR +A     & 55.98 $\pm$ 0.32      & \tb{-6.07} & -6.64  & -5.51  & 65.58 $\pm$ 0.29      & \tb{-4.39} & -4.87  & -3.91   \\
\midrule                                                                                                                                           
MLP       & 41.22 $\pm$ 0.35      & -10.61     & -11.25 & -9.98  & 62.70 $\pm$ 0.31      & -7.44      & -6.05  & -8.82   \\
MLP +S    & \tb{56.16} $\pm$ 0.32 & -6.28      & -6.83  & -5.73  & \tb{65.80} $\pm$ 0.31 & -4.58      & -4.66  & -4.51   \\
MLP +A    & 56.11 $\pm$ 0.33      & 6.11       & -6.79  & -5.43  & 65.52 $\pm$ 0.31      & -4.48      & -4.91  & -4.05   \\
\bottomrule
\end{tabular}
}
\fi
\end{table*}

\section{Dataset Statistics}
In this section, we include more details on the datasets we used in our experiments.
% We include the dataset statistics in Table~\ref{tab:stats}. In \textit{mini}-ImageNet, we use the
% training set for both pretraining and meta-learning. For testing base class classification
% performance, we included the same val/test set as \citep{lwof}. Since the meta-training set is same
% as meta-learning, in each training episode, we masked out the 5 base classes in the base classifier,
% to ``pretend'' they are few-shot classes. In \textit{tiered}-ImageNet, we splits the original
% training set, Train-A and Train-B, for pretraining and meta-learning respectively.

\begin{table}[h]
\begin{small}
\caption{\textit{mini}-ImageNet and \textit{tiered}-ImageNet split statistics}
\vspace{-0.1in}
\label{tab:stats}
\begin{center}
\begin{tabular}{cc|crr|crr}
\toprule
&& \multicolumn{3}{c|}{\textit{mini}-ImageNet}& \multicolumn{3}{c}{\textit{tiered}-ImageNet} \\
Classes                & Purpose & Split         & N. Cls  & N. Img  & Split           & N. Cls   & N. Img \\
\midrule
\multirow{3}{*}{Base}  & Train   & Train-Train   & 64      & 38,400  & Train-A-Train   & 200      & 203,751   \\
                      & Val     & Train-Val     & 64      & 18,748  & Train-A-Val     & 200      & 25,460    \\
                      & Test    & Train-Test    & 64      & 19,200  & Train-A-Test    & 200      & 25,488    \\
\midrule
\multirow{3}{*}{Novel} & Train   & Train-Train   & 64      & 38,400  & Train-B         & 151      & 193,996   \\
                      & Val     & Val           & 16      & 9,600   & Val             & 97       & 124,261   \\
                      & Test    & Test          & 20      & 12,000  & Test            & 160      & 206,209   \\
\bottomrule
\end{tabular}
\end{center}
\end{small}
\vspace{-0.2in}
\end{table}

\subsection{Validation and testing splits for base classes}
In standard few-shot learning, meta-training, validation, and test set have disjoint sets of object
classes. However, in our incremental few-shot learning setting, to evaluate the model performance on
the base class predictions, additional splits of validation and test splits of the meta-training set
are required. Splits and dataset statistics are listed in Table~\ref{tab:stats}. For
\textit{mini}-ImageNet, \citet{lwof} released additional images for evaluating training set, namely
``Train-Val'' and ``Train-Test''. For \textit{tiered}-ImageNet, we split out $\approx$ 20\% of the
images for validation and testing of the base classes.

\subsection{Novel classes}
In \textit{mini}-ImageNet experiments, the same training set is used for both $\mathcal{D}_a$ and
$\mathcal{D}_b$. In order to pretend that the classes in the few-shot episode are novel, following
\citet{lwof}, we masked the base classes in $W_a$, which contains 64 base classes. In other words, we
essentially train for a 59+5 classification task. We found that under this setting, the progress
of meta-learning in the second stage is not very significant, since all classes have already been
seen before.

In \textit{tiered}-ImageNet experiments, to emulate the process of learning novel classes during the
second stage, we split the training classes into base classes (``Train-A'') with 200 classes and novel classes (``Train-B'') with 151 classes, just for meta-learning purpose.
During the first stage the classifier is trained using Train-A-Train data. In each meta-learning episode we sample few-shot examples from the novel classes (Train-B) and a query base set from Train-A-Val.  

{\bf 200 Base Classes (``Train-A''):}

{\tt n02128757, n02950826, n01694178, n01582220, n03075370, n01531178, n03947888, n03884397, n02883205, n03788195, n04141975, n02992529, n03954731, n03661043, n04606251, n03344393, n01847000, n03032252, n02128385, n04443257, n03394916, n01592084, n02398521, n01748264, n04355338, n02481823, n03146219, n02963159, n02123597, n01675722, n03637318, n04136333, n02002556, n02408429, n02415577, n02787622, n04008634, n02091831, n02488702, n04515003, n04370456, n02093256, n01693334, n02088466, n03495258, n02865351, n01688243, n02093428, n02410509, n02487347, n03249569, n03866082, n04479046, n02093754, n01687978, n04350905, n02488291, n02804610, n02094433, n03481172, n01689811, n04423845, n03476684, n04536866, n01751748, n02028035, n03770439, n04417672, n02988304, n03673027, n02492660, n03840681, n02011460, n03272010, n02089078, n03109150, n03424325, n02002724, n03857828, n02007558, n02096051, n01601694, n04273569, n02018207, n01756291, n04208210, n03447447, n02091467, n02089867, n02089973, n03777754, n04392985, n02125311, n02676566, n02092002, n02051845, n04153751, n02097209, n04376876, n02097298, n04371430, n03461385, n04540053, n04552348, n02097047, n02494079, n03457902, n02403003, n03781244, n02895154, n02422699, n04254680, n02672831, n02483362, n02690373, n02092339, n02879718, n02776631, n04141076, n03710721, n03658185, n01728920, n02009229, n03929855, n03721384, n03773504, n03649909, n04523525, n02088632, n04347754, n02058221, n02091635, n02094258, n01695060, n02486410, n03017168, n02910353, n03594734, n02095570, n03706229, n02791270, n02127052, n02009912, n03467068, n02094114, n03782006, n01558993, n03841143, n02825657, n03110669, n03877845, n02128925, n02091032, n03595614, n01735189, n04081281, n04328186, n03494278, n02841315, n03854065, n03498962, n04141327, n02951585, n02397096, n02123045, n02095889, n01532829, n02981792, n02097130, n04317175, n04311174, n03372029, n04229816, n02802426, n03980874, n02486261, n02006656, n02025239, n03967562, n03089624, n02129165, n01753488, n02124075, n02500267, n03544143, n02687172, n02391049, n02412080, n04118776, n03838899, n01580077, n04589890, n03188531, n03874599, n02843684, n02489166, n01855672, n04483307, n02096177, n02088364.}

{\bf 151 Novel Classes (``Train-B''):}

{\tt n03720891, n02090379, n03134739, n03584254, n02859443, n03617480, n01677366, n02490219, n02749479, n04044716, n03942813, n02692877, n01534433, n02708093, n03804744, n04162706, n04590129, n04356056, n01729322, n02091134, n03788365, n01739381, n02727426, n02396427, n03527444, n01682714, n03630383, n04591157, n02871525, n02096585, n02093991, n02013706, n04200800, n04090263, n02493793, n03529860, n02088238, n02992211, n03657121, n02492035, n03662601, n04127249, n03197337, n02056570, n04005630, n01537544, n02422106, n02130308, n03187595, n03028079, n02098413, n02098105, n02480855, n02437616, n02123159, n03803284, n02090622, n02012849, n01744401, n06785654, n04192698, n02027492, n02129604, n02090721, n02395406, n02794156, n01860187, n01740131, n02097658, n03220513, n04462240, n01737021, n04346328, n04487394, n03627232, n04023962, n03598930, n03000247, n04009552, n02123394, n01729977, n02037110, n01734418, n02417914, n02979186, n01530575, n03534580, n03447721, n04118538, n02951358, n01749939, n02033041, n04548280, n01755581, n03208938, n04154565, n02927161, n02484975, n03445777, n02840245, n02837789, n02437312, n04266014, n03347037, n04612504, n02497673, n03085013, n02098286, n03692522, n04147183, n01728572, n02483708, n04435653, n02480495, n01742172, n03452741, n03956157, n02667093, n04409515, n02096437, n01685808, n02799071, n02095314, n04325704, n02793495, n03891332, n02782093, n02018795, n03041632, n02097474, n03404251, n01560419, n02093647, n03196217, n03325584, n02493509, n04507155, n03970156, n02088094, n01692333, n01855032, n02017213, n02423022, n03095699, n04086273, n02096294, n03902125, n02892767, n02091244, n02093859, n02389026.}


\clearpage
\newpage
{
\setstretch{0.93}
\bibliography{ref}
\bibliographystyle{iclr2021_conference}
}

\setstretch{1.0}
\end{document}